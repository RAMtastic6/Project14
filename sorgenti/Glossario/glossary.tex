\documentclass[12pt, oneside]{article} 
\usepackage{amsmath, amsthm, amssymb, calrsfs, wasysym, verbatim, bbm, color, graphicx, geometry, fancyhdr, url, multirow, hyperref}
\usepackage[italian]{babel}

\geometry{tmargin=.75in, bmargin=.75in, lmargin=.75in, rmargin = .75in}


\author{RAMtastic6}

%Intestazione
\pagestyle{fancy}
\fancyhf{}
\fancyhead[R]{Gruppo 14 RAMtastic6\\ramtastic6@gmail.com}
\fancyfoot[C]{\\thepage}

\renewcommand{\headrulewidth}{0pt} 

% Intestazione documento
\begin{document}
% Salta la prima pagina per l'intestazione
\thispagestyle{empty}
\title{Glossario}
\maketitle
\begin{figure}[h]
  \centering
  \includegraphics[scale=0.3]{logo.png}
\end{figure}
\begin{center}
    email: ramtastic6@gmail.com
\end{center}

% Informazioni sul documento
\section*{Informazioni sul documento}
\begin{tabular}{ll}
Versione: & 0.1.0 \\
Redattori:  & Zambon M., Brotto D. \\
Verificatori: & Brotto D., Zaupa R. \\ 
Destinatari: & T. Vardanega, R. Cardin, Imola Informatica \\
Uso: & Esterno
\end{tabular}
\newpage

% Registro dei cambiamenti
\section*{Registro dei Cambiamenti - Changelog}
\begin{tabular}{|c|c|c|p{3cm}|p{6cm}|}
\hline
\textbf{Versione} & \textbf{Data} & \textbf{Autore} & \textbf{Verificatore} & \textbf{Dettaglio} \\
\hline
v.0.1.0 & 2024-03-20 & Brotto D. & Zaupa R. & Stilate le definizioni di tutti i termini, aggiunti alcuni termini ambigui come Prenotazione e Orinazione e rimosso Acquisizione \\
\hline
v.0.0.1 & 2024-01-19 & Zambon M. & Brotto D. & Creata struttura del glossario e aggiunti i termini \\
\hline
\end{tabular}
\newpage

% Sommario
\tableofcontents
\newpage
\section{Introduzione}
Il glossario ha lo scopo di raccogliere i termini tecnici usati nel corso del progetto, al fine di facilitare la comprensione della documentazione, sia per i membri del gruppo che per i lettori esterni. I riferimenti ai termini presenti nel glossario saranno evidenziati tramite una G maiuscola a pedice.\\
\newpage
\section{A} 
\subsection{Analisi Dei Requisiti} 
L'analisi dei requisiti è il processo di raccolta, documentazione e analisi delle esigenze e delle funzionalità di un sistema software. Questo processo coinvolge la comprensione approfondita dei requisiti dell'utente, delle necessità del business e delle specifiche tecniche per sviluppare un sistema che soddisfi le aspettative degli stakeholder. Gli obiettivi principali dell'analisi dei requisiti includono la definizione chiara e completa delle funzionalità del sistema, la specifica dei vincoli e delle restrizioni, e l'identificazione dei requisiti non funzionali come quelli relativi alle prestazioni, sicurezza e usabilità.
\subsection{Analisi Dei Rischi} 
L'analisi dei rischi è il processo di identificazione, valutazione e gestione dei potenziali problemi e delle minacce che potrebbero influenzare il successo di un progetto o di un'attività. Questo processo coinvolge la valutazione dei rischi potenziali, la determinazione della loro probabilità di accadere e dell'impatto che potrebbero avere sul progetto, nonché lo sviluppo di strategie per mitigare o gestire tali rischi. L'obiettivo dell'analisi dei rischi è identificare precocemente le potenziali problematiche e adottare misure preventive o correttive per ridurre al minimo gli impatti negativi sul progetto.
\subsection{API} 
Le API REST (Representational State Transfer) nell'ambito delle applicazioni web sono un insieme di principi architetturali che definiscono come le risorse Web devono essere definite e accessibili. Le API REST consentono alle applicazioni client di comunicare con un server Web in modo standardizzato utilizzando richieste HTTP per accedere e manipolare le risorse, come ad esempio dati o servizi.
\subsection{Applicazione Web Responsive} 
Un'applicazione web progettata per adattarsi automaticamente e fornire un'esperienza utente ottimale su una varietà di dispositivi e dimensioni dello schermo, inclusi computer desktop, laptop, tablet e smartphone, utilizzando tecniche di progettazione e sviluppo che consentono all'interfaccia utente di adattarsi dinamicamente alle dimensioni del dispositivo dell'utente.
\subsection{Architettura} 
Nell'ambito del design di un'applicazione, l'architettura si riferisce alla struttura complessiva dell'applicazione stessa, inclusi i suoi componenti principali, le relazioni tra di essi e il modo in cui sono organizzati per soddisfare gli obiettivi dell'applicazione. Questa struttura definisce come i diversi moduli dell'applicazione interagiscono tra loro per gestire dati, input utente, elaborazione e output. Un'architettura ben progettata mira a garantire la scalabilità, la manutenibilità e l'efficienza dell'applicazione nel tempo.
\subsection{Attore} 
Un attore è entità che interagisce con il sistema svolgendo delle attività, intesa sia come persona che come sistema terzo/esterno. Ciascuna entità è caratterizzata dall’insieme delle azioni che può compiere.

\newpage
\section*{B} 
\addcontentsline{toc}{section}{B}
\subsection*{Backend} 
 \addcontentsline{toc}{subsection}{Backend}
Il backend è la parte di un'applicazione software che gestisce la logica di business, l'accesso ai dati e le operazioni del server. Il backend include server, database, e API che permettono la comunicazione tra il frontend e il database. È responsabile dell'elaborazione delle richieste degli utenti, della gestione della sicurezza, dell'autenticazione, e dell'integrazione con altri servizi esterni.
\subsection*{Best practices} 
 \addcontentsline{toc}{subsection}{Best practices}
Le best practices nel contesto della produzione software sono l'applicazione di metodi e procedure che nel corso del tempo e attraverso l'esperienza pratica hanno dimostrato di essere le migliori in termini di efficienza ed efficacia nel raggiungimento degli obiettivi prefissati.
\subsection*{Branch} 
 \addcontentsline{toc}{subsection}{Branch}
Un ramo (branch) in GitHub è una versione separata del codice sorgente di un progetto software. Consente agli sviluppatori di lavorare su nuove funzionalità o correzioni di bug senza influenzare direttamente il codice principale (ramo principale o 'master'). I branch sono utilizzati per sviluppare in modo isolato, consentendo agli sviluppatori di sperimentare liberamente senza compromettere la stabilità del codice principale. Una volta completate le modifiche su un ramo, è possibile integrare (effettuare il 'merge') le modifiche nel ramo principale preferibilmente tramite una richiesta di pull (pull request) per incorporare le modifiche nel codice principale.

\newpage
\section{C} 
\subsection{Candidatura} 

\subsection{Capitolato} 

\subsection{Caso d'Uso} 

\subsection{Chanelog} 

\subsection{Codifica} 

\subsection{Consuntivo} 

\subsection{Controllo di Configurazione} 


\newpage
\section{D} 
\subsection{Design} 

\subsection{Diagramma dei Casi d'Uso} 

\subsection{Diario di Bordo} 

\subsection{Discord} 

\subsection{Documento Esterno} 

\subsection{Documento Interno} 


\newpage
\section*{E} 
\addcontentsline{toc}{section}{E}
\subsection*{Easy Meal} 
 \addcontentsline{toc}{subsection}{Easy Meal}
Easy Meal è un progetto proposto da Imola Informatica e sviluppato con l'obiettivo di semplificare il processo di prenotazione dei tavoli nei ristoranti, offrendo agli utenti la possibilità di pianificare in anticipo il proprio pasto e ordinarlo in modo collaborativo con altri utenti. Con Easy Meal, gli utenti possono prenotare facilmente un tavolo, selezionare le loro preferenze alimentari e ordinare il cibo desiderato prima del giorno desiderato, consentendo loro di godersi un'esperienza gastronomica senza problemi e senza attese. Grazie a questa piattaforma intuitiva e user-friendly, prenotare un tavolo e decidere cosa mangiare diventa un'esperienza semplice e piacevole per tutti i clienti dei ristoranti.
\subsection*{Eccezione} 
 \addcontentsline{toc}{subsection}{Eccezione}
Un'eccezione è un'anomalia o evento imprevisto durante l'esecuzione di un programma che interrompe il flusso normale di esecuzione. Può essere causata da errori di programmazione, input non validi o condizioni impreviste. Richiede una gestione appropriata per mantenere la stabilità del software.
\subsection*{Editor} 
 \addcontentsline{toc}{subsection}{Editor}
Un editor è un'applicazione software utilizzata per creare, modificare e gestire file di testo o codice sorgente. Fornisce funzionalità come formattazione del testo, ricerca e sostituzione, e strumenti per la gestione dei progetti.
\subsection*{Endpoint} 
 \addcontentsline{toc}{subsection}{Endpoint}
L'endpoint è un URL specifico esposto da un'API (Application Programming Interface) che rappresenta un punto di ingresso per la comunicazione tra un client e un server. Gli endpoint sono utilizzati per eseguire operazioni come il recupero, l'invio, l'aggiornamento o la cancellazione di dati. Ogni endpoint è associato a un'operazione HTTP e consente ai client di interagire con le risorse del server in modo strutturato e sicuro.

\newpage
\section{F} 
\subsection{Feature} 
donna
\subsection{Feedback} 
donna
\subsection{Fornitura} 
donna
\subsection{Framework} 
donna
\subsection{Funzionalitá} 
donna

\newpage
\section*{G} 
\addcontentsline{toc}{section}{G}
\subsection*{Gateway} 
 \addcontentsline{toc}{subsection}{Gateway}
Il gateway è un punto di accesso che funge da intermediario tra diverse reti o sistemi, permettendo la comunicazione e il trasferimento di dati tra di essi. Nel contesto delle applicazioni software, un gateway può gestire richieste da client e inoltrarle ai servizi appropriati, effettuare traduzioni di protocolli, gestire autenticazione e autorizzazione, e applicare politiche di sicurezza.
\subsection*{Git} 
 \addcontentsline{toc}{subsection}{Git}
Git è un sistema di controllo di versione distribuito utilizzato principalmente nello sviluppo software. Consente agli sviluppatori di tenere traccia delle modifiche apportate al codice sorgente nel corso del tempo, coordinare il lavoro con altri membri del team e gestire le diverse versioni di un progetto in modo efficiente. Git consente di creare repository (archivi di file sorgente e cronologia delle modifiche) sia localmente che su server remoti, facilitando la collaborazione tra sviluppatori distribuiti geograficamente. È ampiamente utilizzato nell'industria del software per gestire progetti di qualsiasi dimensione e complessità.
\subsection*{Gitflow} 
 \addcontentsline{toc}{subsection}{Gitflow}
GitFlow è un modello di branching e di workflow basato su Git, progettato per gestire progetti software con un ciclo di sviluppo complesso e ramificato. Questo approccio fornisce una struttura chiara per la gestione delle diverse fasi dello sviluppo, inclusi i rilasci, le correzioni di bug e le nuove funzionalità. GitFlow prevede l'utilizzo di due branch principali, 'master' e 'develop', oltre a una serie di branch temporanei per le funzionalità in sviluppo denominati 'feature branch', i bugfix e i rilasci. Questo modello facilita il lavoro in team e la collaborazione su progetti complessi, consentendo di mantenere una storia di versionamento pulita e organizzata.
\subsection*{Github} 
 \addcontentsline{toc}{subsection}{Github}
GitHub è una piattaforma di hosting di codice sorgente basata su Git, che fornisce strumenti per la gestione dei repository, la collaborazione tra sviluppatori e il controllo delle versioni dei progetti software. Su GitHub, gli sviluppatori possono caricare i propri progetti, tenere traccia delle modifiche tramite commit, creare e gestire branch, collaborare con altri utenti tramite problemi e richieste di pull, e distribuire le proprie applicazioni. È ampiamente utilizzato sia da sviluppatori individuali che da team di sviluppo per lo sviluppo di software open source e progetti privati.

\newpage
\section*{H} 
\addcontentsline{toc}{section}{H}

\newpage
\section*{I} 
\subsection*{Imola Informatica} 
Imola Informatica, proponente del progetto Easy Meal, è una società indipendente di consulenza IT. Tutto ciò che riguarda il mondo dell’information technology la riguarda, gli interessa e appassiona. La società in gioco ogni volta in cui una azienda pubblica o privata vuole migliorare i propri servizi, innovare i propri processi di lavoro e gli approcci di management per cogliere le opportunità business offerte dalla trasformazione digitale. Sono a servizio dei principali gruppi finanziari e assicurativi e ogni giorno sono a fianco di grandi aziende e piccole startup nel gestire il cambiamento tecnologico e culturale.
\subsection*{Infrastruttura} 
L'infrastruttura rappresenta l'insieme degli elementi fisici e tecnologici necessari per il funzionamento di un sistema o di un'applicazione. Questi elementi includono hardware, software, reti, server, database e tutte le risorse e le tecnologie utilizzate per supportare le operazioni di un'organizzazione o di un progetto. Nell'ambito del progetto Easy Meal, l'infrastruttura potrebbe comprendere i server per l'hosting dell'applicazione, i database per la gestione dei dati dei ristoranti e degli utenti, nonché la rete e gli strumenti di sicurezza necessari per garantire il corretto funzionamento e la protezione del sistema.
\subsection*{Ingegneria del software} 
L'ingegneria del software è un insieme di principi e pratiche utilizzati per progettare, sviluppare, mantenere, testare e valutare il software per computer. Le relative tecniche vengono impiegate per guidare il processo di sviluppo del software, che comprende la definizione, l'implementazione, la valutazione, la misurazione, la gestione, il cambiamento e il miglioramento del ciclo di vita del software. Questa disciplina pone un forte accento sulla gestione della configurazione del software, che implica il controllo sistematico delle modifiche alla configurazione e il mantenimento dell'integrità e della tracciabilità della configurazione e del codice durante tutto il ciclo di vita del sistema, mediante l'uso del versioning del software.
\subsection*{Inspection} 
L'Inspection è una tipologia di analisi statica che consiste nella revisione di parti specifiche del codice e della documentazione attraverso liste di controllo (checklist). Viene eseguita quando si ha già un’idea di dove potrebbero esserci possibili problemi, al fine di intervenire tempestivamente e sistematicamente.

\newpage
\section*{J} 
\addcontentsline{toc}{section}{J}
\subsection*{Javascript} 
 \addcontentsline{toc}{subsection}{Javascript}
Un linguaggio di programmazione ad alto livello, interpretato e orientato agli oggetti, comunemente utilizzato per sviluppare applicazioni web interattive e dinamiche. Creato originariamente per essere eseguito nei browser web per migliorare l'esperienza utente sul web, JavaScript è diventato uno dei linguaggi di programmazione più diffusi e versatili al mondo.
\subsection*{Jest} 
 \addcontentsline{toc}{subsection}{Jest}
Jest è un framework di testing open-source per JavaScript utilizzato per testare applicazioni e componenti React, Node.js e JavaScript in generale. Esso è dotato di una serie di funzionalità avanzate per facilitare lo sviluppo dei test (come la reportistica sulla code-coverage di un progetto) e migliorare la qualità del software.
\subsection*{Jira} 
 \addcontentsline{toc}{subsection}{Jira}
Jira è una piattaforma di gestione del lavoro e dei progetti utilizzata per tracciare le attività, assegnare compiti, pianificare progetti e collaborare tra membri del team.
\subsection*{JSON} 
 \addcontentsline{toc}{subsection}{JSON}
JSON (JavaScript Object Notation) è un formato di dati leggero e basato su testo utilizzato per rappresentare dati strutturati, comunemente utilizzato per lo scambio di dati tra un server e un client web e come formato di memorizzazione dei dati (da servizi come MongoDB).
\subsection*{JSX} 
 \addcontentsline{toc}{subsection}{JSX}
JSX è un'estensione di sintassi per JavaScript che consente di scrivere markup HTML all'interno del codice JavaScript ed è comunemente utilizzato con React.

\newpage
\section*{K} 
\addcontentsline{toc}{section}{K}

\newpage
\section*{L} 
\addcontentsline{toc}{section}{L}
\subsection*{LaTeX} 
 \addcontentsline{toc}{subsection}{LaTeX}
LaTeX viene spesso utilizzato per redigere documenti tecnici, come specifiche dei requisiti, documenti di progetto, manuali utente e rapporti di testing. La sua struttura modulare, la gestione avanzata della formattazione e la facilità nel gestire equazioni matematiche e tabelle lo rendono una scelta popolare tra gli sviluppatori e i professionisti del settore. LaTeX consente inoltre di mantenere una formattazione uniforme e di automatizzare alcune operazioni, rendendo più efficiente il processo di documentazione del progetto.
\subsection*{Layer di Persistenza} 
 \addcontentsline{toc}{subsection}{Layer di Persistenza}
Il layer di persistenza, è una parte dell'architettura software responsabile della gestione della persistenza dei dati. Questo strato si occupa di interagire con il sistema di memorizzazione permanente dei dati, come ad esempio un database relazionale, per memorizzare, recuperare, aggiornare ed eliminare i dati.
\subsection*{Logica di Business} 
 \addcontentsline{toc}{subsection}{Logica di Business}
La logica di business si riferisce al componente dell'applicazione che si occupa dell'implementazione delle regole di business specifiche del dominio applicativo. È spesso isolata in uno strato dedicato, noto come Service layer, che si trova tra il livello di presentazione (come l'interfaccia utente) e il livello di persistenza dei dati (come il database).

\newpage
\section*{M} 
\subsection*{Manuale Utente} 
Il Manuale Utente è un documento che fornisce istruzioni dettagliate su come utilizzare un determinato prodotto software o sistema. È destinato agli utenti finali e fornisce indicazioni chiare e concise su come eseguire le diverse operazioni, navigare nell'interfaccia utente e sfruttare le funzionalità offerte dal software. Il manuale utente può includere tutorial, guide passo-passo, spiegazioni delle funzionalità e istruzioni per la risoluzione dei problemi comuni. L'obiettivo principale del manuale utente è quello di consentire agli utenti di utilizzare il software in modo efficace ed efficiente, migliorando così l'esperienza complessiva dell'utente.
\subsection*{Milestone} 
Una milestone è un punto di riferimento significativo o un obiettivo importante all'interno di un progetto, utilizzato per misurare il progresso e il raggiungimento di determinati traguardi. Le milestone sono solitamente associate a date specifiche o a completamenti di specifiche attività o fasi di un progetto. Servono come punti di controllo per valutare se il progetto sta procedendo secondo i piani e per identificare eventuali ritardi o problemi. Le milestone possono essere utilizzate anche per comunicare i progressi del progetto alle parti interessate e per stabilire scadenze chiare e tangibili per il completamento delle attività.
\subsection*{Modello a V} 
Il Modello a V è una rappresentazione grafica del ciclo di vita di un progetto di sviluppo software. Esso mostra in maniera dettagliata le relazioni fra le fasi di analisi e pregettazione, la  produzione e il testing e ciò permette di minimizzare i rischi di progetto e di migliorare e garantire la qualità del prodotto.

\newpage
\section*{N} 
\addcontentsline{toc}{section}{N}
\subsection*{Nest.js} 
 \addcontentsline{toc}{subsection}{Nest.js}
Nest.js è un framework per lo sviluppo di applicazioni server-side basato su Node.js. Le caratteristiche offerte sono: dependency injection (controller), l'uso di decorators per definire i controller e come essi gestiscono le varie route e la modularità. Esso rende possibile anche connessioni basate su altri protocolli, come Websocket, gRPC e altri ancora.
\subsection*{Next.js} 
 \addcontentsline{toc}{subsection}{Next.js}
Next.js è un framework di sviluppo web per React, che consente di creare facilmente applicazioni web con funzionalità avanzate come il server-side rendering (SSR), il rendering statico e la generazione di pagine dinamiche.
\subsection*{Node.js} 
 \addcontentsline{toc}{subsection}{Node.js}
Node.js è un runtime JavaScript open-source basato sul motore JavaScript V8 di Google Chrome; è progettato per eseguire codice JavaScript lato server, consentendo agli sviluppatori di creare applicazioni web e servizi di rete scalabili e performanti.
\subsection*{Norme di Progetto} 
 \addcontentsline{toc}{subsection}{Norme di Progetto}
Nel documento Norme di Progetto sono riportate tutte le linee guida tecniche e procedurali stabilite per regolare l'esecuzione e il controllo di un progetto. Questo documento definisce standard e procedure relative all'organizzazione del lavoro, alla metodologia di sviluppo, alla gestione dei documenti, delle configurazioni e dei rischi, nonché alla pianificazione e al monitoraggio delle attività del progetto.
\subsection*{NPM} 
 \addcontentsline{toc}{subsection}{NPM}
NPM è il gestore di pacchetti predefinito per Node.js, utilizzato per installare, condividere e gestire le dipendenze del progetto JavaScript. NPM consente agli sviluppatori di scaricare e installare librerie e strumenti di terze parti, oltre a pubblicare e condividere i propri pacchetti. Con NPM, è possibile gestire le versioni delle dipendenze e configurare script per automatizzare attività di sviluppo comuni, rendendo più efficiente la gestione del ciclo di vita del progetto.

\newpage
\section{O} 
\subsection{Overleaf} 
donna

\newpage
\section{P} 
\subsection{PB} 
donna
\subsection{PDF} 
fobia degli aracni
\subsection{Piano di Progetto} 
fobia degli aracni
\subsection{Piano di Qualifica} 
fobia degli aracni
\subsection{Poc} 
fobia degli aracni
\subsection{Posta elettronica} 
fobia degli aracni
\subsection{Postcondizione} 
fobia degli aracni
\subsection{Precondizione} 
fobia degli aracni
\subsection{Prenotazione} 
fase nella quale
\subsection{Postcondizione} 
fobia degli aracni
\subsection{Preventivo} 
fobia degli aracni
\subsection{Principio di miglioramento continuo} 
fobia degli aracni
\subsection{Processo} 
fobia degli aracni
\subsection{Processo primario} 
fobia degli aracni
\subsection{Processo di supporto} 
fobia degli aracni
\subsection{Processo organizzativo} 
fobia degli aracni
\subsection{Progettazione architetturale} 
fobia degli aracni
\subsection{Progettazione dettagliata} 
fobia degli aracni
\subsection{Push} 
fobia degli aracni

\newpage
\section*{Q} 
\subsection*{Qualità di processo} 
La qualità di processo si riferisce alla misura in cui un processo software è ben pianificato, gestito ed eseguito per raggiungere gli obiettivi di sviluppo del software in modo efficiente ed efficace.
\subsection*{Qualità di prodotto} 
La qualità di prodotto si riferisce alla misura in cui un prodotto software soddisfa i requisiti specificati e le aspettative degli utenti, fornendo funzionalità affidabili, sicure e adatte all'uso previsto.

\newpage
\section{R} 
\subsection{RAMTASTIC6} 
donna
\subsection{Redattore} 
fobia degli aracni
\subsection{Repository} 
fobia degli aracni
\subsection{Requisito} 
fobia degli aracni
\subsection{Requisiti funzionali desiderabili} 
fobia degli aracni
\subsection{Requisiti funzionali obbligatori} 
fobia degli aracni
\subsection{Riferimento} 
fobia degli aracni
\subsection{Riferimenti normativi} 
fobia degli aracni
\subsection{Riferimenti informativi} 
fobia degli aracni
\subsection{Rilascio} 
fobia degli aracni
\subsection{Rischio} 
fobia degli aracni
\subsection{Rischio organizzativo} 
fobia degli aracni
\subsection{Rischio relativo al prodotto} 
fobia degli aracni
\subsection{Rischio tecnologico} 
fobia degli aracni
\subsection{Progettazione architetturale} 
fobia degli aracni
\subsection{Risorsa Umana} 
fobia degli aracni
\subsection{RTB} 
fobia degli aracni

\newpage
\section*{S} 
\addcontentsline{toc}{section}{S}
\subsection*{Scenario} 
 \addcontentsline{toc}{subsection}{Scenario}
Nel documento di Analisi dei Requisiti, uno scenario rappresenta una sequenza di azioni o eventi che descrivono un possibile flusso di interazione tra l'utente e il sistema. Gli scenari sono utilizzati per comprendere e illustrare come il sistema dovrebbe comportarsi in determinate situazioni o sotto determinate condizioni. Ogni scenario include una serie di passi che descrivono le azioni compiute dagli utenti e le risposte del sistema, consentendo di visualizzare in modo chiaro e dettagliato il comportamento previsto del software. Gli scenari possono essere utilizzati per identificare requisiti funzionali, individuare casi d'uso e validare il sistema durante lo sviluppo e il testing. Lo scenario principale rappresenta il flusso di interazione tipico tra l'utente e il sistema per raggiungere l'obiettivo principale del caso d'uso o della funzionalità in esame. Esso descrive la sequenza di passi che si verificano nella maggior parte dei casi. Gli scenari alternativi, al contrario, rappresentano situazioni non standard o eccezionali che possono verificarsi durante l'esecuzione del caso d'uso. Questi scenari descrivono le deviazioni dal flusso principale causate da input inaspettati, condizioni particolari o errori. Gli scenari alternativi forniscono istruzioni su come gestire tali situazioni e ripristinare il flusso normale delle operazioni.
\subsection*{Scrum} 
 \addcontentsline{toc}{subsection}{Scrum}
Scrum è un framework Agile utilizzato per lo sviluppo di prodotti complessi. Si basa su un approccio iterativo e incrementale, in cui il lavoro viene diviso in intervalli di tempo definiti chiamati sprint. Durante uno sprint, il team si impegna a completare un insieme di attività prioritarie stabilite durante la pianificazione dello sprint. Gli sprint sono seguiti da sessioni di revisione per valutare il lavoro svolto e identificare eventuali miglioramenti. 
\subsection*{Server action} 
 \addcontentsline{toc}{subsection}{Server action}
Funzionalità introdotta dal framework Next.js nella versione 14; consente di definire funzioni server-side che possono essere invocate dai componenti React nel lato client.
\subsection*{Server-side} 
 \addcontentsline{toc}{subsection}{Server-side}
Nell'ambito dello sviluppo software si riferisce a processi, logica e risorse che operano sul lato del server in un'applicazione client-server; in altre parole, "server-side" si riferisce a tutto ciò che avviene sul server, inclusi calcoli, elaborazioni dei dati, accesso al database e altro ancora.
\subsection*{Service layer} 
 \addcontentsline{toc}{subsection}{Service layer}
Il service Layer è lo strato che rappresenta la logica di business dell'applicazione e si trova tra il controller e il livello di accesso ai dati. Il suo compito principale è implementare la logica aziendale dell'applicazione, fornendo servizi e operazioni che il controller richiama per eseguire azioni complesse.
\subsection*{Sistema} 
 \addcontentsline{toc}{subsection}{Sistema}
Il Sistema nel documento di Analisi dei Requisiti si riferisce all'insieme di componenti software, hardware e di sistema che collaborano per fornire le funzionalità richieste dal prodotto. Il sistema può comprendere diverse parti, come applicazioni software, database e server.
\subsection*{Socket.IO} 
 \addcontentsline{toc}{subsection}{Socket.IO}
Socket.IO è una libreria JavaScript che permette di implementare una comunicazione in tempo reale, bidirezionale e event-driven tra client e server attraverso il web. Utilizza WebSocket, quando disponibile, per una comunicazione diretta e offre una serie di alternative, come il polling basato su HTTP, per garantire la compatibilità con una vasta gamma di browser e dispositivi. Socket.IO semplifica lo sviluppo di applicazioni web interattive che richiedono aggiornamenti in tempo reale, come chat in tempo reale, giochi multiplayer, monitoraggio dei dati in tempo reale e altro ancora.
\subsection*{Software} 
 \addcontentsline{toc}{subsection}{Software}
Il termine "Software" si riferisce all'insieme di programmi, procedure e documentazione associata utilizzati su un sistema informatico per eseguire determinate operazioni o risolvere specifici problemi. Questo include sia il codice sorgente dei programmi che il software precompilato eseguibile. Il software può essere progettato per scopi diversi, come gestione dati, elaborazione informatica, comunicazione, intrattenimento e molti altri. In un contesto di sviluppo software, il termine "software" si riferisce al prodotto finale ottenuto dal processo di sviluppo, che può essere installato e utilizzato dagli utenti per scopi specifici. 
\subsection*{Sottocasi d'uso} 
 \addcontentsline{toc}{subsection}{Sottocasi d'uso}
Il termine "Sottocaso d'uso" si riferisce a un'istanza specifica di un caso d'uso più ampio. Mentre un caso d'uso descrive un'interazione completa tra gli attori e il sistema per raggiungere un obiettivo specifico, un sottocaso d'uso dettaglia uno specifico scenario o flusso di lavoro all'interno di quel caso d'uso più ampio. I sottocasi d'uso vengono utilizzati per suddividere i casi d'uso complessi in unità gestibili e per fornire una descrizione più dettagliata delle azioni e dei passaggi coinvolti in un'interazione specifica con il sistema. 
\subsection*{Specifica tecnica} 
 \addcontentsline{toc}{subsection}{Specifica tecnica}
La Specifica tecnica è la descrizione dettagliata della progettazione tecnica del software. Questa descrizione include informazioni sulle decisioni architetturali prese, i design pattern utilizzati, la struttura dei componenti software e le interazioni tra di essi, le tecnologie impiegate e altre specifiche rilevanti per l'implementazione del sistema. La specifica tecnica fornisce una guida chiara agli sviluppatori durante la fase di codifica e contribuisce a garantire coerenza e qualità nel prodotto finale. 
\subsection*{Sprint} 
 \addcontentsline{toc}{subsection}{Sprint}
Lo Sprint è una fase di sviluppo iterativa e incrementale nel framework Agile, come ad esempio Scrum. Durante uno Sprint, il team di sviluppo lavora su una serie di attività concordate all'inizio dello Sprint, con l'obiettivo di consegnare un incremento di funzionalità completamente funzionante e potenzialmente consegnabile al termine del periodo prestabilito, generalmente di durata fissa, solitamente da una a quattro settimane. Gli Sprint offrono una struttura chiara per la pianificazione, l'esecuzione e il monitoraggio del lavoro del team, consentendo un rapido adattamento ai cambiamenti dei requisiti o delle priorità. 
\subsection*{SQL} 
 \addcontentsline{toc}{subsection}{SQL}
SQL (Structured Query Language) è un linguaggio di programmazione dichiarativo utilizzato per gestire e manipolare database relazionali. È progettato per consentire agli utenti di interrogare, aggiornare, inserire e cancellare dati da un database in modo efficiente e standardizzato.
\subsection*{SSR} 
 \addcontentsline{toc}{subsection}{SSR}
SSR (Server Side Rendering) è una tecnica di rendering delle pagine web in cui il contenuto della pagina viene generato e inviato al client dal server, anziché essere costruito nel browser utilizzando JavaScript. Con SSR, il server elabora le richieste dell'utente, recupera i dati necessari e rende la pagina HTML completa prima di inviarla al browser.
\subsection*{Strumento} 
 \addcontentsline{toc}{subsection}{Strumento}
Uno strumento nell'ambito dello sviluppo del software è un'applicazione, un programma o una risorsa che supporta specifiche attività o processi nel ciclo di vita del software. Gli strumenti possono includere ambienti di sviluppo integrati (IDE), sistemi di gestione di progetto come Jira, framework di test come Selenium, sistemi di controllo versione come Git, e molti altri. L'uso degli strumenti giusti può migliorare l'efficienza e la qualità del processo di sviluppo del software. 

\newpage
\section{T} 
\subsection{Tecnologia} 
La "tecnologia" nel contesto dello sviluppo del software si riferisce agli strumenti, ai linguaggi di programmazione, ai framework e ad altre risorse utilizzate per progettare, sviluppare e implementare soluzioni software. Questo termine può anche comprendere hardware, piattaforme di hosting, servizi cloud e altri componenti tecnologici utilizzati nel processo di sviluppo e distribuzione del software. La scelta delle tecnologie giuste può influenzare significativamente le prestazioni, la scalabilità, la manutenibilità e altre caratteristiche del prodotto software. 
\subsection{Telegram} 
Telegram è un'applicazione di messaggistica istantanea e di comunicazione. Consente agli utenti di scambiare messaggi di testo, foto, video, file e altro ancora tramite Internet. Telegram offre anche funzionalità come chat di gruppo, chiamate vocali e video, canali pubblici, bot automatizzati e crittografia end-to-end per garantire la privacy e la sicurezza delle comunicazioni. È stato utilizzato dal gruppo per la messaggistica istantanea e per fissare meeting di progetto anche con l’uso di sondaggi a scelta multipla e variabile.
\subsection{Template} 
Un template, in ambito documentale, è un modello predefinito che fornisce una struttura organizzativa e visiva per la creazione di documenti. Questi modelli includono solitamente sezioni già definite, formattazione standard, stili di testo preimpostati e elementi grafici coerenti (usati ad esempio nei powerpoint dei diari di bordo e per le prime pagine di ogni documento, ovvero copertina e indice). L'uso di un template consente di uniformare l'aspetto e la struttura dei documenti prodotti da diversi autori, facilitando la creazione e la lettura dei contenuti. 
\subsection{Test} 
Il termine "test" si riferisce a un'attività mirata a valutare o verificare le funzionalità o le prestazioni di un sistema, un'applicazione o un componente software. I test possono essere condotti per individuare difetti, verificare il corretto funzionamento di una funzionalità o garantire il rispetto dei requisiti specificati. 
\subsection{Ticket} 
Un "ticket" in Jira è un elemento di tracciamento utilizzato per registrare e monitorare il lavoro in un progetto software. Ogni ticket rappresenta un'attività specifica da completare, come risolvere un bug, implementare una nuova funzionalità o completare un compito di manutenzione. I ticket contengono informazioni dettagliate sul lavoro da svolgere, inclusi i requisiti, le assegnazioni, le scadenze e lo stato di avanzamento. Possono essere assegnati a membri del team e seguiti nel tempo per garantire che vengano completati in modo efficace e tempestivo. 
\subsection{Trigger} 
Il "trigger" nell'analisi dei requisiti si riferisce a un evento o una condizione che innesca l'avvio di un determinato caso d'uso o processo all'interno del sistema software. È un elemento chiave per comprendere quando e come un particolare comportamento o funzionalità deve essere attivato nell'applicazione. I trigger possono essere attivati da azioni degli utenti, come clic su pulsanti o input di dati, o da eventi interni al sistema, come arrivo di notifiche o cambiamenti di stato. Identificare correttamente i trigger è fondamentale per definire in modo completo e accurato il comportamento del sistema. 

\newpage
\section{U} 
\subsection{UML} 
UML, acronimo di Unified Modeling Language, � un linguaggio di modellazione standardizzato utilizzato per descrivere, progettare e documentare sistemi software. Fornisce una serie di diagrammi grafici che consentono agli sviluppatori di visualizzare diversi aspetti di un sistema, inclusi struttura, comportamento, interazioni e flussi di dati. 

\newpage
\section{V} 
\subsection{Verbale} 
donna
\subsection{Verbale esterno} 
donna
\subsection{Verbale interno} 
donna
\subsection{Versionamento} 
donna

\newpage
\section{W} 
\subsection{Walkthrough} 
Il Walkthrough è una tipologia di analisi statica che coinvolge sia il verificatore che l’autore del prodotto e che consiste nella revisione nel suo complesso del codice e della documentazione forniti, seguita da una discussione degli eventuali problemi individuati.
\subsection{Way of Working} 
Il Way of Working è un insieme di processi operativi, regole e specifiche che definiscono il modo di lavorare del team e come vengono gestiti i processi all'interno del progetto. Un Way of Working efficace fornisce una struttura chiara e coesa per il team, promuove la coerenza, la trasparenza e l'efficienza nelle operazioni quotidiane, contribuendo così al successo complessivo del progetto.

\newpage
\section*{X} 
\addcontentsline{toc}{section}{X}

\newpage
\section*{Y} 
\addcontentsline{toc}{section}{Y}

\newpage
\section*{Z} 
\addcontentsline{toc}{section}{Z}

\newpage


\end{document}
