\section{D} 
\subsection{Design} 
Il design nel contesto dello sviluppo del software è la fase in cui vengono definiti l'architettura, la struttura e le specifiche dettagliate del sistema software da creare. Questo processo include la progettazione dell'interfaccia utente, la scelta delle tecnologie e dei linguaggi di programmazione, nonché la pianificazione dell'implementazione del sistema. È una fase cruciale che assicura un'organizzazione efficiente e logica delle funzionalità del software, garantendo la sua scalabilità, manutenibilità e facilità d'uso.
\subsection{Diagramma dei casi d'uso} 
Il diagramma dei casi d'uso è una rappresentazione grafica delle interazioni tra gli attori esterni al sistema software e il sistema software stesso. È composto da attori, casi d'uso e le relazioni tra di essi, e fornisce una panoramica visiva delle funzionalità offerte dal sistema e dei suoi utilizzi da parte degli attori. Questo strumento aiuta a comprendere i requisiti funzionali del sistema, le interazioni tra gli utenti e il sistema stesso, e fornisce una base per la progettazione e lo sviluppo del software.
\subsection{Diario di Bordo} 
Il diario di bordo è un documento utilizzato per registrare in modo sistematico le attività svolte, le decisioni prese, gli eventi significativi e le osservazioni durante lo sviluppo di un progetto. Serve come registro dettagliato del lavoro svolto, delle sfide affrontate e delle soluzioni adottate nel corso del tempo. Il diario di bordo aiuta a tracciare il progresso del progetto, a monitorare le scadenze, a identificare eventuali problemi e a documentare le lezioni apprese per migliorare i processi futuri.
\subsection{Discord} 
Discord è una piattaforma di comunicazione vocale e testuale progettata per la collaborazione remota, la formazione online e la comunicazione tra gruppi di persone. Offre funzionalità come chat di testo e vocali, canali organizzati per argomento, server personalizzabili, integrazioni con altre applicazioni e molto altro. Discord è ampiamente apprezzato per la sua facilità d'uso, la scalabilità e la ricchezza di funzionalità, rendendolo una scelta popolare per la comunicazione online.
\subsection{Documento Esterno} 
Un documento esterno è un'opera o un insieme di documenti che sono destinati a essere condivisi con terzi al di fuori del gruppo di sviluppo o dell'organizzazione. Questi documenti possono includere proposte di progetto, rapporti di avanzamento, documenti di requisiti, manuali utente, documenti di consegna o qualsiasi altra documentazione destinata ad aziende proponenti, clienti, i professori o altri stakeholder esterni al processo di sviluppo. La creazione e la gestione accurata dei documenti esterni sono cruciali per garantire una comunicazione chiara, trasparente ed efficace con gli interessati esterni al progetto.
\subsection{Documento Interno} 
Un documento interno è un file o un insieme di documenti utilizzati all'interno del gruppo di sviluppo o dell'organizzazione per scopi di comunicazione, pianificazione, gestione o documentazione. Questi documenti possono includere specifiche tecniche, documenti di progettazione, diagrammi, report di progresso, verbali delle riunioni, linee guida interne o qualsiasi altra documentazione destinata esclusivamente all'uso all'interno del team di lavoro. La creazione e la gestione efficace dei documenti interni sono fondamentali per garantire una collaborazione efficiente, un flusso di lavoro organizzato e una condivisione accurata delle informazioni all'interno del gruppo.
