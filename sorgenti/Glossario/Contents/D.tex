\section*{D} 
\addcontentsline{toc}{section}{D}
\subsection*{Data Access Layer} 
 \addcontentsline{toc}{subsection}{Data Access Layer}
Questo strato dell'architettura software è responsabile della gestione dell'accesso ai dati all'interno del sistema. È incaricato di comunicare con le fonti di dati sottostanti, come database, file di testo o servizi web, per recuperare, aggiornare o eliminare informazioni. Il livello di accesso ai dati astrae i dettagli di implementazione della persistenza dei dati dal resto dell'applicazione, consentendo alle altre parti del sistema di interagire con i dati.
\subsection*{Deployment} 
 \addcontentsline{toc}{subsection}{Deployment}
La fase di deployment di un'applicazione è il processo di distribuzione e messa in produzione dell'applicazione stessa in un ambiente operativo o una piattaforma di hosting in modo che sia accessibile agli utenti finali. Questa fase segue il completamento dello sviluppo e dei test dell'applicazione; in questa fase, l'applicazione viene installata e configurata in un ambiente di produzione, che può essere un server web, una piattaforma cloud o qualsiasi altro sistema in grado di ospitare e gestire l'applicazione.
\subsection*{Design} 
 \addcontentsline{toc}{subsection}{Design}
Il design nel contesto dello sviluppo del software è la fase in cui vengono definiti l'architettura, la struttura e le specifiche dettagliate del sistema software da creare. Questo processo include la progettazione dell'interfaccia utente, la scelta delle tecnologie e dei linguaggi di programmazione, nonché la pianificazione dell'implementazione del sistema. È una fase cruciale che assicura un'organizzazione efficiente e logica delle funzionalità del software, garantendo la sua scalabilità, manutenibilità e facilità d'uso.
\subsection*{Diagramma dei casi d'uso} 
 \addcontentsline{toc}{subsection}{Diagramma dei casi d'uso}
Il diagramma dei casi d'uso è una rappresentazione grafica delle interazioni tra gli attori esterni al sistema software e il sistema software stesso. È composto da attori, casi d'uso e le relazioni tra di essi, e fornisce una panoramica visiva delle funzionalità offerte dal sistema e dei suoi utilizzi da parte degli attori. Questo strumento aiuta a comprendere i requisiti funzionali del sistema, le interazioni tra gli utenti e il sistema stesso, e fornisce una base per la progettazione e lo sviluppo del software.
\subsection*{Diario di Bordo} 
 \addcontentsline{toc}{subsection}{Diario di Bordo}
Il diario di bordo è un documento utilizzato per registrare in modo sistematico le attività svolte, le decisioni prese, gli eventi significativi e le osservazioni durante lo sviluppo di un progetto. Serve come registro dettagliato del lavoro svolto, delle sfide affrontate e delle soluzioni adottate nel corso del tempo. Il diario di bordo aiuta a tracciare il progresso del progetto, a monitorare le scadenze, a identificare eventuali problemi e a documentare le lezioni apprese per migliorare i processi futuri.
\subsection*{Discord} 
 \addcontentsline{toc}{subsection}{Discord}
Discord è una piattaforma di comunicazione vocale e testuale progettata per la collaborazione remota, la formazione online e la comunicazione tra gruppi di persone. Offre funzionalità come chat di testo e vocali, canali organizzati per argomento, server personalizzabili, integrazioni con altre applicazioni e molto altro. Discord è ampiamente apprezzato per la sua facilità d'uso, la scalabilità e la ricchezza di funzionalità, rendendolo una scelta popolare per la comunicazione online.
\subsection*{Docker} 
 \addcontentsline{toc}{subsection}{Docker}
Docker è una piattaforma di containerizzazione open-source che consente agli sviluppatori di creare, distribuire e gestire applicazioni in container leggeri e portabili. I container Docker sono un tipo di virtualizzazione a livello di sistema operativo che isolano le applicazioni e le relative dipendenze, consentendo loro di essere eseguite in ambienti consistenti e riproducibili su qualsiasi piattaforma che supporti Docker.
\subsection*{Docker Compose} 
 \addcontentsline{toc}{subsection}{Docker Compose}
Uno strumento che permette di definire e gestire applicazioni Docker multi-contenitore. Utilizzando un file .yaml, Docker Compose consente di descrivere i servizi, le reti e i volumi necessari per un'applicazione. Con un singolo comando, è possibile creare, avviare e gestire tutti i contenitori specificati nel file, facilitando lo sviluppo, il test e il deployment di applicazioni complesse che coinvolgono più componenti interdipendenti.
\subsection*{Documento Esterno} 
 \addcontentsline{toc}{subsection}{Documento Esterno}
Un documento esterno è un'opera o un insieme di documenti che sono destinati a essere condivisi con terzi al di fuori del gruppo di sviluppo o dell'organizzazione. Questi documenti possono includere proposte di progetto, rapporti di avanzamento, documenti di requisiti, manuali utente, documenti di consegna o qualsiasi altra documentazione destinata ad aziende proponenti, clienti, i professori o altri stakeholder esterni al processo di sviluppo. La creazione e la gestione accurata dei documenti esterni sono cruciali per garantire una comunicazione chiara, trasparente ed efficace con gli interessati esterni al progetto.
\subsection*{Documento Interno} 
 \addcontentsline{toc}{subsection}{Documento Interno}
Un documento interno è un file o un insieme di documenti utilizzati all'interno del gruppo di sviluppo o dell'organizzazione per scopi di comunicazione, pianificazione, gestione o documentazione. Questi documenti possono includere specifiche tecniche, documenti di progettazione, diagrammi, report di progresso, verbali delle riunioni, linee guida interne o qualsiasi altra documentazione destinata esclusivamente all'uso all'interno del team di lavoro. La creazione e la gestione efficace dei documenti interni sono fondamentali per garantire una collaborazione efficiente, un flusso di lavoro organizzato e una condivisione accurata delle informazioni all'interno del gruppo.
