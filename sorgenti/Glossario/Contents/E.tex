\section*{E} 
\addcontentsline{toc}{section}{E}
\subsection*{Easy Meal} 
 \addcontentsline{toc}{subsection}{Easy Meal}
Easy Meal è un progetto proposto da Imola Informatica e sviluppato con l'obiettivo di semplificare il processo di prenotazione dei tavoli nei ristoranti, offrendo agli utenti la possibilità di pianificare in anticipo il proprio pasto e ordinarlo in modo collaborativo con altri utenti. Con Easy Meal, gli utenti possono prenotare facilmente un tavolo, selezionare le loro preferenze alimentari e ordinare il cibo desiderato prima del giorno desiderato, consentendo loro di godersi un'esperienza gastronomica senza problemi e senza attese. Grazie a questa piattaforma intuitiva e user-friendly, prenotare un tavolo e decidere cosa mangiare diventa un'esperienza semplice e piacevole per tutti i clienti dei ristoranti.
\subsection*{Eccezione} 
 \addcontentsline{toc}{subsection}{Eccezione}
Un'eccezione è un'anomalia o evento imprevisto durante l'esecuzione di un programma che interrompe il flusso normale di esecuzione. Può essere causata da errori di programmazione, input non validi o condizioni impreviste. Richiede una gestione appropriata per mantenere la stabilità del software.
\subsection*{Editor} 
 \addcontentsline{toc}{subsection}{Editor}
Un editor è un'applicazione software utilizzata per creare, modificare e gestire file di testo o codice sorgente. Fornisce funzionalità come formattazione del testo, ricerca e sostituzione, e strumenti per la gestione dei progetti.
\subsection*{Endpoint} 
 \addcontentsline{toc}{subsection}{Endpoint}
L'endpoint è un URL specifico esposto da un'API (Application Programming Interface) che rappresenta un punto di ingresso per la comunicazione tra un client e un server. Gli endpoint sono utilizzati per eseguire operazioni come il recupero, l'invio, l'aggiornamento o la cancellazione di dati. Ogni endpoint è associato a un'operazione HTTP e consente ai client di interagire con le risorse del server in modo strutturato e sicuro.
