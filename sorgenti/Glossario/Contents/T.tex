\section{T} 
\subsection{Tecnologia} 
La "tecnologia" nel contesto dello sviluppo del software si riferisce agli strumenti, ai linguaggi di programmazione, ai framework e ad altre risorse utilizzate per progettare, sviluppare e implementare soluzioni software. Questo termine può anche comprendere hardware, piattaforme di hosting, servizi cloud e altri componenti tecnologici utilizzati nel processo di sviluppo e distribuzione del software. La scelta delle tecnologie giuste può influenzare significativamente le prestazioni, la scalabilità, la manutenibilità e altre caratteristiche del prodotto software. 
\subsection{Telegram} 
Telegram è un'applicazione di messaggistica istantanea e di comunicazione. Consente agli utenti di scambiare messaggi di testo, foto, video, file e altro ancora tramite Internet. Telegram offre anche funzionalità come chat di gruppo, chiamate vocali e video, canali pubblici, bot automatizzati e crittografia end-to-end per garantire la privacy e la sicurezza delle comunicazioni. È stato utilizzato dal gruppo per la messaggistica istantanea e per fissare meeting di progetto anche con l’uso di sondaggi a scelta multipla e variabile.
\subsection{Template} 
Un template, in ambito documentale, è un modello predefinito che fornisce una struttura organizzativa e visiva per la creazione di documenti. Questi modelli includono solitamente sezioni già definite, formattazione standard, stili di testo preimpostati e elementi grafici coerenti (usati ad esempio nei powerpoint dei diari di bordo e per le prime pagine di ogni documento, ovvero copertina e indice). L'uso di un template consente di uniformare l'aspetto e la struttura dei documenti prodotti da diversi autori, facilitando la creazione e la lettura dei contenuti. 
\subsection{Test} 
Il termine "test" si riferisce a un'attività mirata a valutare o verificare le funzionalità o le prestazioni di un sistema, un'applicazione o un componente software. I test possono essere condotti per individuare difetti, verificare il corretto funzionamento di una funzionalità o garantire il rispetto dei requisiti specificati. 
\subsection{Ticket} 
Un "ticket" in Jira è un elemento di tracciamento utilizzato per registrare e monitorare il lavoro in un progetto software. Ogni ticket rappresenta un'attività specifica da completare, come risolvere un bug, implementare una nuova funzionalità o completare un compito di manutenzione. I ticket contengono informazioni dettagliate sul lavoro da svolgere, inclusi i requisiti, le assegnazioni, le scadenze e lo stato di avanzamento. Possono essere assegnati a membri del team e seguiti nel tempo per garantire che vengano completati in modo efficace e tempestivo. 
\subsection{Trigger} 
Il "trigger" nell'analisi dei requisiti si riferisce a un evento o una condizione che innesca l'avvio di un determinato caso d'uso o processo all'interno del sistema software. È un elemento chiave per comprendere quando e come un particolare comportamento o funzionalità deve essere attivato nell'applicazione. I trigger possono essere attivati da azioni degli utenti, come clic su pulsanti o input di dati, o da eventi interni al sistema, come arrivo di notifiche o cambiamenti di stato. Identificare correttamente i trigger è fondamentale per definire in modo completo e accurato il comportamento del sistema. 
