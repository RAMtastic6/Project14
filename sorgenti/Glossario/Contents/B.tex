\section*{B} 
\addcontentsline{toc}{section}{B}
\subsection*{Backend} 
 \addcontentsline{toc}{subsection}{Backend}
Il backend è la parte di un'applicazione software che gestisce la logica di business, l'accesso ai dati e le operazioni del server. Il backend include server, database, e API che permettono la comunicazione tra il frontend e il database. È responsabile dell'elaborazione delle richieste degli utenti, della gestione della sicurezza, dell'autenticazione, e dell'integrazione con altri servizi esterni.
\subsection*{Best practices} 
 \addcontentsline{toc}{subsection}{Best practices}
Le best practices nel contesto della produzione software sono l'applicazione di metodi e procedure che nel corso del tempo e attraverso l'esperienza pratica hanno dimostrato di essere le migliori in termini di efficienza ed efficacia nel raggiungimento degli obiettivi prefissati.
\subsection*{Branch} 
 \addcontentsline{toc}{subsection}{Branch}
Un ramo (branch) in GitHub è una versione separata del codice sorgente di un progetto software. Consente agli sviluppatori di lavorare su nuove funzionalità o correzioni di bug senza influenzare direttamente il codice principale (ramo principale o 'master'). I branch sono utilizzati per sviluppare in modo isolato, consentendo agli sviluppatori di sperimentare liberamente senza compromettere la stabilità del codice principale. Una volta completate le modifiche su un ramo, è possibile integrare (effettuare il 'merge') le modifiche nel ramo principale preferibilmente tramite una richiesta di pull (pull request) per incorporare le modifiche nel codice principale.
