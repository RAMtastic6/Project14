\section{B} 
\subsection{Branch} 
Un ramo (branch) in GitHub è una versione separata del codice sorgente di un progetto software. Consente agli sviluppatori di lavorare su nuove funzionalità o correzioni di bug senza influenzare direttamente il codice principale (ramo principale o 'master'). I branch sono utilizzati per sviluppare in modo isolato, consentendo agli sviluppatori di sperimentare liberamente senza compromettere la stabilità del codice principale. Una volta completate le modifiche su un ramo, è possibile integrare (effettuare il 'merge') le modifiche nel ramo principale preferibilmente tramite una richiesta di pull (pull request) per incorporare le modifiche nel codice principale.
