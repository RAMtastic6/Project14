\section{A} 
\subsection{Analisi Dei Requisiti} 
L'analisi dei requisiti è il processo di raccolta, documentazione e analisi delle esigenze e delle funzionalità di un sistema software. Questo processo coinvolge la comprensione approfondita dei requisiti dell'utente, delle necessità del business e delle specifiche tecniche per sviluppare un sistema che soddisfi le aspettative degli stakeholder. Gli obiettivi principali dell'analisi dei requisiti includono la definizione chiara e completa delle funzionalità del sistema, la specifica dei vincoli e delle restrizioni, e l'identificazione dei requisiti non funzionali come quelli relativi alle prestazioni, sicurezza e usabilità.
\subsection{Analisi Dei Rischi} 
L'analisi dei rischi è il processo di identificazione, valutazione e gestione dei potenziali problemi e delle minacce che potrebbero influenzare il successo di un progetto o di un'attività. Questo processo coinvolge la valutazione dei rischi potenziali, la determinazione della loro probabilità di accadere e dell'impatto che potrebbero avere sul progetto, nonché lo sviluppo di strategie per mitigare o gestire tali rischi. L'obiettivo dell'analisi dei rischi è identificare precocemente le potenziali problematiche e adottare misure preventive o correttive per ridurre al minimo gli impatti negativi sul progetto.
\subsection{API} 
Le API REST (Representational State Transfer) nell'ambito delle applicazioni web sono un insieme di principi architetturali che definiscono come le risorse Web devono essere definite e accessibili. Le API REST consentono alle applicazioni client di comunicare con un server Web in modo standardizzato utilizzando richieste HTTP per accedere e manipolare le risorse, come ad esempio dati o servizi.
\subsection{Applicazione Web Responsive} 
Un'applicazione web progettata per adattarsi automaticamente e fornire un'esperienza utente ottimale su una varietà di dispositivi e dimensioni dello schermo, inclusi computer desktop, laptop, tablet e smartphone, utilizzando tecniche di progettazione e sviluppo che consentono all'interfaccia utente di adattarsi dinamicamente alle dimensioni del dispositivo dell'utente.
\subsection{Architettura} 
Nell'ambito del design di un'applicazione, l'architettura si riferisce alla struttura complessiva dell'applicazione stessa, inclusi i suoi componenti principali, le relazioni tra di essi e il modo in cui sono organizzati per soddisfare gli obiettivi dell'applicazione. Questa struttura definisce come i diversi moduli dell'applicazione interagiscono tra loro per gestire dati, input utente, elaborazione e output. Un'architettura ben progettata mira a garantire la scalabilità, la manutenibilità e l'efficienza dell'applicazione nel tempo.
\subsection{Attore} 
Un attore è entità che interagisce con il sistema svolgendo delle attività, intesa sia come persona che come sistema terzo/esterno. Ciascuna entità è caratterizzata dall’insieme delle azioni che può compiere.
