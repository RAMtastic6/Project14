\section{A} 
\subsection{Analisi Dei Requisiti} 
L'analisi dei requisiti è il processo di raccolta, documentazione e analisi delle esigenze e delle funzionalità di un sistema software. Questo processo coinvolge la comprensione approfondita dei requisiti dell'utente, delle necessità del business e delle specifiche tecniche per sviluppare un sistema che soddisfi le aspettative degli stakeholder. Gli obiettivi principali dell'analisi dei requisiti includono la definizione chiara e completa delle funzionalità del sistema, la specifica dei vincoli e delle restrizioni, e l'identificazione dei requisiti non funzionali come prestazioni, sicurezza e usabilità.
\subsection{Analisi Dei Rischi} 

\subsection{Applicazione Web Responsive} 

\subsection{Acquisizione} 

\subsection{Attore} 

