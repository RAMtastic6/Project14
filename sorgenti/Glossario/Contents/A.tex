\section{A} 
\subsection{Analisi Dei Requisiti} 
L'analisi dei requisiti � il processo di raccolta, documentazione e analisi delle esigenze e delle funzionalit� di un sistema software. Questo processo coinvolge la comprensione approfondita dei requisiti dell'utente, delle necessit� del business e delle specifiche tecniche per sviluppare un sistema che soddisfi le aspettative degli stakeholder. Gli obiettivi principali dell'analisi dei requisiti includono la definizione chiara e completa delle funzionalit� del sistema, la specifica dei vincoli e delle restrizioni, e l'identificazione dei requisiti non funzionali come prestazioni, sicurezza e usabilit�.
\subsection{Analisi Dei Rischi} 
L'analisi dei rischi � il processo di identificazione, valutazione e gestione dei potenziali problemi e delle minacce che potrebbero influenzare il successo di un progetto o di un'attivit�. Questo processo coinvolge la valutazione dei rischi potenziali, la determinazione della loro probabilit� di accadere e dell'impatto che potrebbero avere sul progetto, nonch� lo sviluppo di strategie per mitigare o gestire tali rischi. L'obiettivo dell'analisi dei rischi � identificare precocemente le potenziali problematiche e adottare misure preventive o correttive per ridurre al minimo gli impatti negativi sul progetto.
\subsection{Applicazione Web Responsive} 
Un'applicazione web progettata per adattarsi automaticamente e fornire un'esperienza utente ottimale su una variet� di dispositivi e dimensioni dello schermo, inclusi computer desktop, laptop, tablet e smartphone, utilizzando tecniche di progettazione e sviluppo che consentono all'interfaccia utente di adattarsi dinamicamente alle dimensioni del dispositivo dell'utente.
\subsection{Attore} 
Entit� che interagisce con il sistema svolgendo delle attivit�, intesa sia come persona che come sistema terzo/esterno. Ciascuna entit� � caratterizzata dall�insieme delle azioni che pu� compiere.
