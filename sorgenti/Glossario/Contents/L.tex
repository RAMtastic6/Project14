\section*{L} 
\addcontentsline{toc}{section}{L}
\subsection*{LaTeX} 
 \addcontentsline{toc}{subsection}{LaTeX}
LaTeX viene spesso utilizzato per redigere documenti tecnici, come specifiche dei requisiti, documenti di progetto, manuali utente e rapporti di testing. La sua struttura modulare, la gestione avanzata della formattazione e la facilità nel gestire equazioni matematiche e tabelle lo rendono una scelta popolare tra gli sviluppatori e i professionisti del settore. LaTeX consente inoltre di mantenere una formattazione uniforme e di automatizzare alcune operazioni, rendendo più efficiente il processo di documentazione del progetto.
\subsection*{Layer di Persistenza} 
 \addcontentsline{toc}{subsection}{Layer di Persistenza}
Il layer di persistenza, è una parte dell'architettura software responsabile della gestione della persistenza dei dati. Questo strato si occupa di interagire con il sistema di memorizzazione permanente dei dati, come ad esempio un database relazionale, per memorizzare, recuperare, aggiornare ed eliminare i dati.
\subsection*{Logica di Business} 
 \addcontentsline{toc}{subsection}{Logica di Business}
La logica di business si riferisce al componente dell'applicazione che si occupa dell'implementazione delle regole di business specifiche del dominio applicativo. È spesso isolata in uno strato dedicato, noto come Service layer, che si trova tra il livello di presentazione (come l'interfaccia utente) e il livello di persistenza dei dati (come il database).
