\section*{F} 
\addcontentsline{toc}{section}{F}
\subsection*{Facade} 
 \addcontentsline{toc}{subsection}{Facade}
Il facade è un design pattern strutturale che fornisce un'interfaccia unificata per un insieme di interfacce in un sottosistema più grande. Essenzialmente, la Facade nasconde la complessità del sottosistema e semplifica l'accesso ad esso fornendo un'interfaccia semplice e unificata.
\subsection*{Feature} 
 \addcontentsline{toc}{subsection}{Feature}
Una feature è una caratteristica o funzionalità specifica di un prodotto software. Le feature definiscono il comportamento o le capacità del software e possono includere qualsiasi cosa, dalle semplici azioni dell'utente alle complesse operazioni di sistema. Le feature sono progettate per soddisfare determinati requisiti o fornire determinati vantaggi agli utenti finali. Possono essere descritte attraverso specifiche funzionali, requisiti o casi d'uso e sono spesso evidenziate durante lo sviluppo del software per garantire che il prodotto finale soddisfi le aspettative degli utenti.
\subsection*{Feedback} 
 \addcontentsline{toc}{subsection}{Feedback}
I feedback sono opinioni, valutazioni o commenti forniti dagli utenti o dagli stakeholder su un prodotto, un servizio o un'esperienza. Il feedback può riguardare diversi aspetti, come l'usabilità, le prestazioni, le funzionalità o il design di un'applicazione software. È importante raccogliere, analizzare e utilizzare il feedback per migliorare continuamente il prodotto o il servizio, soddisfare le esigenze degli utenti e garantire una migliore esperienza complessiva.
\subsection*{Fornitura} 
 \addcontentsline{toc}{subsection}{Fornitura}
La fornitura indica il processo primario che riguarda l'approvvigionamento, l'acquisizione e la consegna di prodotti o servizi esterni necessari per il completamento del progetto, nonché la gestione delle relazioni con i fornitori esterni.
\subsection*{Framework} 
 \addcontentsline{toc}{subsection}{Framework}
Un framework è un'infrastruttura software che fornisce un'organizzazione predefinita per lo sviluppo di applicazioni. Essenzialmente, è una struttura di supporto su cui è possibile costruire e organizzare il codice. I framework forniscono una serie di librerie, modelli di progettazione, componenti e strumenti che semplificano lo sviluppo di software, consentendo agli sviluppatori di concentrarsi sulla logica dell'applicazione anziché dover reinventare ogni singolo aspetto. Spesso sono progettati per affrontare problemi comuni nello sviluppo di software, come la gestione delle richieste web, la manipolazione dei dati, la sicurezza e altro ancora. I framework possono essere generici o specifici per determinati tipi di applicazioni o linguaggi di programmazione.
\subsection*{Frontend} 
 \addcontentsline{toc}{subsection}{Frontend}
Il frontend è la parte di un'applicazione software che interagisce direttamente con l'utente finale. Il frontend include l'interfaccia utente (UI) e tutte le componenti che rendono l'esperienza utente interattiva e visivamente piacevole. È sviluppato utilizzando tecnologie come HTML, CSS e JavaScript, e comunica con il backend per recuperare e visualizzare i dati.
\subsection*{Funzionalità} 
 \addcontentsline{toc}{subsection}{Funzionalità}
Le funzionalità si riferiscono alle capacità o agli attributi specifici di un software, di un'applicazione o di un sistema che consentono agli utenti di eseguire determinate azioni o di ottenere determinati risultati. Le funzionalità sono le varie operazioni, servizi o caratteristiche che un prodotto software offre agli utenti per soddisfare determinati scopi o necessità. Possono includere azioni interattive, processi automatizzati, visualizzazioni dei dati, strumenti di gestione e altro ancora. Le funzionalità sono progettate per rispondere ai requisiti e alle esigenze degli utenti, migliorare l'usabilità del software e offrire un'esperienza soddisfacente agli utenti finali.
