\section*{G} 
\addcontentsline{toc}{section}{G}
\subsection*{Git} 
 \addcontentsline{toc}{subsection}{Git}
Git è un sistema di controllo di versione distribuito utilizzato principalmente nello sviluppo software. Consente agli sviluppatori di tenere traccia delle modifiche apportate al codice sorgente nel corso del tempo, coordinare il lavoro con altri membri del team e gestire le diverse versioni di un progetto in modo efficiente. Git consente di creare repository (archivi di file sorgente e cronologia delle modifiche) sia localmente che su server remoti, facilitando la collaborazione tra sviluppatori distribuiti geograficamente. È ampiamente utilizzato nell'industria del software per gestire progetti di qualsiasi dimensione e complessità.
\subsection*{Gitflow} 
 \addcontentsline{toc}{subsection}{Gitflow}
GitFlow è un modello di branching e di workflow basato su Git, progettato per gestire progetti software con un ciclo di sviluppo complesso e ramificato. Questo approccio fornisce una struttura chiara per la gestione delle diverse fasi dello sviluppo, inclusi i rilasci, le correzioni di bug e le nuove funzionalità. GitFlow prevede l'utilizzo di due branch principali, 'master' e 'develop', oltre a una serie di branch temporanei per le funzionalità in sviluppo denominati 'feature branch', i bugfix e i rilasci. Questo modello facilita il lavoro in team e la collaborazione su progetti complessi, consentendo di mantenere una storia di versionamento pulita e organizzata.
\subsection*{Github} 
 \addcontentsline{toc}{subsection}{Github}
GitHub è una piattaforma di hosting di codice sorgente basata su Git, che fornisce strumenti per la gestione dei repository, la collaborazione tra sviluppatori e il controllo delle versioni dei progetti software. Su GitHub, gli sviluppatori possono caricare i propri progetti, tenere traccia delle modifiche tramite commit, creare e gestire branch, collaborare con altri utenti tramite problemi e richieste di pull, e distribuire le proprie applicazioni. È ampiamente utilizzato sia da sviluppatori individuali che da team di sviluppo per lo sviluppo di software open source e progetti privati.
