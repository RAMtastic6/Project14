\section{C} 
\subsection{Candidatura} 
La candidatura in un contesto di progetto didattico indica l'atto attraverso il quale uno studente o un gruppo di studenti manifesta il proprio interesse a partecipare allo sviluppo di un progetto proposto dall'azienda o dall'istituzione didattica. Essa è un'azione volontaria che implica la presentazione di una proposta o di un interesse formale per essere considerati come partecipanti al progetto.
\subsection{Capitolato} 
Un capitolato è un documento tecnico allegato a un contratto di appalto, che definisce le specifiche tecniche delle opere da eseguire nel contesto del contratto stesso. Solitamente parte integrante del contratto, il capitolato stabilisce i diritti, i doveri e le particolarità relative all'esecuzione dei lavori tra le parti coinvolte. Serve da riferimento per delineare gli accordi tra i soggetti privati coinvolti nel progetto.
\subsection{Caption} 
La caption di una tabella o di un'immagine in una documentazione è una breve descrizione o etichetta che fornisce informazioni sul contenuto della tabella o dell'immagine stessa. Ha lo scopo di identificare e spiegare il contenuto visivo o i dati presentati nella tabella, mentre nel caso delle immagini può includere una breve descrizione o un titolo per aiutare il lettore a comprendere il contesto della stessa.
\subsection{Caso d'uso} 
Un caso d'uso definisce le interazioni tra gli attori (utenti) e il sistema software, rappresentando sequenze di azioni che guidano l'utilizzo del sistema per raggiungere determinati obiettivi. Queste interazioni descrivono come il sistema deve essere utilizzato e quali funzionalità devono essere esposte agli utenti.
\subsection{Changelog} 
Un changelog è una sezione di documentazione che tiene traccia delle modifiche apportate a un software o a un progetto nel tempo. Esso elenca le variazioni, le correzioni di bug, le nuove funzionalità e altre modifiche significative, fornendo un registro storico delle revisioni effettuate. Il changelog è utile per gli sviluppatori e gli utenti finali per comprendere le modifiche apportate a una determinata versione del software oppure di un documento e per monitorare l'evoluzione del prodotto nel tempo.
\subsection{Codifica} 
La codifica è il processo attraverso il quale si traducono le specifiche di progettazione in istruzioni eseguibili da un computer. Questo coinvolge la scrittura e la programmazione del codice sorgente utilizzando un linguaggio di programmazione specifico. Durante la codifica, gli sviluppatori scrivono algoritmi e istruzioni che definiscono il comportamento del software, implementando le funzionalità richieste e risolvendo i problemi identificati durante la fase di progettazione. La codifica è un passaggio cruciale nello sviluppo del software e richiede attenzione ai dettagli e competenze tecniche specifiche.
\subsection{Complessità ciclomatica} 
La complessità ciclomatica è una metrica software utilizzata per misurare la complessità di un programma. Essa viene calcolata utilizzando il grafo di controllo di flusso del programma, dove i nodi del grafo rappresentano gruppi indivisibili di istruzioni e gli archi orientati connettono due nodi se il secondo può essere eseguito immediatamente dopo il primo. La complessità ciclomatica di una sezione di codice esprime il numero di cammini linearmente indipendenti attraverso il grafo di controllo di flusso del programma.
\subsection{Consuntivo} 
Il consuntivo è un rapporto che confronta le previsioni o le stime con i dati effettivi o reali, al fine di valutare le variazioni e le differenze tra ciò che era previsto e ciò che è stato effettivamente raggiunto. Nell'ambito dello sviluppo del software o di altri progetti, il consuntivo può essere utilizzato per analizzare il bilancio finanziario, il tempo impiegato, le risorse utilizzate e altre metriche di progetto. Fornisce una valutazione critica delle prestazioni e dell'efficacia del progetto, consentendo di prendere decisioni informate e di apportare eventuali correzioni di rotta necessarie per mantenere il progetto sul giusto percorso.
\subsection{Controllo di Configurazione} 
Il controllo di configurazione è un processo utilizzato nel campo dello sviluppo del software per gestire le modifiche alla configurazione e alla struttura del software stesso. Coinvolge l'identificazione, la registrazione, la revisione e l'approvazione delle modifiche apportate al codice sorgente, alla documentazione, ai file di configurazione e ad altri elementi del progetto. Il controllo di configurazione aiuta a mantenere la coerenza e l'integrità del sistema software durante l'evoluzione del progetto, garantendo che le modifiche vengano gestite in modo ordinato e tracciabile. Questo processo è essenziale per il mantenimento della qualità del software e per garantire che tutte le versioni e le varianti del sistema siano correttamente documentate e controllate.
\subsection{CSS} 
CSS, (Cascading Style Sheets - fogli di stile a cascata), è un linguaggio di stile utilizzato per definire l'aspetto e la formattazione di documenti HTML (HyperText Markup Language) e XML (eXtensible Markup Language); consente di controllare il layout, i colori, i tipi di carattere, le dimensioni e altri aspetti visivi di un documento web. Con CSS è possibile separare la struttura e il contenuto di una pagina web (definiti tramite HTML) dalla loro presentazione visiva, garantendo una maggiore flessibilità e facilità di manutenzione.
