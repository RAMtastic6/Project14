\section{P} 
\subsection{PB} 
La Product Baseline è un insieme di documenti e materiali che rappresentano lo stato approvato di un prodotto in un determinato momento del ciclo di vita del progetto. Include specifiche di progettazione, documenti di requisiti, codice sorgente e altri artefatti pertinenti. La Product Baseline serve come riferimento per valutare il progresso del progetto e assicurare che il prodotto soddisfi i requisiti stabiliti durante la fase di pianificazione. È il risultato di un'analisi e di una revisione approfondita dei requisiti e delle specifiche del progetto, ed è approvata dal cliente o dagli stakeholder.
\subsection{PDF} 
Il Portable Document Format (PDF) è un formato di file sviluppato da Adobe che consente di rappresentare documenti in modo indipendente dall'hardware e dal software utilizzati per crearli o visualizzarli. I file PDF mantengono la formattazione originale del documento, inclusi testo, immagini e layout, consentendo la condivisione e la visualizzazione coerenti su diverse piattaforme e dispositivi. 
\subsection{Piano di Progetto} 
Il Piano di Progetto è un documento che definisce gli obiettivi, le risorse, le attività e le scadenze necessarie per completare con successo un progetto. Include la pianificazione delle attività, la suddivisione dei compiti, la stima dei tempi e dei costi, nonché le strategie per gestire i rischi e rispettare le scadenze. Il Piano di Progetto fornisce una guida dettagliata per il team di progetto e per gli stakeholder, consentendo una gestione efficace e un monitoraggio delle attività lungo tutto il ciclo di vita del progetto. 
\subsection{Piano di Qualifica} 
Il Piano di Qualifica è un documento che definisce le strategie, le attività e le risorse necessarie per garantire che il prodotto soddisfi i requisiti di qualità stabiliti. Esso descrive le procedure di verifica e validazione che verranno utilizzate per valutare e confermare che il prodotto soddisfi gli standard di qualità definiti. Include anche gli obiettivi di qualità, le metriche di valutazione, i criteri di accettazione e le responsabilità del team per garantire che il prodotto finale sia conforme alle aspettative del cliente. Il Piano di Qualifica è essenziale per garantire che il prodotto sia testato in modo rigoroso e che soddisfi gli standard di qualità richiesti. 
\subsection{PoC} 
Il  Proof of Concept (PoC) è una realizzazione preliminare o parziale di un progetto, finalizzata a dimostrare la fattibilità delle idee e delle tecnologie coinvolte nel progetto stesso, basandosi su principi o concetti fondamentali del prodotto finale. Essa costituisce un prototipo, utilizzato per valutare la fattibilità del progetto. 
\subsection{Posta elettronica} 
La posta elettronica è un sistema di comunicazione che consente agli utenti di inviare e ricevere messaggi tramite internet. I messaggi sono inviati e ricevuti elettronicamente attraverso server di posta elettronica, consentendo agli utenti di comunicare in modo rapido e efficiente a distanza. 
\subsection{Postcondizione} 
La postcondizione è lo stato del sistema o le condizioni che devono essere soddisfatte dopo l'esecuzione di un caso d'uso di una funzione specifica. Indica lo stato o le modifiche che il sistema dovrebbe avere raggiunto al termine del caso d'uso o della funzione presa in esame 
\subsection{Precondizione} 
La precondizione è uno stato o una condizione del sistema che deve essere soddisfatta affinché un caso d'uso o una funzione possa essere eseguita con successo. Indica lo stato iniziale o le circostanze necessarie perché il caso d'uso o la funzione presa in esame possa essere avviata e completata correttamente. 
\subsection{Prenotazione} 
La prenotazione è la fase del processo in cui l'utente desidera prenotare un tavolo presso un ristorante utilizzando l'applicazione. Durante questa fase, l'utente inserisce i dettagli della prenotazione, come la data, l'orario, il numero di persone e eventuali altre richieste. Una volta completata la prenotazione, il sistema invia una conferma all'utente e al ristorante, cercando di fare in modo che il tavolo sia riservato per il momento richiesto. 
\subsection{Preventivo} 
Il preventivo è un'analisi stimata dei costi e delle risorse necessarie per lo sviluppo e l'implementazione di un progetto software. Include valutazioni delle ore di lavoro, delle risorse di sviluppo e delle eventuali spese. Il preventivo serve a pianificare e gestire le risorse disponibili per il progetto, nonché a garantire che il lavoro venga completato entro i limiti temporali e finanziari stabiliti. 
\subsection{Principio di miglioramento continuo} 
Il principio di miglioramento continuo si riferisce alla pratica di costante riflessione e aggiornamento delle metodologie, dei processi e degli strumenti impiegati nel ciclo di vita del software. Si adotta questo principio per valutare costantemente il proprio lavoro, identificare punti di forza e di debolezza e apportare miglioramenti incrementali. Questo approccio mira a ottimizzare l'efficienza, la qualità e l'efficacia del lavoro svolto, consentendo al team di adattarsi ai cambiamenti, risolvere eventuali problemi e perseguire un costante progresso nel raggiungimento degli obiettivi del progetto. 
\subsection{Processo} 
Il processo è una successione di eventi/attività che producono un risultato visibile e concreto. Deve essere condotto in modo sistematico, disciplinato e misurabile.
\subsection{Processo primario} 
Il processo primario si riferisce alle attività direttamente coinvolte nella creazione del prodotto software o nel fornire valore diretto al cliente. Questi processi sono centrali per il raggiungimento degli obiettivi del progetto e includono attività come analisi dei requisiti, progettazione, implementazione, test e consegna del software.
\subsection{Processo di supporto} 
I processi di supporto forniscono assistenza e risorse per i processi primari e organizzativi. Questi processi includono attività come la gestione della configurazione, la gestione della qualità, la formazione, la gestione delle risorse e la gestione dei rischi. L'obiettivo dei processi di supporto è quello di garantire che i processi primari e organizzativi siano eseguiti in modo efficiente ed efficace.
\subsection{Processo organizzativo} 
I processi organizzativi sono responsabili della gestione e del coordinamento delle attività all'interno dell'organizzazione. Questi processi includono la definizione delle politiche e delle procedure aziendali, la pianificazione strategica, la gestione dei progetti e la comunicazione interna ed esterna. Gli obiettivi dei processi organizzativi sono quello di garantire che l'organizzazione abbia una struttura e un'infrastruttura adeguate per supportare l'esecuzione dei processi primari e di supporto.
\subsection{Progettazione architetturale} 
La progettazione architetturale è il processo di definizione della struttura di base di un sistema software, che include l'identificazione dei componenti, delle loro interazioni e delle decisioni di progettazione chiave. L'obiettivo è garantire che il software soddisfi i requisiti funzionali e non funzionali. 
\subsection{Progettazione dettagliata} 
La progettazione dettagliata è la fase del processo di sviluppo del software in cui i dettagli tecnici dell'architettura definita nella progettazione architetturale vengono elaborati in modo più specifico. Questo include la definizione delle singole componenti, dei loro comportamenti, delle interfacce e delle relazioni tra di esse. La progettazione dettagliata fornisce una guida chiara per l'implementazione del sistema software. 
\subsection{Push} 
Il push è un'operazione utilizzata in Git per inviare le modifiche locali apportate al repository locale verso un repository remoto. Questo consente di condividere le modifiche con altri membri del team o di archiviarle su un server remoto come GitHub. 
