\section{O} 
\subsection{Open-source} 
Un modello di sviluppo e distribuzione del software in cui il codice sorgente del programma è reso liberamente accessibile, modificabile e redistribuibile da chiunque. Questo implica che il codice sorgente di un software open-source è disponibile per essere esaminato, modificato e migliorato da sviluppatori di tutto il mondo, senza restrizioni di utilizzo.
\subsection{Overleaf} 
Overleaf è una piattaforma online per la scrittura collaborativa di documenti LaTeX. Consente agli utenti di creare e modificare documenti LaTeX direttamente nel browser, senza la necessità di installare software aggiuntivo sul proprio computer. Overleaf offre strumenti per la gestione dei progetti, il controllo delle versioni, la compilazione automatica dei documenti e la condivisione dei lavori con altri collaboratori. È particolarmente popolare nell'ambito accademico e scientifico per la sua facilità d'uso e la possibilità di collaborare in tempo reale. 
\subsection{Ordinazione} 
Nella fase di ordinazione dell'applicazione Easy Meal, l'utente seleziona e richiede un piatto specifico dal menu virtuale disponibile sull'applicazione o sul sito web. Durante questa fase, l'utente esplora le opzioni del menu, visualizza le descrizioni dei piatti, seleziona gli elementi desiderati e invia la propria richiesta di ordine al ristorante. 
