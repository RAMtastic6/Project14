\section{R} 
\subsection{Redattore} 
Un redattore è un individuo responsabile della stesura di documenti tecnici o di altro genere all'interno di un progetto. Il redattore assicura che i documenti siano chiari, accurati e coerenti. 
\subsection{Repository} 
Un repository è uno spazio di archiviazione digitale utilizzato per conservare, gestire e condividere file e dati relativi a un progetto. Può essere utilizzato per memorizzare codice sorgente, documentazione, risorse grafiche e altri file pertinenti al progetto. Un repository facilita la collaborazione tra membri del team consentendo loro di accedere, modificare e aggiornare i file in modo coordinato. 
\subsection{Requisito} 
Un requisito è una specifica o una condizione necessaria che un sistema, un prodotto o un servizio deve soddisfare per essere accettabile, utile o in grado di raggiungere determinati obiettivi. I requisiti vengono definiti durante il processo di analisi dei requisiti per guidare lo sviluppo e valutare il successo del progetto. 
\subsection{Requisiti funzionali desiderabili} 
I requisiti funzionali desiderabili sono quelli che, se soddisfatti, migliorerebbero l'esperienza dell'utente o aggiungerebbero valore al sistema, ma non sono considerati essenziali per il suo funzionamento di base. 
\subsection{Requisiti funzionali obbligatori} 
I requisiti funzionali obbligatori sono quei requisiti indispensabili per il corretto funzionamento del sistema o del prodotto, senza i quali il sistema non può essere considerato soddisfacente. 
\subsection{Riferimenti} 
Un riferimento in un documento si riferisce a un'informazione, un'idea o una fonte esterna che viene citata, menzionata o utilizzata come punto di riferimento per comprendere o supportare il contenuto del documento stesso. Può includere link ipertestuali, citazioni bibliografiche, numeri di pagina o altre forme di indicazioni che facilitano la consultazione o il richiamo ad altre fonti o parti del documento. 
\subsection{Riferimenti normativi} 
I riferimenti normativi sono quei riferimenti che indicano norme, standard o regolamenti ufficiali che devono essere rispettati durante lo svolgimento di un progetto o l'elaborazione di un documento. Questi possono includere leggi, direttive settoriali, standard di qualità o normative specifiche del settore.
\subsection{Riferimenti informativi} 
I riferimenti informativi si riferiscono a documenti, pubblicazioni o fonti informative che forniscono ulteriori dettagli, contestualizzazioni o approfondimenti relativi al contenuto del documento principale, ma non sono considerati vincolanti o obbligatori per il progetto o l'elaborazione del documento stesso.
\subsection{Rilascio} 
Il rilascio è un'istanza specifica di un prodotto software che è stato messo a disposizione degli utenti finali o dei clienti. Indica il momento in cui una determinata versione del software è pronta per essere distribuita e utilizzata dagli utenti. Un rilascio può includere nuove funzionalità, correzioni di bug, miglioramenti delle prestazioni o altre modifiche significative rispetto alle versioni precedenti del software. 
\subsection{Rischi} 
Un rischio è un evento o una condizione che, se si verificasse, potrebbe influenzare negativamente il raggiungimento degli obiettivi di un progetto. I rischi possono riguardare varie aree, come i tempi di consegna, i costi, la qualità del prodotto o la soddisfazione del cliente. Gli sforzi di gestione dei rischi sono finalizzati a identificare, valutare e mitigare i potenziali impatti negativi che i rischi potrebbero avere sul progetto. 
\subsection{Rischio organizzativo} 
Con il termine rischio organizzativo si riferisce alla possibilità che problemi interni all'organizzazione, come mancanza di risorse, conflitti tra membri del team o cambiamenti nell'organigramma aziendale, possano influenzare negativamente il successo del progetto.
\subsection{Rischio relativo al prodotto} 
Il rischio relativo al prodotto riguarda i potenziali difetti, malfunzionamenti o mancanze nel prodotto software stesso. Potrebbe includere problemi di qualità del software, non conformità ai requisiti utente o problemi di sicurezza.
\subsection{Rischio tecnologico} 
Si riferisce alla possibilità che le tecnologie utilizzate nel progetto, come linguaggi di programmazione, framework o strumenti, possano rivelarsi inadeguate, obsolete o incapaci di soddisfare i requisiti del progetto, causando ritardi o problemi durante lo sviluppo.
\subsection{Risorsa Umana} 
Risorsa Umana si riferisce al personale coinvolto nel progetto, composto da membri del team coinvolti nell'implementazione, nell'esecuzione e nel controllo delle attività previste. Questa risorsa è essenziale per il successo del progetto, in quanto contribuisce con le proprie competenze, esperienze e sforzi per raggiungere gli obiettivi stabiliti. La gestione efficace delle risorse umane include l'allocazione delle persone giuste ai compiti appropriati, la formazione e lo sviluppo del personale, la risoluzione dei conflitti e la creazione di un ambiente lavorativo collaborativo e motivante. 
\subsection{RTB} 
La Requirements Technology Baseline (RTB) è una linea di base tecnologica di un progetto software che rappresenta lo stato degli strumenti, delle tecniche e dei processi utilizzati per la gestione dei requisiti funzionali e non del sistema. Include gli strumenti software utilizzati per la raccolta, l'analisi, la tracciabilità e la gestione dei requisiti, nonché le procedure e le pratiche per l'identificazione, la documentazione e la validazione dei requisiti. La RTB fornisce una base stabile e affidabile per il processo di sviluppo del software, garantendo che gli strumenti e le tecnologie utilizzate siano adatti e siano stati compresi per lo scopo e soddisfino le esigenze del progetto.
