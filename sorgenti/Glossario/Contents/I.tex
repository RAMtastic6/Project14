\section{I} 
\subsection{Imola Informatica} 
Imola Informatica, proponente del progetto Easy Meal, � una societ� indipendente di consulenza IT. Tutto ci� che riguarda il mondo dell�information technology la riguarda, gli interessa e appassiona. Entrano in gioco ogni volta in cui una azienda pubblica o privata vuole migliorare i propri servizi. Innovare i propri processi di lavoro, gli approcci di management per cogliere le opportunit� business offerte dalla trasformazione digitale. Sono a servizio dei principali gruppi finanziari e assicurativi e ogni giorno sono a fianco di grandi aziende e piccole startup nel gestire il cambiamento tecnologico e culturale.
\subsection{Infrastruttura} 
L'infrastruttura rappresenta l'insieme degli elementi fisici e tecnologici necessari per il funzionamento di un sistema o di un'applicazione. Questi elementi includono hardware, software, reti, server, database e tutte le risorse e le tecnologie utilizzate per supportare le operazioni di un'organizzazione o di un progetto. Nell'ambito del progetto Easy Meal, l'infrastruttura potrebbe comprendere i server per l'hosting dell'applicazione, i database per la gestione dei dati dei ristoranti e degli utenti, nonch� la rete e gli strumenti di sicurezza necessari per garantire il corretto funzionamento e la protezione del sistema.
\subsection{Ingegneria del software} 
L'ingegneria del software � un insieme di principi e pratiche utilizzati per progettare, sviluppare, mantenere, testare e valutare il software per computer. Le relative tecniche vengono impiegate per guidare il processo di sviluppo del software, che comprende la definizione, l'implementazione, la valutazione, la misurazione, la gestione, il cambiamento e il miglioramento del ciclo di vita del software. Questa disciplina pone un forte accento sulla gestione della configurazione del software, che implica il controllo sistematico delle modifiche alla configurazione e il mantenimento dell'integrit� e della tracciabilit� della configurazione e del codice durante tutto il ciclo di vita del sistema, mediante l'uso del versioning del software.
