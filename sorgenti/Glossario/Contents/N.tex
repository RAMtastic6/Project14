\section*{N} 
\subsection*{Nest.js} 
Nest.js è un framework per lo sviluppo di applicazioni server-side basato su Node.js. Le caratteristiche offerte sono: dependency injection (controller), l'uso di decorators per definire i controller e come essi gestiscono le varie route e la modularità. Esso rende possibile anche connessioni basate su altri protocolli, come Websocket, gRPC e altri ancora.
\subsection*{Next.js} 
Next.js è un framework di sviluppo web per React, che consente di creare facilmente applicazioni web con funzionalità avanzate come il server-side rendering (SSR), il rendering statico e la generazione di pagine dinamiche.
\subsection*{Node.js} 
Node.js è un runtime JavaScript open-source basato sul motore JavaScript V8 di Google Chrome; è progettato per eseguire codice JavaScript lato server, consentendo agli sviluppatori di creare applicazioni web e servizi di rete scalabili e performanti.
\subsection*{Norme di Progetto} 
Nel documento Norme di Progetto sono riportate tutte le linee guida tecniche e procedurali stabilite per regolare l'esecuzione e il controllo di un progetto. Questo documento definisce standard e procedure relative all'organizzazione del lavoro, alla metodologia di sviluppo, alla gestione dei documenti, delle configurazioni e dei rischi, nonché alla pianificazione e al monitoraggio delle attività del progetto.
