\section*{J} 
\addcontentsline{toc}{section}{J}
\subsection*{Javascript} 
 \addcontentsline{toc}{subsection}{Javascript}
Un linguaggio di programmazione ad alto livello, interpretato e orientato agli oggetti, comunemente utilizzato per sviluppare applicazioni web interattive e dinamiche. Creato originariamente per essere eseguito nei browser web per migliorare l'esperienza utente sul web, JavaScript è diventato uno dei linguaggi di programmazione più diffusi e versatili al mondo.
\subsection*{Jest} 
 \addcontentsline{toc}{subsection}{Jest}
Jest è un framework di testing open-source per JavaScript utilizzato per testare applicazioni e componenti React, Node.js e JavaScript in generale. Esso è dotato di una serie di funzionalità avanzate per facilitare lo sviluppo dei test (come la reportistica sulla code-coverage di un progetto) e migliorare la qualità del software.
\subsection*{Jira} 
 \addcontentsline{toc}{subsection}{Jira}
Jira è una piattaforma di gestione del lavoro e dei progetti utilizzata per tracciare le attività, assegnare compiti, pianificare progetti e collaborare tra membri del team.
\subsection*{JSON} 
 \addcontentsline{toc}{subsection}{JSON}
JSON (JavaScript Object Notation) è un formato di dati leggero e basato su testo utilizzato per rappresentare dati strutturati. E' comunemente utilizzato per lo scambio di dati tra un server e un client web e come formato di memorizzazione dei dati (da servizi come MongoDB).
\subsection*{JSX} 
 \addcontentsline{toc}{subsection}{JSX}
JSX è un'estensione di sintassi per JavaScript che consente di scrivere markup HTML all'interno del codice JavaScript ed è comunemente utilizzato con React.
\subsection*{JWT} 
 \addcontentsline{toc}{subsection}{JWT}
JWT (JSON Web Token) è uno standard aperto utilizzato per creare token di accesso che permettono la trasmissione sicura di informazioni tra parti in formato JSON.
