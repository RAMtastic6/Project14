\section{M} 
\subsection{Manuale Utente} 
Il Manuale Utente è un documento che fornisce istruzioni dettagliate su come utilizzare un determinato prodotto software o sistema. È destinato agli utenti finali e fornisce indicazioni chiare e concise su come eseguire le diverse operazioni, navigare nell'interfaccia utente e sfruttare le funzionalità offerte dal software. Il manuale utente può includere tutorial, guide passo-passo, spiegazioni delle funzionalità e istruzioni per la risoluzione dei problemi comuni. L'obiettivo principale del manuale utente è quello di consentire agli utenti di utilizzare il software in modo efficace ed efficiente, migliorando così l'esperienza complessiva dell'utente.
\subsection{Milestone} 
Una milestone è un punto di riferimento significativo o un obiettivo importante all'interno di un progetto, utilizzato per misurare il progresso e il raggiungimento di determinati traguardi. Le milestone sono solitamente associate a date specifiche o a completamenti di specifiche attività o fasi di un progetto. Servono come punti di controllo per valutare se il progetto sta procedendo secondo i piani e per identificare eventuali ritardi o problemi. Le milestone possono essere utilizzate anche per comunicare i progressi del progetto alle parti interessate e per stabilire scadenze chiare e tangibili per il completamento delle attività.
