\section{S} 
\subsection{Scenario} 
Nel documento di Analisi dei Requisiti, uno scenario rappresenta una sequenza di azioni o eventi che descrivono un possibile flusso di interazione tra l'utente e il sistema. Gli scenari sono utilizzati per comprendere e illustrare come il sistema dovrebbe comportarsi in determinate situazioni o sotto determinate condizioni. Ogni scenario include una serie di passi che descrivono le azioni compiute dagli utenti e le risposte del sistema, consentendo di visualizzare in modo chiaro e dettagliato il comportamento previsto del software. Gli scenari possono essere utilizzati per identificare requisiti funzionali, individuare casi d'uso e validare il sistema durante lo sviluppo e il testing. 
\subsection{Scenario principale} 
Lo scenario principale rappresenta il flusso di interazione tipico tra l'utente e il sistema per raggiungere l'obiettivo principale del caso d'uso o della funzionalità in esame. Esso descrive la sequenza di passi che si verificano nella maggior parte dei casi.
\subsection{Scenario alternativo} 
Gli scenari alternativi, al contrario, rappresentano situazioni non standard o eccezionali che possono verificarsi durante l'esecuzione del caso d'uso. Questi scenari descrivono le deviazioni dal flusso principale causate da input inaspettati, condizioni particolari o errori. Gli scenari alternativi forniscono istruzioni su come gestire tali situazioni e ripristinare il flusso normale delle operazioni.
\subsection{Scrum} 
Scrum è un framework Agile utilizzato per lo sviluppo di prodotti complessi. Si basa su un approccio iterativo e incrementale, in cui il lavoro viene diviso in intervalli di tempo definiti chiamati sprint. Durante uno sprint, il team si impegna a completare un insieme di attività prioritarie stabilite durante la pianificazione dello sprint. Gli sprint sono seguiti da sessioni di revisione per valutare il lavoro svolto e identificare eventuali miglioramenti. 
\subsection{Sistema} 
<<<<<<< HEAD

=======
Il Sistema nel documento di Analisi dei Requisiti si riferisce all'insieme di componenti software, hardware e di sistema che collaborano per fornire le funzionalità richieste dal prodotto. Il sistema può comprendere diverse parti, come applicazioni software, database e server.
\subsection{Software} 
Il termine "Software" si riferisce all'insieme di programmi, procedure e documentazione associata utilizzati su un sistema informatico per eseguire determinate operazioni o risolvere specifici problemi. Questo include sia il codice sorgente dei programmi che il software precompilato eseguibile. Il software può essere progettato per scopi diversi, come gestione dati, elaborazione informatica, comunicazione, intrattenimento e molti altri. In un contesto di sviluppo software, il termine "software" si riferisce al prodotto finale ottenuto dal processo di sviluppo, che può essere installato e utilizzato dagli utenti per scopi specifici. 
>>>>>>> P1-135-Aggiornare-il-glossario-con-i-termini-e-le-relative-definizioni-inerenti-dell-ADR
\subsection{Sottocaso d'uso} 
Il termine "Sottocaso d'uso" si riferisce a un'istanza specifica di un caso d'uso più ampio. Mentre un caso d'uso descrive un'interazione completa tra gli attori e il sistema per raggiungere un obiettivo specifico, un sottocaso d'uso dettaglia uno specifico scenario o flusso di lavoro all'interno di quel caso d'uso più ampio. I sottocasi d'uso vengono utilizzati per suddividere i casi d'uso complessi in unità gestibili e per fornire una descrizione più dettagliata delle azioni e dei passaggi coinvolti in un'interazione specifica con il sistema. 
\subsection{Specifica tecnica} 
La Specifica tecnica è la descrizione dettagliata della progettazione tecnica del software. Questa descrizione include informazioni sulle decisioni architetturali prese, i design pattern utilizzati, la struttura dei componenti software e le interazioni tra di essi, le tecnologie impiegate e altre specifiche rilevanti per l'implementazione del sistema. La specifica tecnica fornisce una guida chiara agli sviluppatori durante la fase di codifica e contribuisce a garantire coerenza e qualità nel prodotto finale. 
\subsection{Sprint} 
Lo Sprint è una fase di sviluppo iterativa e incrementale nel framework Agile, come ad esempio Scrum. Durante uno Sprint, il team di sviluppo lavora su una serie di attività concordate all'inizio dello Sprint, con l'obiettivo di consegnare un incremento di funzionalità completamente funzionante e potenzialmente consegnabile al termine del periodo prestabilito, generalmente di durata fissa, solitamente da una a quattro settimane. Gli Sprint offrono una struttura chiara per la pianificazione, l'esecuzione e il monitoraggio del lavoro del team, consentendo un rapido adattamento ai cambiamenti dei requisiti o delle priorità. 
\subsection{Strumento} 
Uno strumento nell'ambito dello sviluppo del software è un'applicazione, un programma o una risorsa che supporta specifiche attività o processi nel ciclo di vita del software. Gli strumenti possono includere ambienti di sviluppo integrati (IDE), sistemi di gestione di progetto come Jira, framework di test come Selenium, sistemi di controllo versione come Git, e molti altri. L'uso degli strumenti giusti può migliorare l'efficienza e la qualità del processo di sviluppo del software. 
