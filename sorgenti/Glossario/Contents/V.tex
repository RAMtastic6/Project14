\section{V} 
\subsection{Verbale} 
Un verbale è un documento che riporta in modo dettagliato le discussioni, le decisioni prese e gli eventi avvenuti durante una riunione, un incontro o una sessione di lavoro. Serve a registrare e documentare in maniera accurata ciò che è stato discusso e deciso durante l'incontro, consentendo ai partecipanti di ricordare e consultare le informazioni trattate in seguito. Un verbale può includere una lista degli argomenti affrontati, i partecipanti presenti, le decisioni prese, le azioni da intraprendere e le eventuali osservazioni o commenti aggiuntivi. 
\subsection{Verbale esterno} 
Un verbale esterno è un documento che riporta le discussioni, le decisioni e gli eventi avvenuti durante una riunione o un incontro con soggetti esterni al team di lavoro o all'organizzazione. È utilizzato nel nostro caso per comunicare in modo formale e documentato con aziende proponenti, professori, o altre parti interessate esterne.
\subsection{Verbale interno} 
Un verbale interno è documento che riporta le discussioni, le decisioni e gli eventi avvenuti durante una riunione o un incontro all'interno del team di lavoro o dell'organizzazione stessa. Viene utilizzato per registrare e documentare le attività e le decisioni prese all'interno del gruppo di lavoro.
\subsection{Versionamento} 
Il versionamento è un processo di gestione delle diverse versioni di un prodotto software o di altri documenti, che consente di tracciare e registrare le modifiche apportate nel tempo. Il versionamento permette di tenere traccia delle variazioni, delle correzioni di bug e delle nuove funzionalità implementate, facilitando il controllo delle revisioni e garantendo una corretta gestione delle versioni. 
