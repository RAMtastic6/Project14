\subsection{UC9 - Modifica del proprio ordine}\label{usecase:9}
\begin{figure}[H]
    \centering
    \includegraphics[width=0.9\linewidth]{ucd/ucd9.png}
\end{figure}
\textbf{Attori}:
\begin{itemize}
    \item Utente base
\end{itemize}
\textbf{Precondizioni}:
\begin{itemize}
    \item L'utente è autenticato
    \item L'utente ha creato la propria $\textit{ordinazione}_G$ in precedenza (\nameref{usecase:3})
    \item Il tempo per l'ordinazione non è scaduto (?)
\end{itemize}
\textbf{Postcondizioni}:
\begin{itemize}
    \item L'utente ha modificato il proprio ordine e il $\textit{sistema}_G$ aggiorna le informazioni di conseguenza
\end{itemize}
\textbf{Scenario principale}:
\begin{enumerate}
    \item L'utente può compiere le seguenti azioni per gestire il proprio ordine
    \begin{enumerate}
        \item Inserire un piatto nell'ordine scegliendo tra la lista delle pietanze
        \item Scelto un piatto inserito nell'ordine si può:
        \begin{enumerate}
            \item Rimuovere ingredienti
            \item Aggiungere ingredienti
            \item Modificare la quantità della pietanza, incluso rimuoverlo
        \end{enumerate}
    \end{enumerate}
\end{enumerate}

\newpage