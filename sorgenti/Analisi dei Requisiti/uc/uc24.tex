\subsection{UC24 - Ricerca ristorante}\label{usecase:24}

\begin{figure}[H]
    \centering
    \includegraphics[width=0.9\linewidth]{ucd/UCD24.png}
    \caption{Ricerca ristorante}
\end{figure}

\textbf{Attori}:
\begin{itemize}
    \item Utente generico.
\end{itemize}
\textbf{Precondizioni}:
\begin{itemize}
    \item L'utente generico è connesso al $\textit{sistema}_G$.
\end{itemize}
\textbf{Postcondizioni}:
\begin{itemize}
    \item L'utente visualizza una lista di ristoranti che corrispondono ai criteri di ricerca.
\end{itemize}
\textbf{Scenario principale}:
\begin{enumerate}
    \item L'utente può ricercare il ristorante inserendo:
    \begin{itemize}
        \item Il nome (\nameref{usecase:24_1});
        \item La data e il luogo per escludere ristoranti chiusi(\nameref{usecase:24_2})
        \item Il tipo di cucina(\nameref{usecase:24_3});
    \end{itemize}
    \item L'utente conferma;
    \item L'utente visualizza la lista dei ristoranti che rispettano tali criteri (\nameref{usecase:24_4}).
\end{enumerate}

\subsubsection{UC24.1 - Inserimento nome ristorante
}\label{usecase:24_1}
\textbf{Attori}:
\begin{itemize}
    \item Utente generico.
\end{itemize}
\textbf{Precondizioni}:
\begin{itemize}
    \item L'utente generico è connesso al $\textit{sistema}_G$.
\end{itemize}
\textbf{Postcondizioni}:
\begin{itemize}
    \item L'utente ha inserito correttamente il nome del ristorante per la ricerca.
\end{itemize}
\textbf{Scenario principale}:
\begin{enumerate}
    \item L'utente inserisce il nome del ristorante che desidera cercare.
\end{enumerate}

\subsubsection{UC24.2 - Inserimento data e luogo
}\label{usecase:24_2}
\textbf{Attori}:
\begin{itemize}
    \item Utente generico.
\end{itemize}
\textbf{Precondizioni}:
\begin{itemize}
    \item L'utente generico è connesso al $\textit{sistema}_G$.
\end{itemize}
\textbf{Postcondizioni}:
\begin{itemize}
    \item L'utente ha inserito correttamente la data e/o il luogo per la ricerca dei ristoranti.
\end{itemize}
\textbf{Scenario principale}:
\begin{enumerate}
    \item L'utente inserisce la data per escludere i ristoranti chiusi;
    \item L'utente inserisce il luogo desiderato per la ricerca dei ristoranti.
\end{enumerate}


\subsubsection{UC24.3 - Inserimento tipologia cucina
}\label{usecase:24_3}
\textbf{Attori}:
\begin{itemize}
    \item Utente generico.
\end{itemize}
\textbf{Precondizioni}:
\begin{itemize}
    \item L'utente generico è connesso al $\textit{sistema}_G$.
\end{itemize}
\textbf{Postcondizioni}:
\begin{itemize}
    \item L'utente ha inserito correttamente la tipologia di cucina desiderata per la ricerca dei ristoranti.
\end{itemize}
\textbf{Scenario principale}:
\begin{enumerate}
    \item L'utente specifica il tipo di cucina che desidera cercare nei ristoranti.
\end{enumerate}


\subsubsection{UC24.4 - Visualizzazione lista ristoranti
}\label{usecase:24_4}
\textbf{Attori}:
\begin{itemize}
    \item Utente generico.
\end{itemize}
\textbf{Precondizioni}:
\begin{itemize}
    \item L'utente generico è connesso al $\textit{sistema}_G$;
    \item L'utente ha inserito correttamente tutti i criteri di ricerca per i ristoranti.
\end{itemize}
\textbf{Postcondizioni}:
\begin{itemize}
    \item L'utente visualizza una lista di ristoranti che corrispondono ai criteri di ricerca.
\end{itemize}
\textbf{Scenario principale}:
\begin{enumerate}
    \item Dopo aver confermato i criteri di ricerca, l'utente visualizza la lista dei ristoranti che rispettano tali criteri. Per ogni ristorante si visualizza:
    \begin{itemize}
        \item Il nome;
        \item La città e l'indirizzo.
    \end{itemize}
\end{enumerate}



\newpage