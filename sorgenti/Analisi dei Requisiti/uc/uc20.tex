\subsection{UC20 - Rifiuto richiesta prenotazione}\label{usecase:20}
\begin{figure}[H]
    \centering
    \includegraphics[width=0.9\linewidth]{ucd/ucd20.png}
\end{figure}
\textbf{Attori principali}:
\begin{itemize}
    \item Utente amministratore
\end{itemize}
\textbf{Precondizioni}:
\begin{itemize}
    \item L'amministratore è autenticato dal sistema
    \item \`E arrivata almeno una richiesta di prenotazione
\end{itemize}
\textbf{Postcondizioni}:
\begin{itemize}
    \item L'amministratore ha rifiutato la richiesta
    \item Il \textit{Sistema_G} ha notificato gli utenti del rifiuto della prenotazione
\end{itemize}
\textbf{Trigger}:
\begin{itemize}
    \item L'amministratore vuole gestire le prenotazioni arrivate
\end{itemize}
\textbf{Scenario principale}:
\begin{enumerate}
    \item L'amministratore seleziona una richiesta di prenotazione.
    \item L'amministratore controlla se ci sono posti disponibili in base alle specifiche della prenotazione.
\end{enumerate}
\newpage

\begin{comment}
\subsection{UC20 - Rifiuto prenotazione (utente amministratore)}\label{usecase:10}
\textbf{Attori}:
\begin{itemize}
    \item Utente amministratore
\end{itemize}
\textbf{Precondizioni}:
\begin{itemize}
    \item L'utente amministratore si è autenticato all'interno del sistema.
    \item Vi deve essere almeno un'ordinazione presente all'interno della lista associata al ristorante.
\end{itemize}
\textbf{Postcondizioni}:
\begin{itemize}
    \item L'amministratore ha rifiutato una prenotazione
\end{itemize}
\textbf{Scenario principale}:
\begin{enumerate}
    \item L'utente amministratore trova la lista delle prenotazioni (fare UC apposito?)
    \item L'utente amministratore sceglie una delle possibili prenotazioni da confermare e ne visualizza i dettagli associati (UC apposito per la visualizzazione di una prenotazione?)
    \item L'utente amministratore rifiuta la prenotazione.
\end{enumerate}
\textbf{Scenari secondari}:
\begin{itemize}
    \item nel caso in cui l'utente amministratore decida di non rifiutare la \textit{Prenotazione_G}, ritorna alla lista delle prenotazioni, lasciando la \textit{Prenotazione_G} inalterata.
\end{itemize}
\newpage
\end{comment}