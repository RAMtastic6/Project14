\subsection{UC2 - Registrazione come amministratore}\label{usecase:2}

\begin{figure}[H]
    \centering
    \includegraphics[width=0.9\linewidth]{ucd/UCD2_corretto.png}
\end{figure}

\textbf{Attori principali}: 
\begin{itemize}
    \item Utente non autenticato
\end{itemize}
\textbf{Precondizioni}:
\begin{itemize}
    \item L'utente non è ancora autenticato
    \item L'utente ha ricevuto un link di invito speciale che permette la registrazione come amministratore
\end{itemize}
\textbf{Postcondizioni}: 
\begin{itemize}
    \item L'utente si è registrato ed è riconosciuto come amministratore
\end{itemize}
\textbf{Trigger}:
\begin{itemize}
    \item L'utente sceglie di registrarsi come amministratore
\end{itemize}
\textbf{Scenario principale}:
\begin{enumerate}
    \item L'utente inserisce nome
    \item L'utente inserisce cognome
    \item L'utente inserisce email valida
    \item L'utente inserisce password che rispetta i criteri indicati
    \item L'utente inserisce le informazioni del ristorante (\nameref{usecase:2_1})
\end{enumerate}
\textbf{Scenari alternativi}:
\begin{itemize}
    \item 1.1.  L'utente lascia il campo nome vuoto
    
    1.2.  L'utente lascia il campo cognome vuoto
    
    1.3a.  L'utente lascia il campo email vuoto
    
    1.3b.  L'utente non inserisce una email nel formato: xxxxx@yyy.zz
    
    1.3c.  L'utente inserisce una email già registrata
    
    1.4a.  L'utente lascia il campo password vuoto
    
    1.4b.  L'utente inserisce una password troppo corta (minore di 6 caratteri)
    
    1.4c.  L'utente inserisce una password troppo lunga (maggiore di 24 caratteri)
    
    1.4d.  L'utente inserisce una password senza caratteri minuscoli
    
    1.4e.  L'utente inserisce una password senza caratteri maiuscoli
    
    1.4f.  L'utente inserisce una password senza caratteri numerici
    
    1.4g.  L'utente inserisce una password senza caratteri speciali

    1.5.  L'utente lascia il campo nome ristorante vuoto

    1.6.  L'utente lascia il campo città vuoto
    
    1.7.  L'utente lascia il campo recapiti del ristorante vuoto
    
    1.8.  L'utente lascia il campo orario apertura vuoto
    
    1.9.  L'utente lascia il campo coperti disponibili vuoto
    
    1.10.  L'utente lascia il campo tipologia cucina vuoto
    
    2.  Viene visualizzato un errore esplicativo
    
    3.  Viene data la possibilità di rifare la registrazione come amministratore
    \item L'utente decide di annullare l'operazione di registrazione
\end{itemize}