\section{Pianificazione}
\subsection{Verso la \textit{RTB}_G}

\subsubsection{Primo periodo (12/10/2023 - 30/10/2023)}

\subsubsubsection{Cose fatte}
In questo periodo il gruppo è stato formato, è stato ideato il nome (RAMtastic6), ci si è dati un recapito di posta elettronica (ramtastic6@gmail.com) ed è stato creato il logo. Inoltre, sono stati rilevati i capitolati di interesse, oltre che è stato creato il repository e l'organizzazione su \textit{Git}_GHub ufficiali del gruppo. Tra i capitolati di interesse è stato scelto di effettuare con Imola informatica un incontro durante il quale si è discusso del capitolato e riguardo a consigli sull'organizzazione del gruppo. 

\subsubsubsection{Obiettivi}
\begin{itemize}
    \item Presentare la candidatura per il capitolato C3 - \textit{Easy Meal}_G
    \item Discutere delle tecnologie da utilizzare con il proponente
    \item Effettuare un primo tentativo di organizzazione del lavoro
\end{itemize}

\subsubsubsection{Dubbi}
I dubbi riguardano principalmente la definizione degli strumenti da usare, sia a livello di realizzazione (tecnologie frontend,
backend, database, etc.), sia a livello organizzativo (versioning, project management).

\subsubsection{Secondo periodo (31/10/2023 - 13/11/2023)}
Fase iniziale di assestamento e di allineamento sulle tecnologie base da utilizzare per lo svolgimento del progetto.

\subsubsubsection{Cose fatte}
Successivamente al rifiuto della candidatura del prof. Vardanega sono stati effettuati i seguenti aggiustamenti:
\begin{itemize}
    \item riorganizzato il repository
    \item aggiunto un versionamento ai documenti
    \item aggiornati i documenti della candidatura
    \item iniziato il documento \textit{Norme di Progetto}_G
    \item assegnato un ruolo ad ogni membro
\end{itemize}

\subsubsubsection{Obiettivi}
\begin{itemize}
    \item approndire con il proponente i requisiti
    \item Sfruttare gli incontri di formazione sulle tecnologie da utilizzare fornite
dall’azienda.
\end{itemize}

\subsubsubsection{Dubbi}
Dall'analisi della candidatura era stato evidenziato uno sbilanciamento tra le ore totali di progettazione (150)
e programmazione (126); dopo una discussione il gruppo ha pensato che le ore di
progettazione devono essere minori o uguali alle ore di programmazione, inoltre le
ore del verificatore (120) potrebbero non essere bilanciate.


\subsubsection{Terzo periodo (14/11/2023 - 26/11/2023)}

\subsubsubsection{Cose fatte}
\begin{itemize}
    \item Scelto un workflow per l'uso di \textit{Git}_GHub (\textit{Git}_Gflow). (vedere \textit{Norme di Progetto}_G per approfondire)
    \item Elaborato un sistema per il versionamento dei documenti (vedere \textit{Norme di Progetto}_G per approfondire)
    \item Discusso strategie di automazione per la gestione dei documenti.
    \item Aggiornato il Documento Impegni con una corretta distribuzione oraria.
\end{itemize}

\subsubsubsection{Obiettivi}
\begin{itemize}
    \item approfondire con il proponente le tecnologie da utilizzare
    \item cominciare la stesura dei documenti \textit{Piano di Progetto}_G e Analisi dei Requisiti
    \item approfondire i linguaggi Javascript e React
\end{itemize}

\subsubsubsection{Dubbi}
\begin{itemize}
    \item sulla stesura del documento \textit{Piano di Qualifica}_G
\end{itemize}

\subsubsection{Quarto periodo (27/11/2023 - 10/12/2023)}
All'inizio di questo periodo si sono cambiati i ruoli, riportando la seguente assegnazione:
% Inserire tabella all'interno della quale mettere la suddivisione dei ruoli per questo periodo

\vspace{10 mm}
\begin{tabular}{|c|c|c|c|}
\hline
\textbf{Membro} & \textbf{Ruolo} & \textbf{Ore individuali} & \textbf{Ore produttive} \\
\hline
Samuele V. & Responsabile & 15 & 3 \\
\hline
Michele Z. & Analista & 10 & 1 \\
\hline
Leonardo B. & Analista & 5 & 2 \\
\hline
Riccardo Z. & Programmatore & 6 & N/A \\
\hline
Filippo T. & Verificatore & 3 & 3 \\
\hline
Davide B. & \textit{Amministratore}_G & 3 & 3 \\
\hline
\end{tabular}
\vspace{10 mm}

\subsubsubsection{Cose fatte}
\begin{itemize}
    \item è stato iniziato il documento di \textit{Analisi Dei Requisiti}_G e \textit{Norme di Progetto}_G
    \item è stato effettuato dello studio individuale di Javascript e React
    \item è stato effettuato un incontro con il proponente durante il quale:
    \begin{itemize}
        \item Si sono risolti dubbi sulle tecnologie da utilizzare
        \item Sono state fornite risorse e link utili per approfondire gli strumenti che andremo ad usare
        \item È stata fatta enfasi sull'uso di Docker per il deployment
        \item Si è discusso dei casi d'uso
        \item sono stati risolti i dubbi riguardanti il PoC
        \item sono stati introdotto i iframewrok da utilizzare
    \end{itemize}
    \item correzzioni su norme di progetto
    \item è stata fatta una prima rotazione dei ruoli
\end{itemize}

\subsubsubsection{Obiettivi}
\begin{itemize}
    \item ultimare lo studio dello tecnologie
    \item continuare la stesura del documento di Analisi dei Requisiti (in particolare dei casi d'uso)
    \item iniziare a sviluppare il PoC
    \item iniziare la stesura di \textit{Piano di Progetto}_G
    \item creare una prima bozza del glossario tecnico
    \item organizzare seminari con Imola informatica
\end{itemize}

\subsubsubsection{Dubbi}

I dubbi principali sono sorti durante l'uso degli USE CASE per stilare il documento di analisi dei requisiti. Inoltre ci sono stati dei dubbi riguardo la verifica corretto dello stato di avanzamento dei lavori e su come individuare in quali ambiti si stanno riscontrando criticità. 

\subsubsection{Quinto periodo (11/12/2023 - 6/1/2024)}
All'inizio di questo periodo si sono cambiati i ruoli, riportando la seguente assegnazione:

\vspace{10 mm}
\begin{tabular}{|c|c|c|c|}
\hline
\textbf{Membro} & \textbf{Ruolo} & \textbf{Ore individuali} & \textbf{Ore produttive} \\
\hline
Samuele V. & Programmatore & N/A & N/A \\
\hline
Michele Z. & Analista & 5 & 1 \\
\hline
Leonardo B. & Programmatore & 9.25 & 1 \\
\hline
Riccardo Z. & Analista & 9 & N/A \\
\hline
Filippo T. & Responsabile & N/A & N/A \\
\hline
Davide B. & Verificatore & N/A & N/A \\
\hline
\end{tabular}
\vspace{10 mm}

\subsubsubsection{Cose fatte}
\begin{itemize}
    \item è continuata la stesura del documento di Analisi dei Requisiti e Piano di progetto
    \item sono stati ruotati i ruoli
    \item è continuato lo studio individuale delle tecnologie da utilizzare
    \item è stato effettuato con il prof Cardin un incontro per risolvere i dubbi riguardo il documento Analisi dei Requisiti
    \item "Big Bang" dal punto di vista organizzativo, il quale è andato ad impattare sia l'Analisi de Requisiti che il PoC
\end{itemize}

\subsubsubsection{Obiettivi}
\begin{itemize}
    \item Delinare una prima versione del piano di progetto
    \item Continuare ad aggiornare il documento norme di progetto
    \item Iniziare a sviluppare il \textit{Poc}_G 
    \item Continuare la stesura di Analisi dei requisiti
    \item Normare i progessi organizzativi
    \item Richiedere un incontro con il proponente per discutere dei requisiti funzionali trovati mediante i casi d'uso
    \item Iniziare il glossario tecnico
\end{itemize}

In particolare quello che si vorrebbe sviluppare inizialmente è: 
\begin{itemize}
    \item items da decidere tra programmatori.
\end{itemize}

\subsubsubsection{Dubbi}

I dubbi principali sono sorti durante l'uso degli USE CASE per stilare il documento di analisi dei requisiti. Inoltre vi sono state difficoltà all'inizio del PoC perchè non tutti i membri del gruppo sono familiari con l'uso di framework per lo sviluppo.
Oltre a cioò sono state rilevate diverse difficoltà organizzative, in particolare:
\begin{itemize}
    \item Come gestire in modo ottimale il tempo
    \item Assenza di obiettivi ben definiti per ogni membro del gruppo.
\end{itemize}

\subsubsection{Sesto periodo (7/1/2024 - 15/1/2024)}
All'inizio di questo periodo si sono cambiati i ruoli, riportando la seguente assegnazione:

\vspace{10 mm}
\begin{tabular}{|c|c|c|c|}
\hline
\textbf{Membro} & \textbf{Ruolo} & \textbf{Ore individuali} & \textbf{Ore produttive} \\
\hline
Samuele V. & Analista & 4 & 1 \\
\hline
Michele Z. & \textit{Amministratore}_G & 4 & 1 \\
\hline
Leonardo B. & Responsabile & 5 e 10m & 1 e 40m \\
\hline
Riccardo Z. & Programmatore & 5 & N/A \\
\hline
Filippo T. & Verificatore & 30 & 30 \\
\hline
Davide B. & Analista & 3 & 1 e 30m \\
\hline
\end{tabular}
\vspace{10 mm}

\subsubsubsection{Cose fatte}
Periodo di assestamento causa sessione di esami
\subsubsubsection{Obiettivi}
\begin{itemize}
    \item Riprendere lo sviluppo del PoC
    \item Completare il documento di Analisi dei Requisiti
\end{itemize}
\subsubsubsection{Dubbi}
Ci sono stati rallentamenti nell'avanzamento dei lavori e difficoltà nel coordinamento delle attività causa sessione d'esami.


\subsubsection{Settimo periodo (16/1/2024 - 12/2/2024)}
All'inizio di questo periodo si sono cambiati i ruoli, riportando la seguente assegnazione:

\vspace{10 mm}
\begin{tabular}{|c|c|c|c|}
\hline
\textbf{Membro} & \textbf{Ruolo} & \textbf{Ore individuali} & \textbf{Ore produttive} \\
\hline
Samuele V. & Analista & N/A & N/A \\
\hline
Michele Z. & \textit{Amministratore}_G & N/A & N/A \\
\hline
Leonardo B. & Responsabile & 1h e 15m & N/A \\
\hline
Riccardo Z. & Programmatore & 7h e 30m & N/A \\
\hline
Filippo T. & Verificatore & N/A & N/A \\
\hline
Davide B. & Analista & N/A & N/A \\
\hline
\end{tabular}
\vspace{10 mm}

\subsubsubsection{Cose fatte}
DA RIEMPIRE
\subsubsubsection{Obiettivi}
DA RIEMPIRE
\subsubsubsection{Dubbi}
DA RIEMPIRE

\subsubsection{Ottavo periodo (13/2/2024 - 6/3/2024)}
All'inizio di questo periodo si sono cambiati i ruoli, riportando la seguente assegnazione:

\vspace{10 mm}
\begin{tabular}{|c|c|c|c|}
\hline
\textbf{Membro} & \textbf{Ruolo} & \textbf{Ore individuali} & \textbf{Ore produttive} \\
\hline
Samuele V. & Analista & N/A & N/A \\
\hline
Michele Z. & Responsabile & N/A & N/A \\
\hline
Leonardo B. & \textit{Amministratore}_G & N/A & N/A \\
\hline
Riccardo Z. & Analista & N/A & N/A \\
\hline
Filippo T. & Verificatore & N/A & N/A \\
\hline
Davide B. & Analista & N/A & N/A \\
\hline
\end{tabular}
\vspace{10 mm}

\subsubsubsection{Cose fatte}
\begin{itemize}
    \item sono stati ruotati i ruoli
    \item è continuata la stesura di Analisi dei Requisiti e quasi completata
    \item è stato effettuato un incontro con il proponente per risolvere qualche dubbio e per fare il punto della situazione
\end{itemize}

\subsubsubsection{Obiettivi}
Gli obiettivi posti per lo sprint sono stati i seguenti:
\begin{itemize}
    \item Ultimare gli ultimi UC e risolvere eventuali dubbi sul documento di Analisi dei Requisiti
    \item Iniziare a stilare l'automazione per il glossario e a redigerlo
    \item Riprendere lo sviluppo del PoC
\end{itemize}

\subsubsubsection{Dubbi}
\begin{itemize}
    \item Rallentamenti nell'avanzamento dei lavori e difficoltà nel coordinamento delle attività causa sessione d'esami.
\end{itemize}


\subsubsection{Nono periodo (7/3/2024 - ... )}
All'inizio di questo periodo si sono cambiati i ruoli, riportando la seguente assegnazione:

\vspace{10 mm}
\begin{tabular}{|c|c|c|c|}
\hline
\textbf{Membro} & \textbf{Ruolo} & \textbf{Ore attese} & \textbf{Ore effettive} \\
\hline
Samuele V. & Programmatore & 0 & 0 \\
\hline
Michele Z. & Programmatore & 0 & 0 \\
\hline
Leonardo B. & \textit{Amministratore}_G & 3 e 30m & N/A \\
\hline
Riccardo Z. & Responsabile & 4 & N/A \\
\hline
Filippo T. & Programmatore & 0 & 0 \\
\hline
Davide B. & Analista & 40m & N/A \\
\hline
\end{tabular}
\vspace{10 mm}

\subsubsubsection{Cose fatte}
Il gruppo ritiene di aver raggiunto un livello soddisfacente per poter presentare il documento di Analisi dei Requisiti ad Imola informatica e poter dedicarsi maggiormente allo sviluppo del PoC.

\subsubsubsection{Obiettivi}
Gli obiettivi posti per lo sprint sono stati i seguenti:
\begin{itemize}
    \item Studiare le tecnologie riguardanti il PoC come Next js, Nest js e Docker
    \item Aggiornare \textit{Piano di Progetto}_G
    \item Redigere \textit{Piano di Qualifica}_G
    \item Completare l'automazione e la stesura di una prima versione del glossario
    \item Completare l'automazione riguardante il sunto dei documenti (README o pagina github.io)
    \item Rivedere l'organizzazione del repo per lo sviluppo del PoC e del way of working
    \item Esplorare le potenzialità di jira per analizzare il lavoro fatto
\end{itemize}

\subsubsubsection{Dubbi}
Si sono presentati dubbi sullo sprint:
\begin{itemize}
    \item Sulla gestione delle tecnologie consigliate da \textit{Imola Informatica}_G durante l'ultimo incontro del 28 febbraio, da approfondire
    \item Sull'aggiornamento di \textit{Piano di Progetto}_G e sulla questione ore produttive/individuali
\end{itemize}


