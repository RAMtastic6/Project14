\section{Analisi dei Rischi}
Questa parte del documento si focalizza sull'elencare e descrivere le possibili fonti di rischio all'interno di un progetto $\textit{software}_G$ al fine di mitigare gli impatti delle difficoltà riscontrate.\\
Il $\textit{processo}_G$ di gestione dei $\textit{rischi}_G$ si articola essenzialmente in 4 fasi:
\begin{enumerate}
    \item \textbf{Identificazione}:\\
    in questa fase ci si concentra sull'individuare le possibili fonti di rischio che potrebbero influenzare gli obiettivi del progetto;
    \item \textbf{Analisi}:\\
    in questa fase l'intento è fornire una probabilità di occorrenza e pericolosità oltre che valutare le conseguenze dei $\textit{rischi}_G$ individuati;
    \item \textbf{Pianificazione}:\\
    consiste nell'adozione di misure per prevenire e mitigare l'incidenza e l'impatto dei $\textit{rischi}_G$ individuati e analizzati;
    \item \textbf{Controllo}:\\
    consiste nel rilevare i $\textit{rischi}_G$ durante lo svolgimento del progetto, nell'attuare delle misure di prevenzione precedentemente stabilite e raffinare le strategie, eventualmente ridefinendo la fase di $\textit{Analisi Dei Rischi}_G$.
\end{enumerate}

\subsection{Rischi tecnologici}
%
% RT1
%
\subsubsection[RT1]{RT1 - Scarsa esperienza tecnologica}\label{rt:1}
\begin{longtable}{|c|p{12cm}|}
\hline
\textbf{Descrizione} & Il gruppo ha scarsa esperienza nell'uso delle $\textit{tecnologie}_G$ scelte per lo sviluppo del prodotto $\textit{software}_G$. \\
\hline
\textbf{Occorrenza} & Alta \\
\hline
\textbf{Pericolosità} & Alta \\
\hline
\textbf{Conseguenze} & Rischio di decisioni poco efficaci e di rallentamenti nello sviluppo del progetto.\\
\hline
\textbf{Misure di prevenzione} & I membri si impegnano per:
\begin{itemize}
    \item comunicare le difficoltà incontrate con l'uso delle $\textit{tecnologie}_G$ sia con i docenti che con i proponenti.
    \item assistere agli incontri di formazione organizzati dal proponente focalizzati sulle $\textit{tecnologie}_G$ da utilizzare per il progetto
    \item organizzare dei Workshop interni in modo da incentivare l'apprendimento reciproco
\end{itemize}
 \\
\hline
\textbf{Misure di mitigazione} & I membri si impegnano ad imparare l'uso delle $\textit{tecnologie}_G$ tramite auto-formazione e, se necessario, comunicando eventuali dubbi su di esse sia alla proponente che ai docenti. \\
\hline
\caption{RT1 - rischi relativi alla scarsa esperienza nel gruppo sulle tecnologie scelte}
\label{tab:scarsa-esperienza}
\end{longtable}

%
% RT2
%
\subsubsection[RT2]{RT2 - Scarsa esperienza con gli strumenti di gestione del progetto}\label{rt:2}
\begin{longtable}{|c|p{12cm}|}
\hline
\textbf{Descrizione} & Il gruppo ha scarsa esperienza nell'uso degli strumenti a supporto dello sviluppo del prodotto $\textit{software}_G$ e ha dubbi o incertezze riguardo il loro utilizzo \\
\hline
\textbf{Occorrenza} & Bassa \\
\hline
\textbf{Pericolosità} & Media \\
\hline
\textbf{Conseguenze} & Possibili rallentamenti nello sviluppo del progetto dovuti ai possibili errori nell'uso degli strumenti\\
\hline
\textbf{Misure di prevenzione} & 
I membri si impegnano per
\begin{itemize}
    \item comunicare tra di loro le difficoltà incontrate nell'uso degli strumenti di sviluppo software
    \item inserire nel documento $\textit{Norme di Progetto}_G$ le procedure dettagliate da seguire per il Way of Working.
\end{itemize} \\
\hline
\textbf{Misure di mitigazione} & I membri si impegnano a scegliere e ad auto-apprendere strumenti $\textit{software}_G$ dotati di una buona e ricca documentazione.\\
\hline
\caption{RT2 - rischi relativi alla scarsa esperienza con gli strumenti di gestione del progetto}
\end{longtable}

\subsection{Rischi organizzativi}
%
% RO1
%
\subsubsection[RO1]{RO1 - Scarsa esperienza nell'organizzazione di un progetto complesso}  \label{ro:1} 
 
 
\begin{longtable}{|c|p{12cm}|}
\hline
\textbf{Descrizione} & Il gruppo ha scarsa esperienza nell'organizzazione di un progetto complesso \\
\hline
\textbf{Occorrenza} & Alta \\
\hline
\textbf{Pericolosità} & Alta \\
\hline
\textbf{Conseguenze} & Imprecisioni sulla pianificazione delle attività e sottostima/sovrastima delle risorse e del tempo necessari\\
\hline
\textbf{Misure di prevenzione} & I membri si impegnano per comunicare tempestivamente le difficoltà incontrate durante lo svolgimento delle loro attività e a riportare periodicamente lo stato di avanzamento nel documento $\textit{Piano di Progetto}_G$ \\
\hline
\textbf{Misure di mitigazione} & I membri si impegnano ad instaurare frequentemente una comunicazione sia interna che esterna verso il proponente e i docenti di corso oltre che revisionare il $\textit{Piano di Progetto}_G$ per adeguare le attività in base al progresso\\
\hline
\caption{RO1 - rischi relativi alla scarsa esperienza nell'organizzazione di un progetto complesso}
\end{longtable}

%
% RO2
%
\subsubsection[RO2]{RO2 - Comunicazione interna inefficace}\label{ro:2}
\begin{longtable}{|c|p{12cm}|}
\hline
\textbf{Descrizione} & Alcune comunicazioni importanti per l'avanzamento del progetto potrebbero essere poco chiare o poco efficaci\\
\hline
\textbf{Occorrenza} & Media \\
\hline
\textbf{Pericolosità} & Media \\
\hline
\textbf{Conseguenze} & Fraintendimenti, duplicazione e discrepanze dalle aspettative, rallentamento del lavoro \\
\hline
\textbf{Misure di prevenzione} & I membri si impegnano a comunicare le informazioni ad ogni altro membro \\
\hline
\textbf{Misure di mitigazione} & I membri presenti alle riunioni interne al gruppo si impegnano a comunicare l'oggetto delle discussioni avvenute mediante gli strumenti di comunicazione interna e $\textit{verbali interni}_G$ oltre che usare gli appositi canali comunicativi stabiliti ed eventualmente richiedere incontri non pianificati\\
\hline
\caption{RO2 - rischi relativi ad una comunicazione interna inefficace}
\end{longtable}

%
%R03
%
\subsubsection[RO3]{RO3 - Impegni personali e universitari}\label{ro:3}
\begin{longtable}{|c|p{12cm}|}
\hline
\textbf{Descrizione} & Alcuni membri del gruppo potrebbero non essere reperibili durante incontri interni al gruppo nei quali vengono prese delle decisioni importanti per l'avanzamento del progetto. Alcuni impegni personali e universitari potrebbero limitare la disponibilità temporale di alcuni membri del gruppo. \\
\hline
\textbf{Occorrenza} & Media \\
\hline
\textbf{Pericolosità} & Media \\
\hline
\textbf{Conseguenze} & Rallentamento del lavoro, aumento dei costi, riduzione della qualità del lavoro, pianificazione errata \\
\hline
\textbf{Misure di prevenzione} & I membri si impegnano per comunicare tempestivamente le loro disponibilità al Responsabile così da pianificare attentamente le attività. \\
\hline
\textbf{Misure di mitigazione} &  il Responsabile si riserva di spostare alcune scadenze se la pianificazione non tiene conto di alcuni periodi\\
\hline
\caption{RO3 - Impegni personali e universitari}
\end{longtable}

%
%R04
%
\subsubsection[RO4]{RO4 - Mancanza di chiarezza nei ruoli e delle attività}\label{ro:4}
\begin{longtable}{|c|p{12cm}|}
\hline
\textbf{Descrizione} & La mancanza di chiarezza riguardo ai ruoli e alle responsabilità all’interno del team può generare confusione, conflitti e ritardi nelle attività. \\
\hline
\textbf{Occorrenza} & Media \\
\hline
\textbf{Pericolosità} & Media \\
\hline
\textbf{Conseguenze} & Rallentamento delle attività, aumento del rischio di errori, scarsa coesione del team, inefficienza nella gestione delle risorse \\
\hline
\textbf{Misure di prevenzione} & Comunicazioni chiare e complete riguardo ai compiti e alle responsabilità, incontri regolari per chiarire dubbi e garantire consapevolezza dei compiti individuali oltre che approfondire le sezioni apposite nel documento $\textit{Norme di Progetto}_G$. \\
\hline
\textbf{Misure di mitigazione} & Stesura e costante aggiornamento di una chiara matrice dei ruoli e responsabilità, consultazione regolare per garantire comprensione e adesione, possibilità di spostare scadenze in caso di necessità. \\
\hline
\caption{RO4 - Mancanza di chiarezza nei ruoli e delle attività}
\end{longtable}

%
%R05
%
\subsubsection[RO5]{RO5 - Inesperienza nell'esecuzione di attività}\label{ro:5}
\begin{longtable}{|c|p{12cm}|}
\hline
\textbf{Descrizione} & La mancanza di esperienza pregressa nell'esecuzione di attività che richiedono competenze specifiche possono rallentare il completamento delle stesse generando ritardi considerevoli. \\
\hline
\textbf{Occorrenza} & Alta \\
\hline
\textbf{Pericolosità} & Alta \\
\hline
\textbf{Conseguenze} & Rallentamento delle attività, aumento del rischio di errori \\
\hline
\textbf{Misure di prevenzione} & I membri del gruppo si impegnato a comunicare al responsabile eventuali difficoltà riscontrate durante l'esecuzione di determinate attività. \\
\hline
\textbf{Misure di mitigazione} &  I membri del gruppo si impegnano ad informarsi e a coinvolgere il proponente e il committente per ottenere una consulenza tempestiva.\\
\hline
\caption{RO5 - Inesperienza nell'esecuzione di attività}
\end{longtable}

\subsection{Rischi relativi al prodotto}
%
%RP1
%
\subsubsection[RP1]{RP1 - Comprensione erronea dei requisiti}\label{rp:1}
\begin{longtable}{|c|p{12cm}|}
\hline
\textbf{Descrizione} & L'azienda potrebbe essere non soddisfatta dalle modalità scelte nell'attuare lo sviluppo del prodotto\\
\hline
\textbf{Occorrenza} & Bassa \\
\hline
\textbf{Pericolosità} & Alta \\
\hline
\textbf{Misure di prevenzione} & I membri si impegnano a prendere visione del $\textit{capitolato}_G$, in modo da avere una chiara comprensione delle necessità e dei vincoli imposti dal cliente. \\
\hline
\textbf{Misure di mitigazione} &  I membri si impegnano ad instaurare una comunicazione che sia più frequente possibile con la proponente ogni qual volta vi siano dei dubbi in merito alle funzionalità da implementare.\\
\hline
\caption{RP1 - rischi relativi ad un erronea comprensione dei requisiti}
\end{longtable}

\subsection{Riassunto dei rischi individuati}
\begin{table}[h]
\centering
\begin{tabular}{|c|c|c|c|}
\hline
\textbf{Rischio} & \textbf{Nome} & \textbf{Occorrenza} & \textbf{Pericolosità} \\
\hline
\textbf{RT1} & Scarsa esperienza tecnologica & Alta & Alta \\
\hline
\textbf{RT2} & Scarsa esperienza con gli strumenti di gestione del progetto & Bassa & Media \\
\hline
\textbf{RO1} & Scarsa esperienza nell’organizzazione di un progetto complesso & Alta & Alta \\
\hline
\textbf{RO2} & Comunicazione interna inefficace & Media & Media \\
\hline
\textbf{RO3} & Impegni personali e universitari & Media & Media \\
\hline
\textbf{RO4} & Mancanza di chiarezza nei ruoli e delle attività & Media & Media \\
\hline
\textbf{RO5} & Inesperienza nell'esecuzione di attività & Alta & Alta \\
\hline
\textbf{RP1} & Comprensione erronea dei requisiti & Bassa & Alta \\
\hline
\end{tabular}
\caption{Tabella riassuntiva dei rischi individuati}
\label{tab:rischi}
\end{table}