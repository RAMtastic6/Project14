\documentclass[12pt, oneside]{article} 
\usepackage{amsmath, amsthm, amssymb, calrsfs, wasysym, verbatim, bbm, color, graphicx, geometry, fancyhdr, url, multirow, hyperref, titlesec, tabularx, spreadtab, pgf-pie}
\usepackage{pgf-pie}
\usepackage{tikz}
\usepackage{fp}
\usepackage{pgfplots}
\pgfplotsset{compat=1.17}
\usepackage{csvsimple}
\usepackage{pgfplotstable}
\usepackage[italian]{babel}

\geometry{tmargin=.75in, bmargin=.75in, lmargin=.75in, rmargin = .75in}

% Creo subsubsubsection
\titleclass{\subsubsubsection}{straight}[\subsection]

\newcounter{subsubsubsection}[subsubsection]
\renewcommand\thesubsubsubsection{\thesubsubsection.\arabic{subsubsubsection}}
\renewcommand\theparagraph{\thesubsubsubsection.\arabic{paragraph}} % optional; useful if paragraphs are to be numbered

\titleformat{\subsubsubsection}
  {\normalfont\normalsize\bfseries}{\thesubsubsubsection}{1em}{}
\titlespacing*{\subsubsubsection}
{0pt}{3.25ex plus 1ex minus .2ex}{1.5ex plus .2ex}

\makeatletter
\renewcommand\paragraph{\@startsection{paragraph}{5}{\z@}%
  {3.25ex \@plus1ex \@minus.2ex}%
  {-1em}%
  {\normalfont\normalsize\bfseries}}
\renewcommand\subparagraph{\@startsection{subparagraph}{6}{\parindent}%
  {3.25ex \@plus1ex \@minus .2ex}%
  {-1em}%
  {\normalfont\normalsize\bfseries}}
\def\toclevel@subsubsubsection{4}
\def\toclevel@paragraph{5}
\def\toclevel@paragraph{6}
\def\l@subsubsubsection{\@dottedtocline{4}{7em}{4em}}
\def\l@paragraph{\@dottedtocline{5}{10em}{5em}}
\def\l@subparagraph{\@dottedtocline{6}{14em}{6em}}
\makeatother

\setcounter{secnumdepth}{4}
\setcounter{tocdepth}{4}
% Fine creazione subsubsubsection

\author{RAMtastic6}

%Intestazione
\pagestyle{fancy}
\fancyhf{}
\fancyhead[R]{Gruppo 14 RAMtastic6\\ramtastic6@gmail.com}
\fancyfoot[C]{\thepage}

% Linea intestazione
\renewcommand{\headrulewidth}{0pt} 

% Intestazione documento
\begin{document}
% Salta la prima pagina per l'intestazione
\thispagestyle{empty}
\title{\textit{Piano di Progetto}_G}
\maketitle
\begin{figure}[h]
  \centering
  \includegraphics[scale=0.3]{logo.png}
\end{figure}
\begin{center}
    email: ramtastic6@gmail.com
\end{center}

% Informazioni sul documento
\section*{Informazioni sul documento}
\begin{tabular}{ll}
Versione: & 0.2.0 \\
Redattori: & Leonardo B. Riccardo Z.\\
Verificatori: & Davide B. Filippo T.\\ 
Destinatari: & T. Vardanega, R. Cardin, \textit{Imola Informatica}_G \\
Uso: & Interno
\end{tabular}
\newpage

% Registro dei cambiamenti
\section*{Registro dei Cambiamenti - Changelog}
\begin{tabular}{|c|c|c|p{3cm}|p{6cm}|}
\hline
\textbf{Versione} & \textbf{Data} & \textbf{Autore} & \textbf{Verificatore} & \textbf{Dettaglio} \\
\hline
v.0.2.1 & N/A & Riccardo Z. & N/A & Aggiunta dei periodi mancanti nella sezione 4 "Pianificazione", aggiunta delle sezione 5 "Preventivi" e 6 "Consuntivi" \\
\hline
v.0.2.0 & 19-12-2023 & Leonardo B. & Filippo T. & Stesura di una prima versione dell'analisi dei rischi. \\
\hline
v.0.1.0 & 19-12-2023 & Leonardo B. & Brotto D. & Stesura della struttura del documento \\
\hline
\end{tabular}
\newpage


% Sommario
\tableofcontents
\newpage
% Introduzione
\section{Introduzione}
\subsection{Scopo del documento}
Il presente documento si propone di definire le metriche e le metodologie di controllo e misurazione necessarie per garantire la qualità del prodotto e del $\textit{processo}_G$. In particolare, le metriche di valutazione del prodotto sono correlate ai requisiti e alle aspettative del fornitore.
Il $\textit{Piano di Qualifica}_G$ è concepito per essere dinamico ed incrementale, in particolar modo per quanto riguarda le metriche descritte e mira a fornire una valutazione il più obiettiva possibile di ciò che è stato realizzato.\\
Le procedure del $\textit{Way Of Working}_G$ devono essere costantemente osservate e migliorate, al fine di garantire che il prodotto soddisfi le aspettative del cliente e mantenga gli standard di qualità richiesti. Eventuali termini tecnici sono definiti all'interno del documento "Glossario Tecnico".
\subsection{Scopo del prodotto}
Il prodotto finale, realizzato tramite un'$\textit{Applicazione Web Responsive}_G$, si propone di realizzare un $\textit{software}_G$ innovativo volto a semplificate il $\textit{processo}_G$ di $\textit{prenotazione}_G$ e $\textit{ordinazione}_G$ nei ristoranti, contribuendo a migliorare l'esperienza per clienti e ristoratori. In particolare, \textit{Easy Meal} dovrà consentire agli utenti di personalizzare gli ordini in base alle proprie preferenze, allergie ed esigenze alimentari; interagire direttamente con lo staff del ristorante attraverso una chat integrata e in ultimo, consentire di dividere il conto tra i partecipanti al tavolo.
\subsection{Riferimenti}
\subsubsection{Riferimenti normativi}
\begin{enumerate}
    \item $\textit{Norme di Progetto}_G$ v2.0.0
    \item Presentazione del $\textit{capitolato}_G$ d'appalto C3 - Progetto $\textit{Easy Meal}_G$: \\ 
    \url{https://www.math.unipd.it/~tullio/IS-1/2023/Progetto/C3.pdf}
    \item Regolamento del progetto didattico: \\ 
    \url{https://www.math.unipd.it/~tullio/IS-1/2023/Dispense/PD2.pdf}
\end{enumerate}
\subsubsection{Riferimenti informativi}
\label{sec:rif_inf}
\begin{enumerate}
    \item Lezione \emph{"Progettazione $\textit{software}_G$ (T6)"} del corso di $\textit{Ingegneria del $\textit{software}_G$}_G$: \\
    \url{https://www.math.unipd.it/~tullio/IS-1/2023/Dispense/T6.pdf}
    \item Lezione \emph{"Qualità del $\textit{software}_G$ (T7)"} del corso di $\textit{Ingegneria del $\textit{software}_G$}_G$: \\
    \url{https://www.math.unipd.it/~tullio/IS-1/2023/Dispense/T7.pdf}
    \item Lezione \emph{"Qualità di $\textit{processo}_G$ (T8)"} del corso di $\textit{Ingegneria del software}_G$: \\
    \url{https://www.math.unipd.it/~tullio/IS-1/2023/Dispense/T8.pdf}
    \item Lezione \emph{"Verifica e validazione: introduzione (T9)"} del corso di $\textit{Ingegneria del software}_G$: \\
    \url{https://www.math.unipd.it/~tullio/IS-1/2023/Dispense/T9.pdf}
    \item Lezione \emph{"Verifica e validazione: analisi statica (T10)"} del corso di $\textit{Ingegneria del software}_G$: \\
    \url{https://www.math.unipd.it/~tullio/IS-1/2023/Dispense/T10.pdf}
    \item Lezione \emph{"Verifica e validazione: analisi dinamica (T11)"} del corso di $\textit{Ingegneria del software}_G$: \\
    \url{https://www.math.unipd.it/~tullio/IS-1/2023/Dispense/T11.pdf}
     \item Documento \emph{"Dichiarazione impegni v1.2"}: \\ \url{https://github.com/RAMtastic6/Project14/blob/main/documenti/CANDIDATURA/documento_impegni_v1.2.pdf}
     \item Metriche di progetto (\emph{Earned Value Analysis}):\\
     \url{https://it.wikipedia.org/wiki/Metriche_di_progetto}
     \item Glossario v2.0.0;
     \item Analisi dei Requisiti v3.0.0.
\end{enumerate}
\subsection{Codifica delle metriche}
In questa sottosezione verranno definite le metriche che utilizzeremo, utilizzando un codice standardizzato.

Una metrica è identificata dal seguente formato di codice:
\[
\text{M[Tipo][Id]-[Acronimo]}
\]

Dove:
\begin{itemize}
    \item \textbf{M} sta per "Metrica"
    \item \textbf{Tipo} può essere PC (per un $\textit{processo}_G$) o PD (per un prodotto)
    \item \textbf{Id} rappresenta un identificativo all'interno di una metrica di un certo tipo
    \item \textbf{Acronimo} indica l'acronimo del nome della metrica utilizzata
\end{itemize}

Per ciascuna metrica verranno fornite descrizioni, valori accettabili e valori preferibili.
\subsection{Codifica dei test}
In questa sottosezione verranno definiti i $\textit{test}_G$ che utilizzeremo, utilizzando un codice standardizzato.

Un $\textit{test}_G$ è identificato dal seguente formato di codice:
\[
\text{T[Tipo]-[Id]}
\]

Dove:
\begin{itemize}
    \item \textbf{T} sta per "$\textit{Test}_G$"
    \item \textbf{Tipo} può essere S (di $\textit{sistema}_G$) o I (di integrazione) o U (di unità) oppure A (di accettazione)
    \item \textbf{Id} rappresenta un identificativo all'interno di un $\textit{test}_G$ di un certo tipo
\end{itemize}

Per ciascun $\textit{test}_G$ verranno fornite descrizioni e il loro stato se implementato o meno oltre che un loro tracciamento.
\newpage
% Analisi dei Rischi
\section{Analisi dei Rischi}
Questa parte del documento si focalizza sull'elencare e descrivere le possibili fonti di rischio all'interno di un progetto $\textit{software}_G$ al fine di mitigare gli impatti delle difficoltà riscontrate.\\
Il $\textit{processo}_G$ di gestione dei $\textit{rischi}_G$ si articola essenzialmente in 4 fasi:
\begin{enumerate}
    \item \textbf{Identificazione}:\\
    in questa fase ci si concentra sull'individuare le possibili fonti di rischio che potrebbero influenzare gli obiettivi del progetto;
    \item \textbf{Analisi}:\\
    in questa fase l'intento è fornire una probabilità di occorrenza e pericolosità oltre che valutare le conseguenze dei $\textit{rischi}_G$ individuati;
    \item \textbf{Pianificazione}:\\
    consiste nell'adozione di misure per prevenire e mitigare l'incidenza e l'impatto dei $\textit{rischi}_G$ individuati e analizzati;
    \item \textbf{Controllo}:\\
    consiste nel rilevare i $\textit{rischi}_G$ durante lo svolgimento del progetto, nell'attuare delle misure di prevenzione precedentemente stabilite e raffinare le strategie, eventualmente ridefinendo la fase di $\textit{Analisi Dei Rischi}_G$.
\end{enumerate}

\subsection{Rischi tecnologici}
%
% RT1
%
\subsubsection[RT1]{RT1 - Scarsa esperienza tecnologica}\label{rt:1}
\begin{longtable}{|c|p{12cm}|}
\hline
\textbf{Descrizione} & Il gruppo ha scarsa esperienza nell'uso delle $\textit{tecnologie}_G$ scelte per lo sviluppo del prodotto $\textit{software}_G$. \\
\hline
\textbf{Occorrenza} & Alta \\
\hline
\textbf{Pericolosità} & Alta \\
\hline
\textbf{Conseguenze} & Rischio di decisioni poco efficaci e di rallentamenti nello sviluppo del progetto.\\
\hline
\textbf{Misure di prevenzione} & I membri si impegnano per:
\begin{itemize}
    \item comunicare le difficoltà incontrate con l'uso delle $\textit{tecnologie}_G$ sia con i docenti che con i proponenti.
    \item assistere agli incontri di formazione organizzati dal proponente focalizzati sulle $\textit{tecnologie}_G$ da utilizzare per il progetto
    \item organizzare dei Workshop interni in modo da incentivare l'apprendimento reciproco
\end{itemize}
 \\
\hline
\textbf{Misure di mitigazione} & I membri si impegnano ad imparare l'uso delle $\textit{tecnologie}_G$ tramite auto-formazione e, se necessario, comunicando eventuali dubbi su di esse sia alla proponente che ai docenti. \\
\hline
\caption{RT1 - rischi relativi alla scarsa esperienza nel gruppo sulle tecnologie scelte}
\label{tab:scarsa-esperienza}
\end{longtable}

%
% RT2
%
\subsubsection[RT2]{RT2 - Scarsa esperienza con gli strumenti di gestione del progetto}\label{rt:2}
\begin{longtable}{|c|p{12cm}|}
\hline
\textbf{Descrizione} & Il gruppo ha scarsa esperienza nell'uso degli strumenti a supporto dello sviluppo del prodotto $\textit{software}_G$ e ha dubbi o incertezze riguardo il loro utilizzo \\
\hline
\textbf{Occorrenza} & Bassa \\
\hline
\textbf{Pericolosità} & Media \\
\hline
\textbf{Conseguenze} & Possibili rallentamenti nello sviluppo del progetto dovuti ai possibili errori nell'uso degli strumenti\\
\hline
\textbf{Misure di prevenzione} & 
I membri si impegnano per
\begin{itemize}
    \item comunicare tra di loro le difficoltà incontrate nell'uso degli strumenti di sviluppo software
    \item inserire nel documento $\textit{Norme di Progetto}_G$ le procedure dettagliate da seguire per il Way of Working.
\end{itemize} \\
\hline
\textbf{Misure di mitigazione} & I membri si impegnano a scegliere e ad auto-apprendere strumenti $\textit{software}_G$ dotati di una buona e ricca documentazione.\\
\hline
\caption{RT2 - rischi relativi alla scarsa esperienza con gli strumenti di gestione del progetto}
\end{longtable}

\subsection{Rischi organizzativi}
%
% RO1
%
\subsubsection[RO1]{RO1 - Scarsa esperienza nell'organizzazione di un progetto complesso}  \label{ro:1} 
 
 
\begin{longtable}{|c|p{12cm}|}
\hline
\textbf{Descrizione} & Il gruppo ha scarsa esperienza nell'organizzazione di un progetto complesso \\
\hline
\textbf{Occorrenza} & Alta \\
\hline
\textbf{Pericolosità} & Alta \\
\hline
\textbf{Conseguenze} & Imprecisioni sulla pianificazione delle attività e sottostima/sovrastima delle risorse e del tempo necessari\\
\hline
\textbf{Misure di prevenzione} & I membri si impegnano per comunicare tempestivamente le difficoltà incontrate durante lo svolgimento delle loro attività e a riportare periodicamente lo stato di avanzamento nel documento $\textit{Piano di Progetto}_G$ \\
\hline
\textbf{Misure di mitigazione} & I membri si impegnano ad instaurare frequentemente una comunicazione sia interna che esterna verso il proponente e i docenti di corso oltre che revisionare il $\textit{Piano di Progetto}_G$ per adeguare le attività in base al progresso\\
\hline
\caption{RO1 - rischi relativi alla scarsa esperienza nell'organizzazione di un progetto complesso}
\end{longtable}

%
% RO2
%
\subsubsection[RO2]{RO2 - Comunicazione interna inefficace}\label{ro:2}
\begin{longtable}{|c|p{12cm}|}
\hline
\textbf{Descrizione} & Alcune comunicazioni importanti per l'avanzamento del progetto potrebbero essere poco chiare o poco efficaci\\
\hline
\textbf{Occorrenza} & Media \\
\hline
\textbf{Pericolosità} & Media \\
\hline
\textbf{Conseguenze} & Fraintendimenti, duplicazione e discrepanze dalle aspettative, rallentamento del lavoro \\
\hline
\textbf{Misure di prevenzione} & I membri si impegnano a comunicare le informazioni ad ogni altro membro \\
\hline
\textbf{Misure di mitigazione} & I membri presenti alle riunioni interne al gruppo si impegnano a comunicare l'oggetto delle discussioni avvenute mediante gli strumenti di comunicazione interna e $\textit{verbali interni}_G$ oltre che usare gli appositi canali comunicativi stabiliti ed eventualmente richiedere incontri non pianificati\\
\hline
\caption{RO2 - rischi relativi ad una comunicazione interna inefficace}
\end{longtable}

%
%R03
%
\subsubsection[RO3]{RO3 - Impegni personali e universitari}\label{ro:3}
\begin{longtable}{|c|p{12cm}|}
\hline
\textbf{Descrizione} & Alcuni membri del gruppo potrebbero non essere reperibili durante incontri interni al gruppo nei quali vengono prese delle decisioni importanti per l'avanzamento del progetto. Alcuni impegni personali e universitari potrebbero limitare la disponibilità temporale di alcuni membri del gruppo. \\
\hline
\textbf{Occorrenza} & Media \\
\hline
\textbf{Pericolosità} & Media \\
\hline
\textbf{Conseguenze} & Rallentamento del lavoro, aumento dei costi, riduzione della qualità del lavoro, pianificazione errata \\
\hline
\textbf{Misure di prevenzione} & I membri si impegnano per comunicare tempestivamente le loro disponibilità al Responsabile così da pianificare attentamente le attività. \\
\hline
\textbf{Misure di mitigazione} &  il Responsabile si riserva di spostare alcune scadenze se la pianificazione non tiene conto di alcuni periodi\\
\hline
\caption{RO3 - Impegni personali e universitari}
\end{longtable}

%
%R04
%
\subsubsection[RO4]{RO4 - Mancanza di chiarezza nei ruoli e delle attività}\label{ro:4}
\begin{longtable}{|c|p{12cm}|}
\hline
\textbf{Descrizione} & La mancanza di chiarezza riguardo ai ruoli e alle responsabilità all’interno del team può generare confusione, conflitti e ritardi nelle attività. \\
\hline
\textbf{Occorrenza} & Media \\
\hline
\textbf{Pericolosità} & Media \\
\hline
\textbf{Conseguenze} & Rallentamento delle attività, aumento del rischio di errori, scarsa coesione del team, inefficienza nella gestione delle risorse \\
\hline
\textbf{Misure di prevenzione} & Comunicazioni chiare e complete riguardo ai compiti e alle responsabilità, incontri regolari per chiarire dubbi e garantire consapevolezza dei compiti individuali oltre che approfondire le sezioni apposite nel documento $\textit{Norme di Progetto}_G$. \\
\hline
\textbf{Misure di mitigazione} & Stesura e costante aggiornamento di una chiara matrice dei ruoli e responsabilità, consultazione regolare per garantire comprensione e adesione, possibilità di spostare scadenze in caso di necessità. \\
\hline
\caption{RO4 - Mancanza di chiarezza nei ruoli e delle attività}
\end{longtable}

%
%R05
%
\subsubsection[RO5]{RO5 - Inesperienza nell'esecuzione di attività}\label{ro:5}
\begin{longtable}{|c|p{12cm}|}
\hline
\textbf{Descrizione} & La mancanza di esperienza pregressa nell'esecuzione di attività che richiedono competenze specifiche possono rallentare il completamento delle stesse generando ritardi considerevoli. \\
\hline
\textbf{Occorrenza} & Alta \\
\hline
\textbf{Pericolosità} & Alta \\
\hline
\textbf{Conseguenze} & Rallentamento delle attività, aumento del rischio di errori \\
\hline
\textbf{Misure di prevenzione} & I membri del gruppo si impegnato a comunicare al responsabile eventuali difficoltà riscontrate durante l'esecuzione di determinate attività. \\
\hline
\textbf{Misure di mitigazione} &  I membri del gruppo si impegnano ad informarsi e a coinvolgere il proponente e il committente per ottenere una consulenza tempestiva.\\
\hline
\caption{RO5 - Inesperienza nell'esecuzione di attività}
\end{longtable}

\subsection{Rischi relativi al prodotto}
%
%RP1
%
\subsubsection[RP1]{RP1 - Comprensione erronea dei requisiti}\label{rp:1}
\begin{longtable}{|c|p{12cm}|}
\hline
\textbf{Descrizione} & L'azienda potrebbe essere non soddisfatta dalle modalità scelte nell'attuare lo sviluppo del prodotto\\
\hline
\textbf{Occorrenza} & Bassa \\
\hline
\textbf{Pericolosità} & Alta \\
\hline
\textbf{Misure di prevenzione} & I membri si impegnano a prendere visione del $\textit{capitolato}_G$, in modo da avere una chiara comprensione delle necessità e dei vincoli imposti dal cliente. \\
\hline
\textbf{Misure di mitigazione} &  I membri si impegnano ad instaurare una comunicazione che sia più frequente possibile con la proponente ogni qual volta vi siano dei dubbi in merito alle funzionalità da implementare.\\
\hline
\caption{RP1 - rischi relativi ad un erronea comprensione dei requisiti}
\end{longtable}

\subsection{Riassunto dei rischi individuati}
\begin{table}[h]
\centering
\begin{tabular}{|c|c|c|c|}
\hline
\textbf{Rischio} & \textbf{Nome} & \textbf{Occorrenza} & \textbf{Pericolosità} \\
\hline
\textbf{RT1} & Scarsa esperienza tecnologica & Alta & Alta \\
\hline
\textbf{RT2} & Scarsa esperienza con gli strumenti di gestione del progetto & Bassa & Media \\
\hline
\textbf{RO1} & Scarsa esperienza nell’organizzazione di un progetto complesso & Alta & Alta \\
\hline
\textbf{RO2} & Comunicazione interna inefficace & Media & Media \\
\hline
\textbf{RO3} & Impegni personali e universitari & Media & Media \\
\hline
\textbf{RO4} & Mancanza di chiarezza nei ruoli e delle attività & Media & Media \\
\hline
\textbf{RO5} & Inesperienza nell'esecuzione di attività & Alta & Alta \\
\hline
\textbf{RP1} & Comprensione erronea dei requisiti & Bassa & Alta \\
\hline
\end{tabular}
\caption{Tabella riassuntiva dei rischi individuati}
\label{tab:rischi}
\end{table}
\newpage
% Modello di sviluppo
\section{Modello di sviluppo}
\subsection{Modello agile}
Il modello di sviluppo scelto è quello Agile ed in particolare di utilizzare il \textit{Framework_G} \emph{Scrum}. Nel documento \emph{Norme di Progetto} nella sezione 4.3.1 è esplicitato nel dettaglio il modello utilizzato.\\ 
Tale modello si basa sull'adattabilità, sulla flessibilità e sulla collaborazione.
In particolare si punta in tale modello ai seguenti obiettivi:
\begin{itemize}
    \item \textbf{Iterazioni}: lo sviluppo avviene tramite "iterazioni" o "sprint" ciascuno dei quali produce un incremento significativo e funzionante del prodotto;
    \item \textbf{Adattabilità al cambiamento}: lo sviluppo tiene conto del fatto che i requisiti e le priorità possano cambiare nel corso del progetto e si propone di rispondere prontamente a tali cambiamenti;
    \item \textbf{Gestione mirata di rischi}: data l'organizzazione su brevi periodi (di media di due settimane) i problemi verranno dovranno essere individuati velocemente e avranno una risoluzione tempestiva;
    \item \textbf{Collaborazione con il proponente}: viene promosso il coinvolgimento del proponente durante tutto il processo di sviluppo per aderire alle esigenze rivolte al prodotto finale.
\end{itemize}
\newpage
% Pianificazione
\section{Pianificazione}
In tale sezione, aggiornata progressivamente, sono riportate le descrizioni dei vari periodi e i relativi sprint.
La pianificazione è suddivisa nelle seguenti fasi:
\begin{enumerate}
    \item Studio dei capitolati e assegnazione dell'appalto;
    \item Verso la RTB (\emph{Requirements and Technology Baseline});
    \item Verso la PB (\emph{Product Baseline}).
\end{enumerate}

\subsection{Studio dei capitolati e assegnazione dell'appalto}
In questa fase preliminare il gruppo, appena formato, discute dei capitolati e si candida per uno di essi. Una volta aggiudicato l'appalto dal prof. Vardanega inizierà la fase successiva ovvero la RTB (\emph{Requirements and Technology Baseline}).

\begin{table}[h]
\centering
\captionsetup{justification=centering}

\begin{tabular}{|c|c|c|}
\hline
\textbf{Data di inizio} & \textbf{Data di fine prevista} & \textbf{Data di assegnazione dell'appalto} \\
\hline
14/10/2023 & 6/11/2023 & 8/11/2023 \\
\hline
\end{tabular}
\caption{Periodo di studio dei capitolati e assegnazione dell'appalto}
\end{table}

\subsubsection{Attività svolte}
Durante questa fase preliminare, dopo la formazione del gruppo \emph{RAMtastic6}, sono state svolte diverse attività. \\
Inizialmente, sono stati individuati i capitolati di interesse e, dopo varie riunioni interne, si è deciso di presentare la candidatura per il capitolato C3, \emph{Easy Meal}. A supporto di questa decisione, è stata creata un'organizzazione omonima al gruppo e un repository chiamato \emph{Project14} sulla piattaforma \emph{GitHub}. 

Successivamente, è stato organizzato un incontro con \emph{Imola Informatica}, il proponente del capitolato C3, per discutere dei dettagli.

Inizialmente i dubbi principali che sono emersi riguardavano la definizione degli strumenti da utilizzare, sia a livello di realizzazione (tecnologie front-end, back-end, database, ecc.), sia a livello organizzativo (versioning, project management).

Nel secondo periodo, che va dal 31 ottobre 2023 al 6 novembre 2023, sono state pianificate e svolte ulteriori attività. In particolare, si è proceduto con la presentazione della candidatura per il capitolato C3 - Easy Meal tramite i relativi documenti. In particolare è stata redatta la seguente documentazione: \emph{"Preventivo costi e assunzione impegni"}, \emph{"Lettera di presentazione"} e \emph{"Valutazione dei capitolati"}. In ultimo si è effettuato un primo tentativo di organizzazione del lavoro.

Durante questo periodo, la candidatura è stata valutata dal Prof. Vardanega ed è stata inizialmente rifiutata; tuttavia, dopo varie riunioni interne, sono stati apportati diversi miglioramenti. Il repository è stato riorganizzato, è stato introdotto un sistema di versionamento per i documenti, i documenti della candidatura sono stati aggiornati, è stato iniziato il documento \emph{"Norme di Progetto"} e sono stati assegnati ruoli specifici a ciascun membro del gruppo. In particolare, è stato modificato il documento di \emph{"Dichiarazione degli impegni"} con la corretta retribuzione oraria.

Nella retrospettiva dopo la valutazione della candidatura da parte del prof. Vardanega, è emerso il dubbio riguardante uno sbilanciamento tra le ore totali di progettazione (150) e programmazione (126) evidenziato dall'analisi della candidatura. Dopo una discussione interna, il gruppo ha ritenuto necessario che le ore di progettazione siano minori o uguali alle ore di programmazione, inoltre si è notato che le ore del verificatore (120) potrebbero non essere bilanciate.

Dopo il secondo tentativo, la candidatura è stata approvata e si è potuto procedere con la fase RTB. 
\subsection{Verso la RTB}
La fase RTB (\emph{Requirements and Technology Baseline}) comprende diversi obiettivi da soddisfare, di seguito elencati:
\begin{itemize}
    \item definire i requisiti, in accordo con il proponente, nel documento \textbf{Analisi dei Requisiti};
    \item dimostrare l'adeguatezza, la compatibilità e la fattibilità delle tecnologie, framework e librerie scelte tramite il \textbf{PoC}.
\end{itemize}
A supporto di tali attività vi sono altri obblighi documentali:
\begin{itemize}
    \item il presente \textbf{Piano di Progetto};
    \item il \textbf{Piano di Qualifica}:\\
    documento che identifica le metriche e le strategie necessarie per garantire qualità al prodotto finale;
    \item il \textbf{Glossario}:\\
    documento che fornisce definizioni chiare e precise di termini ambigui o ritenuti necessari per comprendere un determinato contesto;
    \item \textbf{Norme di Progetto}:\\
    documento che definisce con precisione le regole che il gruppo dovrà rispettare durante l'arco di svolgimento del progetto;
    \item \textbf{Verbali} interni ed esterni.
\end{itemize}
Ciascuno dei documenti elencati dovrà essere archiviato in un  repository accessibile dal committente e dal proponente. Ciascuno degli obiettivi sopra elencati ed evidenziati in grassetto corrisponde nel software di versionamento \emph{jira} ad un elemento di tipo \emph{"Epic"} utilizzato per raggruppare un insieme di task di minori dimensioni.\\
I periodi presentati in questa sezione saranno descritti tramite le fasi di \emph{sprint planning}, \emph{sprint review} e \emph{sprint retrospective}. In ogni periodo, a partire dall'introduzione di \emph{Jira}, le ore che riguardano il preventivo e il consuntivo sono riportate in formato decimale perché rappresentano in modo più' dettagliato le task assegnate.

\newpage
% Primo periodo
% Primo periodo
\subsubsection{Primo periodo (2023/11/08 - 2023/11/26)}

\subsubsubsection{Planning}
\subsubsubsubsection*{Attività pianificate}
All'inizio del periodo ad ogni membro del gruppo sono stati assegnati ruoli specifici, di seguito riportati:
\begin{table}[H]
\centering
\begin{tabular}{|c|c|c|}
\hline
\textbf{Membro} & \textbf{Ruolo} \\
\hline
Samuele V. & Analista \\
\hline
Michele Z. & Verificatore \\
\hline
Leonardo B. & Amministratore \\
\hline
Riccardo Z. & Analista \\
\hline
Filippo T. & Responsabile \\
\hline
Davide B. & Progettista \\
\hline
\end{tabular}
\caption{Ruoli assunti per ciascun membro del team all'inizio del periodo}
\end{table}
Gli obiettivi posti per lo $\textit{sprint}_G$ sono stati i seguenti:
\begin{itemize}
    \item Approfondire in collaborazione con il proponente le $\textit{tecnologie}_G$ da utilizzare e i requisiti del progetto;
    \item Iniziare il documento di \emph{Analisi dei Requisiti};
    \item Ideare un $\textit{sistema}_G$ di $\textit{versionamento}_G$ regolamentato per i documenti;
    \item Continuare \emph{Norme di Progetto}.
\end{itemize}

\subsubsubsubsection*{Preventivo}
\begin{table}[H]
    \centering
\begin{spreadtab}{{tabular}{|c|c|c|c|c|c|c|c|}}
    \hline
    @\textbf{Membro} & @\textbf{Re} & @\textbf{Amm} & @\textbf{An} & @\textbf{Progr} & @\textbf{Proge} & @\textbf{Ve} & @\textbf{Totale} \\
    \hline
    @ Samuele V.   & 0          & 0          & 3         & 0          & 0     & 0     & sum(b2:g2) \\
    @ Leonardo B.  & 0         & 1          & 0        & 0        & 0     & 0    & sum(b3:g3) \\
    @ Riccardo Z.  & 0          & 0          & 3          & 0          & 0     & 0   & sum(b4:g4) \\
    @ Davide B.    & 0          & 0          & 0       & 0       & 0     & 3     & sum(b5:g5) \\
    @ Michele Z.   & 0          & 0          & 0         & 0          & 0     & 4     & sum(b6:g6) \\
    @ Filippo T.   & 2          & 0          & 0         & 0          & 0     & 0     & sum(b7:g7) \\
    \hline
    @\textbf{Ore totali} & sum(b2:b7) & sum(c2:c7) & sum(d2:d7) & sum(e2:e7) & sum(f2:f7) & sum(g2:g7) &  sum(b8:g8)\\
    \hline
    @\textbf{Costo totale} & 30*b8 & 20*c8 & 25*d8 & 15*e8 & 25*f8 & 15*g8 & sum(b9:g9)\\
    \hline
\end{spreadtab}
    \caption{Preventivo orario ed economico parziale per il primo periodo, in base al ruolo}
    \label{tab:prev_rtb}
    \vspace{5mm}
    \textbf{Legenda:} \textit{Re} = Responsabile, \textit{Amm} = Amministratore, \textit{An} = Analista, \textit{Progr} = Programmatore, \textit{Proge} = Progettista, \textit{Ve} = Verificatore
\end{table}

\begin{figure}[H]
  \centering
  \includegraphics[width=0.6\linewidth]{grafici/1_periodo_torta.png}
  \caption{Ripartizione dei costi per ruolo nel $1^\circ$ periodo}
        \vspace{10mm}
  \includegraphics[width=0.7\linewidth]{grafici/1_periodo_istogramma.png}
  \caption{Ore preventivate per ciascuna persona nel $1^\circ$ periodo}
\end{figure}

\subsubsubsection{Review}
\subsubsubsubsection*{Attività svolte}
Le attività preventivate sono state svolte con successo e sono state le seguenti:
\begin{itemize}
    \item E' stato deciso un un workflow per l'utilizzo dei $\textit{repository}_G$ di GitHub ovvero \emph{GitFlow};
    \item E' stato elaborato un $\textit{sistema}_G$ regolamentato per il $\textit{versionamento}_G$ dei documenti;
    \item E' stato modificato il documento di \emph{"Dichiarazione degli impegni"} secondo le indicazioni del prof. Vardanega;
    \item E' continuata la stesura del documento \emph{Norme di Progetto};
    \item Si è deciso di utilizzare \emph{Overleaf} per la stesura dei documenti e solo dopo la verifica il documento verrà caricato nel $\textit{repository}_G$ \emph{Project14} tramite file $\textit{pdf}_G$;
    \item E' stata ideata una prima bozza del documento di \emph{Analisi dei Requisiti};
    \item E' stato effettuato un incontro con il proponente durante il quale:
    \begin{itemize}
        \item Si è individuato \emph{React} come probabile libreria da utilizzare;
        \item Sono state fornite risorse e link utili per approfondire gli strumenti il gruppo andrà ad utilizzare;
        \item È stata posta enfasi sull'uso di \emph{Docker} per il $\textit{deployment}_G$;
        \item Si è discusso dei casi d'uso;
        \item Sono stati risolti i dubbi riguardanti il \emph{PoC} e in particolare riguardo alle $\textit{feature}_G$ da includere.
    \end{itemize}
\end{itemize}
\subsubsubsubsection*{Consuntivo}
\begin{table}[H]
    \centering
\begin{spreadtab}{{tabular}{|c|c|c|c|c|c|c|c|}}
    \hline
    @\textbf{Membro} & @\textbf{Re} & @\textbf{Amm} & @\textbf{An} & @\textbf{Progr} & @\textbf{Proge} & @\textbf{Ve} & @\textbf{Totale} \\
    \hline
    @ Samuele V.   & 0          & 0          & 3         & 0          & 0     & 0     & sum(b2:g2) \\
    @ Leonardo B.  & 0         & 1          & 0        & 0        & 0     & 0    & sum(b3:g3) \\
    @ Riccardo Z.  & 0          & 0          & 3          & 0          & 0     & 0   & sum(b4:g4) \\
    @ Davide B.    & 0          & 0          & 0       & 0       & 0     & 3     & sum(b5:g5) \\
    @ Michele Z.   & 0          & 0          & 0         & 0          & 0     & 4     & sum(b6:g6) \\
    @ Filippo T.   & 2          & 0          & 0         & 0          & 0     & 0     & sum(b7:g7) \\
    \hline
    @\textbf{Ore totali} & sum(b2:b7) & sum(c2:c7) & sum(d2:d7) & sum(e2:e7) & sum(f2:f7) & sum(g2:g7) &  sum(b8:g8)\\
    \hline
    @\textbf{Costo totale} & 30*b8 & 20*c8 & 25*d8 & 15*e8 & 25*f8 & 15*g8 & sum(b9:g9)\\
    \hline
\end{spreadtab}
    \caption{Consuntivo orario ed economico parziale per il primo periodo, in base al ruolo}
    \label{tab:prev_rtb}
    \vspace{5mm}
    \textbf{Legenda:} \textit{Re} = Responsabile, \textit{Amm} = Amministratore, \textit{An} = Analista, \textit{Progr} = Programmatore, \textit{Proge} = Progettista, \textit{Ve} = Verificatore
\end{table}
\subsubsubsection{Retrospective}
I $\textit{rischi}_G$ verificati in questa fase sono stati: \nameref{ro:1}, \nameref{ro:4}.
In questo primo periodo di assestamento ci sono stati diversi dubbi sulla stesura del documento \emph{Piano di Qualifica}, non ancora iniziato. Inoltre, rispetto al documento \emph{Analisi dei Requisiti} non si sono definite delle linee guida utili per il suo sviluppo causando rallentamenti successivi. Inoltre non è risultata chiara la distinzione ore individuali/produttive e, di conseguenza, il loro tracciamento.
%\subsubsubsubsection*{Rischi verificati}

%\subsubsubsubsection*{Analisi retrospettiva}
\newpage
% Secondo periodo
% Secondo periodo
\subsubsection{Secondo periodo (27/11/2023 - 12/12/2023)}

\subsubsubsection{Planning}
\subsubsubsubsection*{Attività pianificate}
All'inizio del periodo ad ogni membro del gruppo sono stati assegnati ruoli specifici, di seguito riportati:
\begin{table}[H]
\centering
\begin{tabular}{|c|c|c|}
\hline
\textbf{Membro} & \textbf{Ruolo} \\
\hline
Samuele V. & Responsabile \\
\hline
Michele Z. & Analista \\
\hline
Leonardo B. & Analista \\
\hline
Riccardo Z. & Programmatore \\
\hline
Filippo T. & Verificatore \\
\hline
Davide B. & Amministratore \\
\hline
\end{tabular}
\caption{Ruoli assunti per ciascun membro del team all'inizio del periodo}
\end{table}
Si è deciso di non coinvolgere il ruolo di "Progettista"  e, invece, di introdurre il ruolo di "Programmatore" come figura che, almeno nelle prime fasi di vita del progetto, si occuperà della stesura dei verbali e altri documenti.

Gli obiettivi posti per lo $\textit{sprint}_G$ sono stati i seguenti:
\begin{itemize}
    \item Approfondire con il proponente le $\textit{tecnologie}_G$ da utilizzare;
    \item Cominciare la stesura del documento \emph{Piano di Progetto};
    \item Continuare la stesura del documento \emph{Analisi dei Requisiti};
    \item Approfondire i linguaggi e librerie emersi dall'ultimo incontro con il proponente:
    \begin{itemize}
        \item \emph{Javascript};
        \item \emph{React};
        \item \emph{Docker}.
    \end{itemize}
    \item Approfondire con il proponente alcune questioni emerse sul documento di \emph{Analisi dei Requisiti};
\end{itemize}

\subsubsubsubsection*{Preventivo}
\begin{table}[H]
    \centering
\begin{spreadtab}{{tabular}{|c|c|c|c|c|c|c|c|}}
    \hline
    @\textbf{Membro} & @\textbf{Re} & @\textbf{Amm} & @\textbf{An} & @\textbf{Progr} & @\textbf{Proge} & @\textbf{Ve} & @\textbf{Totale} \\
    \hline
    @ Samuele V.   & 4          & 0          & 0         & 0          & 0     & 1     & sum(b2:g2) \\
    @ Leonardo B.  & 0         & 0          & 3        & 0        & 0     & 0    & sum(b3:g3) \\
    @ Riccardo Z.  & 0          & 0          & 0          & 4          & 0     & 0   & sum(b4:g4) \\
    @ Davide B.    & 0          & 4          & 0       & 0       & 0     & 0     & sum(b5:g5) \\
    @ Michele Z.   & 0          & 0          & 3         & 0          & 0     & 0     & sum(b6:g6) \\
    @ Filippo T.   & 0          & 0          & 0         & 0          & 0     & 4     & sum(b7:g7) \\
    \hline
    @\textbf{Ore totali} & sum(b2:b7) & sum(c2:c7) & sum(d2:d7) & sum(e2:e7) & sum(f2:f7) & sum(g2:g7) &  sum(b8:g8)\\
    \hline
    @\textbf{Costo totale} & 30*b8 & 20*c8 & 25*d8 & 15*e8 & 25*f8 & 15*g8 & sum(b9:g9)\\
    \hline
\end{spreadtab}
    \caption{Preventivo orario ed economico parziale per il secondo periodo, in base al ruolo}
    \label{tab:prev_rtb}
    \vspace{5mm}
    \textbf{Legenda:} \textit{Re} = Responsabile, \textit{Amm} = Amministratore, \textit{An} = Analista, \textit{Progr} = Programmatore, \textit{Proge} = Progettista, \textit{Ve} = Verificatore
\end{table}


\begin{figure}[H]
  \centering
  \includegraphics[width=0.6\linewidth]{grafici/2_periodo_torta.png}
  \caption{Ripartizione dei costi per ruolo nel $2^\circ$ periodo}
        \vspace{10mm}
  \includegraphics[width=0.7\linewidth]{grafici/2_periodo_istogramma.png}
  \caption{Ore preventivate per ciascuna persona nel $2^\circ$ periodo}
\end{figure}



\subsubsubsection{Review}
\subsubsubsubsection*{Attività svolte}
Le attività preventivate sono state svolte con successo e sono state le seguenti:
\begin{itemize}
    \item E' stato continuato il documento \emph{Analisi Dei Requisiti};
    \item E' stato continuato il documento \emph{Norme di Progetto};
    \item E' stato effettuato dello studio individuale di \emph{Javascript} e \emph{React};
    \item E' stato effettuato un incontro con il proponente durante il quale:
    \begin{itemize}
        \item Sono stati chiariti dubbi sugli scenari secondari di alcuni casi d'uso;
        \item E' stato proposto con successo un timer entro lo scadere del quale è possibile fare le modifiche all'$\textit{ordinazione}_G$ collaborativa;
        \item E' stato consigliato l'utilizzo dei $\textit{framework}_G$ \emph{Bootstrap} e di \emph{NextJS}.
    \end{itemize}
\end{itemize}
\subsubsubsubsection*{Consuntivo}
\begin{table}[H]
    \centering
\begin{spreadtab}{{tabular}{|c|c|c|c|c|c|c|c|}}
    \hline
    @\textbf{Membro} & @\textbf{Re} & @\textbf{Amm} & @\textbf{An} & @\textbf{Progr} & @\textbf{Proge} & @\textbf{Ve} & @\textbf{Totale} \\
    \hline
    @ Samuele V.   & 3          & 0          & 0         & 0          & 0     & 1     & sum(b2:g2) \\
    @ Leonardo B.  & 0         & 0          & 4        & 0        & 0     & 0    & sum(b3:g3) \\
    @ Riccardo Z.  & 0          & 0          & 0          & 3          & 0     & 0   & sum(b4:g4) \\
    @ Davide B.    & 0          & 3          & 0       & 0       & 0     & 0     & sum(b5:g5) \\
    @ Michele Z.   & 0          & 0          & 3         & 0          & 0     & 0     & sum(b6:g6) \\
    @ Filippo T.   & 0          & 0          & 0         & 0          & 0     & 3     & sum(b7:g7) \\
    \hline
    @\textbf{Ore totali} & sum(b2:b7) & sum(c2:c7) & sum(d2:d7) & sum(e2:e7) & sum(f2:f7) & sum(g2:g7) &  sum(b8:g8)\\
    \hline
    @\textbf{Costo totale} & 30*b8 & 20*c8 & 25*d8 & 15*e8 & 25*f8 & 15*g8 & sum(b9:g9)\\
    \hline
   % @\textbf{Diff. preventivo} & 0 & 0 & 0 & 0 & 0 & 0 & sum(b10:g10)\\
   % \hline
\end{spreadtab}
    \caption{Consuntivo orario ed economico parziale per il secondo periodo, in base al ruolo}
    \label{tab:prev_rtb}
    \vspace{5mm}
    \textbf{Legenda:} \textit{Re} = Responsabile, \textit{Amm} = Amministratore, \textit{An} = Analista, \textit{Progr} = Programmatore, \textit{Proge} = Progettista, \textit{Ve} = Verificatore
\end{table}
\subsubsubsection{Retrospective}

I $\textit{rischi}_G$ verificati in questa fase sono stati:\nameref{ro:1},\nameref{ro:4}.
Dal punto di vista organizzativo ci sono stati diversi problemi. Inoltre sono emersi diversi dubbi riguardanti la stesura del documento di \emph{Analisi dei Requisiti}.
In particolare, sono sorti durante l'uso degli USE CASE. Infine ci sono stati dei dubbi riguardo la verifica dello stato di avanzamento dei lavori e su come individuare in quali ambiti si stanno riscontrando criticità.
In particolare sono emerse le seguenti criticità:
\begin{itemize}
    \item Lo scarso utilizzo fino al quel momento di scenari secondari;
    \item Dubbi sulla gestione di eventuali scenari che potrebbero riguardare casi d’uso principali o sotto casi;
    \item Dubbi riguardanti casi d’uso che coinvolgono più attori contemporaneamente;
    \item Dubbi riguardanti l'utilizzo della locuzione \emph{extend}.
\end{itemize}
Per mitigare tali problemi si è ideato un $\textit{documento interno}_G$ che riguarda i pattern da seguire per stilare il documento di Analisi dei Requisiti.
\newpage
% Terzo periodo
\subsubsection{Terzo periodo (13/12/2023 - 6/1/2024)}
\subsubsubsection{Planning}
\subsubsubsubsection*{Attività pianificate}
All'inizio del periodo ad ogni membro del gruppo sono stati assegnati ruoli specifici, di seguito riportati:
\begin{table}[H]
\centering
\begin{tabular}{|c|c|c|}
\hline
\textbf{Membro} & \textbf{Ruolo} \\
\hline
Samuele V. & Programmatore \\
\hline
Michele Z. & Analista \\
\hline
Leonardo B. & Programmatore \\
\hline
Riccardo Z. & Analista \\
\hline
Filippo T. & Responsabile \\
\hline
Davide B. & Verificatore \\
\hline
\end{tabular}
\caption{Ruoli assunti per ciascun membro del team all'inizio del periodo}
\end{table}


Gli obiettivi posti per lo sprint sono stati i seguenti:
\begin{itemize}
    \item Ultimare lo studio dello tecnologie, in particolare approfondire \emph{NextJS};
    \item Continuare la stesura del documento \emph{Analisi dei Requisiti} (in particolare dei casi d'uso);
    \item Iniziare a sviluppare il \emph{PoC} utilizzando le tecnologie scelte;
    \item Iniziare la stesura di \emph{Piano di Progetto};
    \item Creare una prima bozza del glossario tecnico;
    \item Organizzare seminari con Imola informatica;
    \item Effettuare un incontro con il prof. Cardin.
\end{itemize} 

\subsubsubsubsection*{Preventivo}
\begin{table}[H]
    \centering
\begin{spreadtab}{{tabular}{|c|c|c|c|c|c|c|c|}}
    \hline
    @\textbf{Membro} & @\textbf{Re} & @\textbf{Amm} & @\textbf{An} & @\textbf{Progr} & @\textbf{Proge} & @\textbf{Ve} & @\textbf{Totale} \\
    \hline
    @ Samuele V.   & 0          & 0          & 0         & 3          & 0     & 0     & sum(b2:g2) \\
    @ Leonardo B.  & 0         & 0          & 0        & 4        & 0     & 0    & sum(b3:g3) \\
    @ Riccardo Z.  & 0          & 0          & 5          & 0          & 0     & 0   & sum(b4:g4) \\
    @ Davide B.    & 0          & 0          & 0       & 0       & 0     & 4     & sum(b5:g5) \\
    @ Michele Z.   & 0          & 0          & 4.5         & 0          & 0     & 0     & sum(b6:g6) \\
    @ Filippo T.   & 4          & 0          & 0         & 0          & 0     & 0     & sum(b7:g7) \\
    \hline
    @\textbf{Ore totali} & sum(b2:b7) & sum(c2:c7) & sum(d2:d7) & sum(e2:e7) & sum(f2:f7) & sum(g2:g7) &  sum(b8:g8)\\
    \hline
    @\textbf{Costo totale} & 30*b8 & 20*c8 & 25*d8 & 15*e8 & 25*f8 & 15*g8 & sum(b9:g9)\\
    \hline
\end{spreadtab}
    \caption{Preventivo orario ed economico parziale per il terzo periodo, in base al ruolo}
    \label{tab:prev_rtb}
    \vspace{5mm}
    \textbf{Legenda:} \textit{Re} = Responsabile, \textit{Amm} = Amministratore, \textit{An} = Analista, \textit{Progr} = Programmatore, \textit{Proge} = Progettista, \textit{Ve} = Verificatore
\end{table}

\begin{figure}[H]
  \centering
  \includegraphics[width=0.6\linewidth]{grafici/3_periodo_torta.png}
  \caption{Ripartizione dei costi per ruolo nel $3^\circ$ periodo}
        \vspace{10mm}
  \includegraphics[width=0.7\linewidth]{grafici/3_periodo_istogramma.png}
  \caption{Ore preventivate per ciascuna persona nel $3^\circ$ periodo}
\end{figure}

\subsubsubsection{Review}
\subsubsubsubsection*{Attività svolte}
Le attività preventivate sono state svolte con successo e sono state le seguenti:
\begin{itemize}
    \item E' continuata la stesura del documento di \emph{Analisi dei Requisiti} e \emph{Piano di Progetto};
    \item E' continuato lo studio individuale delle tecnologie da utilizzare;
    \item E' stato effettuato con il prof. Cardin un incontro per risolvere i dubbi riguardo il documento \emph{Analisi dei Requisiti}, in particolare:
    \begin{itemize}
        \item E' stata chiarita la profondità di dettaglio per specificare ogni caso d'uso;
        \item E' stato fornito un feedback positivo per quanto riguarda i sotto casi d'uso di \emph{Modifica del menu' di un ristorante};
        \item E' stato chiarito l'utilizzo degli scenari alternativi;
        \item E' stato chiarito che non fosse corretto l'utilizzo di \emph{Database} come attore secondario.
    \end{itemize}
    \item "Big Bang" dal punto di vista organizzativo, il quale è andato ad impattare sia l'\emph{Analisi de Requisiti} che il \emph{PoC};
\end{itemize}
\subsubsubsubsection*{Consuntivo}
\begin{table}[H]
    \centering
\begin{spreadtab}{{tabular}{|c|c|c|c|c|c|c|c|}}
    \hline
    @\textbf{Membro} & @\textbf{Re} & @\textbf{Amm} & @\textbf{An} & @\textbf{Progr} & @\textbf{Proge} & @\textbf{Ve} & @\textbf{Totale} \\
    \hline
    @ Samuele V.   & 0          & 0          & 0         & 2          & 0     & 0     & sum(b2:g2) \\
    @ Leonardo B.  & 0         & 0          & 0        & 3        & 0     & 0    & sum(b3:g3) \\
    @ Riccardo Z.  & 0          & 0          & 5          & 0          & 0     & 0   & sum(b4:g4) \\
    @ Davide B.    & 0          & 0          & 0       & 0       & 0     & 3     & sum(b5:g5) \\
    @ Michele Z.   & 0          & 0          & 5         & 0          & 0     & 0     & sum(b6:g6) \\
    @ Filippo T.   & 6          & 0          & 0         & 0          & 0     & 0     & sum(b7:g7) \\
    \hline
    @\textbf{Ore totali} & sum(b2:b7) & sum(c2:c7) & sum(d2:d7) & sum(e2:e7) & sum(f2:f7) & sum(g2:g7) &  sum(b8:g8)\\
    \hline
    @\textbf{Costo totale} & 30*b8 & 20*c8 & 25*d8 & 15*e8 & 25*f8 & 15*g8 & sum(b9:g9)\\
    \hline
  %  @\textbf{Diff. preventivo} & 0 & 0 & 0 & 0 & 0 & 0 & sum(b10:g10)\\
  %  \hline
\end{spreadtab}
    \caption{Consuntivo orario ed economico parziale per il terzo periodo, in base al ruolo}
    \label{tab:prev_rtb}
    \vspace{5mm}
    \textbf{Legenda:} \textit{Re} = Responsabile, \textit{Amm} = Amministratore, \textit{An} = Analista, \textit{Progr} = Programmatore, \textit{Proge} = Progettista, \textit{Ve} = Verificatore
\end{table}
\subsubsubsection{Retrospective}
I rischi verificati in questa fase sono stati: \nameref{rt:1}, \nameref{ro:4}.
Quasi del tutto risolti i problemi dal punto di vista organizzativo, e inoltre il gruppo si sta mettendo in moto per cercare di integrare conoscenze sulle tecnologie da usare in futuro nel progetto.
 \\
I dubbi principali sono sorti durante l'uso degli USE CASE per stilare il documento di analisi dei requisiti. Inoltre vi sono state difficoltà nell'iniziare lo sviluppo del del \emph{PoC} in quanto non tutti i membri del gruppo sono familiari con l'uso di framework per lo sviluppo.
Oltre a ciò sono state rilevate diverse difficoltà organizzative, in particolare:
\begin{itemize}
    \item Come gestire in modo ottimale il tempo;
    \item Assenza di obiettivi ben definiti per ogni membro del gruppo.
\end{itemize}
Per mitigare tali problemi attualmente si sta cercando un supporto per rendere quanto più preciso e chiaro possibile il conteggio delle ore.
\newpage
% Quarto periodo
\subsubsection{Quarto periodo (2024/01/07 - 2024/01/15)}
\subsubsubsection{Planning}
\subsubsubsubsection*{Attività pianificate}
All'inizio del periodo ad ogni membro del gruppo sono stati assegnati ruoli specifici, di seguito riportati:
\begin{table}[H]
\centering
\begin{tabular}{|c|c|c|}
\hline
\textbf{Membro} & \textbf{Ruolo} \\
\hline
Samuele V. & Analista \\
\hline
Michele Z. & Amministratore \\
\hline
Leonardo B. & Responsabile \\
\hline
Riccardo Z. & Programmatore \\
\hline
Filippo T. & Verificatore \\
\hline
Davide B. & Analista \\
\hline
\end{tabular}
\caption{Ruoli assunti per ciascun membro del team all'inizio del periodo}
\end{table}


Gli obiettivi posti per lo $\textit{sprint}_G$ sono stati i seguenti:
\begin{itemize}
    \item Delineare una prima versione del \emph{Piano di Progetto};
    \item Continuare ad aggiornare il documento \emph{Norme di progetto};
    \item Iniziare a sviluppare il \emph{PoC}; 
    \item Continuare la stesura di \emph{Analisi dei Requisiti};
    \item Normare in modo più dettagliato i $\textit{processi organizzativi}_G$;
    \item Richiedere un incontro con il proponente per discutere dei requisiti funzionali trovati mediante i casi d'uso;
    \item Iniziare la stesura del glossario tecnico.
\end{itemize}
\subsubsubsubsection*{Preventivo}
\begin{table}[H]
    \centering
\begin{spreadtab}{{tabular}{|c|c|c|c|c|c|c|c|}}
    \hline
    @\textbf{Membro} & @\textbf{Re} & @\textbf{Amm} & @\textbf{An} & @\textbf{Progr} & @\textbf{Proge} & @\textbf{Ve} & @\textbf{Totale} \\
    \hline
    @ Samuele V.   & 0          & 0          & 4         & 0          & 0     & 0     & sum(b2:g2) \\
    @ Leonardo B.  & 2         & 0          & 0        & 0        & 0     & 2    & sum(b3:g3) \\
    @ Riccardo Z.  & 0          & 0          & 0          & 4          & 0     & 0   & sum(b4:g4) \\
    @ Davide B.    & 0          & 0          & 5       & 0       & 0     & 0     & sum(b5:g5) \\
    @ Michele Z.   & 0          & 4          & 0         & 0          & 0     & 0     & sum(b6:g6) \\
    @ Filippo T.   & 0          & 0          & 0         & 0          & 0     & 3     & sum(b7:g7) \\
    \hline
    @\textbf{Ore totali} & sum(b2:b7) & sum(c2:c7) & sum(d2:d7) & sum(e2:e7) & sum(f2:f7) & sum(g2:g7) &  sum(b8:g8)\\
    \hline
    @\textbf{Costo totale} & 30*b8 & 20*c8 & 25*d8 & 15*e8 & 25*f8 & 15*g8 & sum(b9:g9)\\
    \hline
\end{spreadtab}
    \caption{Preventivo orario ed economico parziale per il quarto periodo, in base al ruolo}
    \label{tab:prev_rtb}
    \vspace{5mm}
    \textbf{Legenda:} \textit{Re} = Responsabile, \textit{Amm} = Amministratore, \textit{An} = Analista, \textit{Progr} = Programmatore, \textit{Proge} = Progettista, \textit{Ve} = Verificatore
\end{table}

\begin{figure}[H]
  \centering
  \includegraphics[width=0.6\linewidth]{grafici/4_periodo_torta.png}
  \caption{Ripartizione dei costi per ruolo nel $4^\circ$ periodo}
        \vspace{10mm}
  \includegraphics[width=0.7\linewidth]{grafici/4_periodo_istogramma.png}
  \caption{Ore preventivate per ciascuna persona nel $4^\circ$ periodo}
\end{figure}


\subsubsubsection{Review}
\subsubsubsubsection*{Attività svolte}
Periodo di assestamento causa sessione di esami.
Sono state prese le seguenti decisioni:
\begin{itemize}
    \item Utilizzo del $\textit{software}_G$ \emph{Jira} per assegnare le issues e tenere conto del tracciamento temporale di ognuna di esse;
    \item Utilizzo del modello di sviluppo \emph{Scrum};
    \item Utilizzo di linee guida per il documento di \emph{Analisi dei Requisiti} in merito allo stile da adottare nell'uso dei diagrammi e nella specifica testuale dei casi d'uso.
\end{itemize}
\subsubsubsubsection*{Consuntivo}
\begin{table}[H]
    \centering
\begin{spreadtab}{{tabular}{|c|c|c|c|c|c|c|c|}}
    \hline
    @\textbf{Membro} & @\textbf{Re} & @\textbf{Amm} & @\textbf{An} & @\textbf{Progr} & @\textbf{Proge} & @\textbf{Ve} & @\textbf{Totale} \\
    \hline
    @ Samuele V.   & 0          & 0          & 3         & 0          & 0     & 0     & sum(b2:g2) \\
    @ Leonardo B.  & 1.25         & 0          & 1.5        & 0        & 0     & 2    & sum(b3:g3) \\
    @ Riccardo Z.  & 0          & 0          & 0          & 3.5          & 0     & 0   & sum(b4:g4) \\
    @ Davide B.    & 0          & 0          & 3.5       & 0       & 0     & 0     & sum(b5:g5) \\
    @ Michele Z.   & 0          & 3        & 0         & 0          & 0     & 0     & sum(b6:g6) \\
    @ Filippo T.   & 0          & 0          & 0         & 0          & 0     & 3     & sum(b7:g7) \\
    \hline
    @\textbf{Ore totali} & sum(b2:b7) & sum(c2:c7) & sum(d2:d7) & sum(e2:e7) & sum(f2:f7) & sum(g2:g7) &  sum(b8:g8)\\
    \hline
    @\textbf{Costo totale} & 30*b8 & 20*c8 & 25*d8 & 15*e8 & 25*f8 & 15*g8 & sum(b9:g9)\\
    \hline
    %@\textbf{Diff. preventivo} & 0 & -1 & -7 & 0 & 0 & 0 & sum(b10:g10)\\
    %\hline
\end{spreadtab}
    \caption{Preventivo orario ed economico parziale per il quarto periodo, in base al ruolo}
    \label{tab:prev_rtb}
    \vspace{5mm}
    \textbf{Legenda:} \textit{Re} = Responsabile, \textit{Amm} = Amministratore, \textit{An} = Analista, \textit{Progr} = Programmatore, \textit{Proge} = Progettista, \textit{Ve} = Verificatore
\end{table}
\subsubsubsection{Retrospective}
\subsubsubsubsection*{Rischi verificati}
I $\textit{rischi}_G$ riscontrati in questa fase sono stati: \nameref{ro:3},\nameref{ro:4}
Ci sono stati rallentamenti nell'avanzamento dei lavori e difficoltà nel coordinamento delle attività causa l'imminente sessione d'esame. Sono stati risolti, almeno parzialmente, i problemi di tracciamento delle ore tramite l'uso dello $\textit{strumento}_G$ \emph{Jira}.
\newpage
% Quinto periodo
\subsubsection{Quinto periodo (16/1/2024 - 12/2/2024)}
\subsubsubsection{Planning}
\subsubsubsubsection*{Attività pianificate}
Nel corso dello sprint, il team \emph{RAMtastic6} ha previsto un periodo di sospensione parziale delle attività di circa due settimane, dal 22/1/2024 fino al 5/2/2024.
Gli obiettivi posti per lo sprint sono stati i seguenti:
\begin{itemize}
    \item Implementare una prima versione di ricerca e prenotazione del ristorante all'interno del Poc;
    \item Completare il documento \emph{Analisi dei Requisiti} aggiungendo i requisiti funzionali.
\end{itemize}

\subsubsubsubsection*{Preventivo}
\begin{table}[H]
    \centering
\begin{spreadtab}{{tabular}{|c|c|c|c|c|c|c|c|}}
    \hline
    @\textbf{Membro} & @\textbf{Re} & @\textbf{Amm} & @\textbf{An} & @\textbf{Progr} & @\textbf{Proge} & @\textbf{Ve} & @\textbf{Totale} \\
    \hline
    @ Samuele V.   & 0          & 0          & 3         & 0          & 0     & 0     & sum(b2:g2) \\
    @ Leonardo B.  & 4         & 0          & 0        & 0        & 0     & 0.5   & sum(b3:g3) \\
    @ Riccardo Z.  & 0          & 0          & 0          & 5.5          & 0     & 0   & sum(b4:g4) \\
    @ Davide B.    & 0          & 0          & 4       & 0       & 0     & 0     & sum(b5:g5) \\
    @ Michele Z.   & 0          & 2          & 0         & 0          & 0     & 0     & sum(b6:g6) \\
    @ Filippo T.   & 0          & 0          & 0         & 0          & 0     & 4     & sum(b7:g7) \\
    \hline
    @\textbf{Ore totali} & sum(b2:b7) & sum(c2:c7) & sum(d2:d7) & sum(e2:e7) & sum(f2:f7) & sum(g2:g7) &  sum(b8:g8)\\
    \hline
    @\textbf{Costo totale} & 30*b8 & 20*c8 & 25*d8 & 15*e8 & 25*f8 & 15*g8 & sum(b9:g9)\\
    \hline
\end{spreadtab}
    \caption{Preventivo orario ed economico parziale per il quinto periodo, in base al ruolo}
    \label{tab:prev_rtb}
    \vspace{5mm}
    \textbf{Legenda:} \textit{Re} = Responsabile, \textit{Amm} = Amministratore, \textit{An} = Analista, \textit{Progr} = Programmatore, \textit{Proge} = Progettista, \textit{Ve} = Verificatore
\end{table}
\begin{figure}[H]
  \centering
  \includegraphics[width=0.6\linewidth]{grafici/5_periodo_torta.png}
  \caption{Ripartizione dei costi per ruolo nel $5^\circ$ periodo}
        \vspace{10mm}
  \includegraphics[width=0.7\linewidth]{grafici/5_periodo_istogramma.png}
  \caption{Ore preventivate per ciascuna persona nel $5^\circ$ periodo}
\end{figure}
\subsubsubsection{Review}
\subsubsubsubsection*{Attività svolte}
Per quanto riguarda l'Analisi dei Requisiti:
\begin{itemize}
    \item Sono state apportate correzioni a degli UC esistenti;
    \item Si è creata la sezione requisiti funzionali nel documento \emph{Analisi dei Requisiti}.
\end{itemize} 
\subsubsubsubsection*{Consuntivo}
\begin{table}[H]
    \centering
\begin{spreadtab}{{tabular}{|c|c|c|c|c|c|c|c|}}
    \hline
    @\textbf{Membro} & @\textbf{Re} & @\textbf{Amm} & @\textbf{An} & @\textbf{Progr} & @\textbf{Proge} & @\textbf{Ve} & @\textbf{Totale} \\
    \hline
    @ Samuele V.   & 0          & 0          & 3.75         & 0          & 0     & 0     & sum(b2:g2) \\
    @ Leonardo B.  & 3.25         & 0          & 0        & 0        & 0     & 0.75    & sum(b3:g3) \\
    @ Riccardo Z.  & 0          & 0          & 0          & 5.75          & 0     & 0   & sum(b4:g4) \\
    @ Davide B.    & 0          & 0          & 4.5       & 0       & 0     & 0     & sum(b5:g5) \\
    @ Michele Z.   & 0          & 1.17          & 0         & 0          & 0     & 0     & sum(b6:g6) \\
    @ Filippo T.   & 0          & 0          & 0         & 0          & 0     & 2.42     & sum(b7:g7) \\
    \hline
    @\textbf{Ore totali} & sum(b2:b7) & sum(c2:c7) & sum(d2:d7) & sum(e2:e7) & sum(f2:f7) & sum(g2:g7) &  sum(b8:g8)\\
    \hline
    @\textbf{Costo totale} & 30*b8 & 20*c8 & 25*d8 & 15*e8 & 25*f8 & 15*g8 & sum(b9:g9)\\
    \hline
    %@\textbf{Diff. preventivo} & 0 & 0 & 0 & 1 & 0 & 0 & sum(b10:g10)\\
    %\hline
\end{spreadtab}
    \caption{Consuntivo orario ed economico parziale per il quinto periodo, in base al ruolo}
    \label{tab:prev_rtb}
    \vspace{5mm}
    \textbf{Legenda:} \textit{Re} = Responsabile, \textit{Amm} = Amministratore, \textit{An} = Analista, \textit{Progr} = Programmatore, \textit{Proge} = Progettista, \textit{Ve} = Verificatore
\end{table}
\subsubsubsection{Retrospective}
In questo periodo sono state riscontrate delle problematiche nello sviluppo dei casi d'uso; in particolare è stata rivista la gerarchia degli attori e si è cercato di dare un senso atomico al loro nome. Inoltre, si è riscontrato il rischio \nameref{ro:3} in quanto diversi membri del gruppo erano impegnati con la sessione d'esame. 



\newpage
% Sesto periodo
\subsubsection{Sesto periodo (13/2/2024 - 6/3/2024)}
\subsubsubsection{Planning}
\subsubsubsubsection*{Attività pianificate}
All'inizio del periodo ad ogni membro del gruppo sono stati assegnati ruoli specifici, di seguito riportati:
\begin{table}[H]
\centering
\begin{tabular}{|c|c|c|}
\hline
\textbf{Membro} & \textbf{Ruolo} \\
\hline
Samuele V. & Analista \\
\hline
Michele Z. & Responsabile \\
\hline
Leonardo B. & Amministratore \\
\hline
Riccardo Z. & Analista \\
\hline
Filippo T. & Verificatore \\
\hline
Davide B. & Analista \\
\hline
\end{tabular}
\caption{Ruoli assunti per ciascun membro del team all'inizio del periodo}
\end{table}

Gli obiettivi posti per lo $\textit{sprint}_G$ sono stati i seguenti:
\begin{itemize}
    \item Ultimare gli ultimi \emph{UC} e il documento di \emph{Analisi dei Requisiti};
    \item Iniziare a redigere il glossario tecnico e implementare un'insieme di automazioni utili alla sua costruzione automatica;
    \item Riprendere lo sviluppo del \emph{PoC}.
\end{itemize}


\subsubsubsubsection*{Preventivo}
\begin{table}[H]
    \centering
\begin{spreadtab}{{tabular}{|c|c|c|c|c|c|c|c|}}
    \hline
    @\textbf{Membro} & @\textbf{Re} & @\textbf{Amm} & @\textbf{An} & @\textbf{Progr} & @\textbf{Proge} & @\textbf{Ve} & @\textbf{Totale} \\
    \hline
    @ Samuele V.   & 0          & 0          & 8         & 0          & 0     & 0     & sum(b2:g2) \\
    @ Leonardo B.  & 0         & 3          & 0        & 2.5        & 0     & 0    & sum(b3:g3) \\
    @ Riccardo Z.  & 0          & 0          & 8.5          & 0          & 0     & 0   & sum(b4:g4) \\
    @ Davide B.    & 0          & 0          & 8       & 1.5       & 0     & 0     & sum(b5:g5) \\
    @ Michele Z.   & 5          & 0          & 0         & 0          & 0     & 0     & sum(b6:g6) \\
    @ Filippo T.   & 0          & 0          & 0         & 0          & 0     & 4.5     & sum(b7:g7) \\
    \hline
    @\textbf{Ore totali} & sum(b2:b7) & sum(c2:c7) & sum(d2:d7) & sum(e2:e7) & sum(f2:f7) & sum(g2:g7) &  sum(b8:g8)\\
    \hline
    @\textbf{Costo totale} & 30*b8 & 20*c8 & 25*d8 & 15*e8 & 25*f8 & 15*g8 & sum(b9:g9)\\
    \hline
\end{spreadtab}
    \caption{Preventivo orario ed economico parziale per il sesto periodo, in base al ruolo}
    \label{tab:prev_rtb}
    \vspace{5mm}
    \textbf{Legenda:} \textit{Re} = Responsabile, \textit{Amm} = Amministratore, \textit{An} = Analista, \textit{Progr} = Programmatore, \textit{Proge} = Progettista, \textit{Ve} = Verificatore
\end{table}

\newpage

\begin{figure}[H]
  \centering
  \includegraphics[width=0.6\linewidth]{grafici/6_periodo_torta.png}
  \caption{Ripartizione dei costi per ruolo nel $6^\circ$ periodo}
        \vspace{10mm}
  \includegraphics[width=0.7\linewidth]{grafici/6_periodo_istogramma.png}
  \caption{Ore preventivate per ciascuna persona nel $6^\circ$ periodo}
\end{figure}

\subsubsubsection{Review}
\subsubsubsubsection*{Attività svolte}
\begin{itemize}
    \item Il gruppo ritiene di aver raggiunto un livello soddisfacente per poter presentare il documento \emph{Analisi dei Requisiti} ad Imola informatica e poter dedicarsi maggiormente allo sviluppo del \emph{PoC};
    \item E' stato effettuato un incontro con il proponente il 28/2/2024 per risolvere qualche dubbio e per fare il punto della situazione. In particolare:
    \begin{itemize}
        \item Aggiornamento sullo stato dei lavori.
        \item Fissata una ipotetica data di consegna del \emph{PoC}.
        \item $\textit{Feedback}_G$ sull'attuale versione dell'\emph{Analisi dei Requisiti} e aggiornamento  sulle $\textit{tecnologie}_G$ da usare, in particolare \emph{NestJs} per il Back-End.
        \item Supporto sulla scelta della quantità di ristoranti amministrabili da un admin.
        
    \end{itemize}
\end{itemize}
\subsubsubsubsection*{Consuntivo}
\begin{table}[H]
    \centering
\begin{spreadtab}{{tabular}{|c|c|c|c|c|c|c|c|}}
    \hline
    @\textbf{Membro} & @\textbf{Re} & @\textbf{Amm} & @\textbf{An} & @\textbf{Progr} & @\textbf{Proge} & @\textbf{Ve} & @\textbf{Totale} \\
    \hline
    @ Samuele V.   & 0          & 0          & 6.42         & 0          & 0     & 0     & sum(b2:g2) \\
    @ Leonardo B.  & 0         & 3          & 0        & 1        & 0     & 0    & sum(b3:g3) \\
    @ Riccardo Z.  & 0          & 0          & 7.92          & 0          & 0     & 0   & sum(b4:g4) \\
    @ Davide B.    & 0          & 0          & 6.33       & 1.5       & 0     & 0     & sum(b5:g5) \\
    @ Michele Z.   & 4.25          & 0          & 0         & 0          & 0     & 0     & sum(b6:g6) \\
    @ Filippo T.   & 0          & 0          & 0         & 0          & 0     & 3.63     & sum(b7:g7) \\
    \hline
    @\textbf{Ore totali} & sum(b2:b7) & sum(c2:c7) & sum(d2:d7) & sum(e2:e7) & sum(f2:f7) & sum(g2:g7) &  sum(b8:g8)\\
    \hline
    @\textbf{Costo totale} & 30*b8 & 20*c8 & 25*d8 & 15*e8 & 25*f8 & 15*g8 & sum(b9:g9)\\
    \hline
\end{spreadtab}
    \caption{Consuntivo orario ed economico parziale per il sesto periodo, in base al ruolo}
    \label{tab:prev_rtb}
    \vspace{5mm}
    \textbf{Legenda:} \textit{Re} = Responsabile, \textit{Amm} = Amministratore, \textit{An} = Analista, \textit{Progr} = Programmatore, \textit{Proge} = Progettista, \textit{Ve} = Verificatore
\end{table}
\subsubsubsection{Retrospective}
Ritardata la stesura del \emph{Glossario Tecnico} a causa delle difficoltà riscontrate nel far funzionare a dovere le automazioni create per generare il documento. \\
Rallentato lo sviluppo del \emph{PoC} dato che è stata data priorità alla finalizzazione del documento \emph{Analisi dei Requisiti}.
\newpage
% Settimo periodo
\subsubsection{Settimo periodo (2024/03/07 - 2024/03/24)}

\subsubsubsection{Planning}
%Il gruppo \emph{Ramtastic6} ha deciso di pianificare uno $\textit{sprint}_G$ breve della durata di una settimana per raggiungere risultati per obiettivi ben definiti e significativi seppur di durata per il loro raggiungimento contenuta.


\subsubsubsubsection*{Attività pianificate}
All'inizio del periodo ad ogni membro del gruppo sono stati assegnati ruoli specifici, di seguito riportati:
\begin{table}[H]
\centering
\begin{tabular}{|c|c|c|}
\hline
\textbf{Membro} & \textbf{Ruolo} \\
\hline
Samuele V. & Programmatore \\
\hline
Michele Z. & Programmatore \\
\hline
Leonardo B. & Amministratore \\
\hline
Riccardo Z. & Responsabile \\
\hline
Filippo T. & Programmatore \\
\hline
Davide B. & Analista \\
\hline
\end{tabular}
\caption{Ruoli assunti per ciascun membro del team all'inizio del periodo}
\end{table}

Gli obiettivi posti per lo $\textit{sprint}_G$ sono stati i seguenti:
\begin{itemize}
    \item Per quanto riguarda il \emph{PoC}:
    \begin{enumerate}
        \item Approfondire ulteriormente le $\textit{tecnologie}_G$ individuate per la sua realizzazione come Next js, Nest js e $\textit{Docker}_G$ e contestualmente rivedere l'organizzazione del $\textit{repository}_G$ per lo sviluppo.
        \item Raggiungere una configurazione accettabile per il \emph{PoC};
        \item Creare una prima versione del database (lato back-end);
        \item Raggiungere una prima implementazione della funzionalità di ricerca dei ristoranti (lato back-end e front-end);
        \item Chiedere $\textit{feedback}_G$ al proponente riguardo la configurazione raggiunta e consigli per quanto riguarda la $\textit{feature}_G$ di $\textit{ordinazione}_G$.
    \end{enumerate}
    \item Aggiornare \emph{Piano di Progetto} riportando gli $\textit{sprint}_G$ mancanti e i nuovi $\textit{sprint}_G$;
    \item Redigere una prima versione del documento \emph{Piano di Qualifica};
    \item Completare l'automazione e la stesura di una prima versione del glossario;
    \item Completare la pagina $\textit{github}_G$.io riguardante il sunto dei documenti nel $\textit{repository}_G$ \emph{RAMtastic6.$\textit{github}_G$.io}, completare l'automazione inerente e inserirla nello stesso $\textit{repository}_G$;
    \item Esplorare le potenzialità di $\textit{jira}_G$ per analizzare il lavoro fatto.
\end{itemize}

\subsubsubsubsection*{Preventivo}
\begin{table}[H]
    \centering
\begin{spreadtab}{{tabular}{|c|c|c|c|c|c|c|c|}}
    \hline
    @\textbf{Membro} & @\textbf{Re} & @\textbf{Amm} & @\textbf{An} & @\textbf{Progr} & @\textbf{Proge} & @\textbf{Ve} & @\textbf{Totale} \\
    \hline
    @ Samuele V.   & 0          & 0          & 0        & 6.5          & 0     & 1     & sum(b2:g2) \\
    @ Leonardo B.  & 0         & 5          & 0        & 0        & 0     & 1    & sum(b3:g3) \\
    @ Riccardo Z.  & 6          & 3          & 0          & 2          & 0     & 1   & sum(b4:g4) \\
    @ Davide B.    & 0          & 1          & 2      & 0       & 0     & 4.5     & sum(b5:g5) \\
    @ Michele Z.   & 0          & 1          & 0         & 2          & 0     & 0     & sum(b6:g6) \\
    @ Filippo T.   & 0          & 0          & 0         & 5          & 0     & 0     & sum(b7:g7) \\
    \hline
    @\textbf{Ore totali} & sum(b2:b7) & sum(c2:c7) & sum(d2:d7) & sum(e2:e7) & sum(f2:f7) & sum(g2:g7) &  sum(b8:g8)\\
    \hline
    @\textbf{Costo totale} & 30*b8 & 20*c8 & 25*d8 & 15*e8 & 25*f8 & 15*g8 & sum(b9:g9)\\
    \hline
\end{spreadtab}
    \caption{Consuntivo orario ed economico parziale per il settimo periodo, in base al ruolo}
    \label{tab:prev_rtb}
    \vspace{5mm}
    \textbf{Legenda:} \textit{Re} = Responsabile, \textit{Amm} = Amministratore, \textit{An} = Analista, \textit{Progr} = Programmatore, \textit{Proge} = Progettista, \textit{Ve} = Verificatore
\end{table}
\begin{figure}[H]
  \centering
  \includegraphics[width=0.6\linewidth]{grafici/7_periodo_torta.png}
  \caption{Ripartizione dei costi per ruolo nel $7^\circ$ periodo}
        \vspace{10mm}
  \includegraphics[width=0.7\linewidth]{grafici/7_periodo_istogramma.png}
  \caption{Ore preventivate per ciascuna persona nel $7^\circ$ periodo}
\end{figure}
\subsubsubsection{Review}
\subsubsubsubsection*{Attività svolte}
Le attività previste sono state svolte con relativo successo, in particolare:
\begin{itemize}
    \item Per quanto riguarda il \emph{PoC}:
    \begin{enumerate}
        \item I programmatori coinvolti si sono allineati sulle $\textit{tecnologie}_G$;
        \item E' stata raggiunta una configurazione accettabile per il suo sviluppo;
        \item E' stata implementata una prima versione della funzionalità di ricerca dei ristoranti con la nuova configurazione (lato back-end e front-end).
    \end{enumerate}
    \item \emph{Piano di Progetto} è stato aggiornato come previsto;
    \item \emph{Piano di Qualifica} è stato iniziato come previsto;
    \item L'automazione per generare il glossario tecnico è stata completata;
    \item Una prima versione della pagina $\textit{github}_G$.io; è stata completata l'automazione atta a generarla ed è stata inserita nel $\textit{repository}_G$ \emph{RAMtastic6.$\textit{github}_G$.io};
    \item L'organizzazione del \emph{PoC} e del $\textit{way of working}_G$ è stata rivista come riportato nel $\textit{verbale}_G$ del 13/03/2024;
    \item Le potenzialità di $\textit{jira}_G$ sono state esplorate al fine di garantire una miglior collaborazione come riportato nel $\textit{verbale}_G$ del 13/03/2024;
    \item E' stato effettuato un incontro con il proponente in data 21/03/2024. In particolare:
    \begin{itemize}
        \item E' stato fornito un $\textit{feedback}_G$ positivo riguardante l'utilizzo dei file \emph{Json} e delle \emph{API rest} come mezzo di comunicazione tra \emph{NextJs} e \emph{NestJs}
        \item E' stato consigliato l'uso dei \emph{custom hooks};
        \item Sono stati forniti dei consigli per quanto riguarda la realizzazione della $\textit{feature}_G$ dell'$\textit{ordinazione}_G$ collaborativa. In questo contesto stato introdotto brevemente il concetto di \emph{socket};
        \item Sono stati chiariti dubbi riguardanti il ruolo del \emph{project manager}.
    \end{itemize}
\end{itemize}
\subsubsubsubsection*{Consuntivo}
\begin{table}[H]
    \centering
\begin{spreadtab}{{tabular}{|c|c|c|c|c|c|c|c|}}
    \hline
    @\textbf{Membro} & @\textbf{Re} & @\textbf{Amm} & @\textbf{An} & @\textbf{Progr} & @\textbf{Proge} & @\textbf{Ve} & @\textbf{Totale} \\
    \hline
    @ Samuele V.   & 0          & 0          & 0         & 9          & 0     & 2.5     & sum(b2:g2) \\
    @ Leonardo B.  & 0         & 5          & 2        & 0        & 0     & 1    & sum(b3:g3) \\
    @ Riccardo Z.  & 6.5          & 2.92          & 0          & 0.58          & 0     & 1.5   & sum(b4:g4) \\
    @ Davide B.    & 0          & 0.67          & 3       & 0       & 0     & 4     & sum(b5:g5) \\
    @ Michele Z.   & 0          & 3          & 0         & 3          & 0     & 0     & sum(b6:g6) \\
    @ Filippo T.   & 0          & 0          & 0         & 5.5          & 0     & 0     & sum(b7:g7) \\
    \hline
    @\textbf{Ore totali} & sum(b2:b7) & sum(c2:c7) & sum(d2:d7) & sum(e2:e7) & sum(f2:f7) & sum(g2:g7) &  sum(b8:g8)\\
    \hline
    @\textbf{Costo totale} & 30*b8 & 20*c8 & 25*d8 & 15*e8 & 25*f8 & 15*g8 & sum(b9:g9)\\
    \hline
\end{spreadtab}
    \caption{Consuntivo orario ed economico parziale per il settimo periodo, in base al ruolo}
    \label{tab:prev_rtb}
    \vspace{5mm}
    \textbf{Legenda:} \textit{Re} = Responsabile, \textit{Amm} = Amministratore, \textit{An} = Analista, \textit{Progr} = Programmatore, \textit{Proge} = Progettista, \textit{Ve} = Verificatore
\end{table}
\subsubsubsection{Retrospective}
Il rischio atteso è stato principalmente: \nameref{rt:1}, in quanto per la prima volta i componenti del gruppo si sono interfacciati con il $\textit{framework}_G$ \emph{NestJs}. Tuttavia, i programmatori dopo diverse ore di studio individuale hanno dimostrato di aver compreso le $\textit{tecnologie}_G$ studiate raggiungendo una configurazione stabile per lo sviluppo del \emph{PoC}.
\newline Tra i $\textit{rischi}_G$ attesi verificati rientra la scarsa attenzione avuta in precedenza riguardo il conteggio delle ore e la difficoltà da parte del responsabile nel calcolo delle ore prima dell'introduzione dello $\textit{strumento}_G$ \emph{Jira}. Per mitigare questo $\textit{rischi}_G$ si è deciso per il periodo successivo di introdurre un secondo responsabile e di organizzare tramite file \emph{Excel} le ore in modo sistematico e preciso.
\newline
In questa fase i costi emersi dal $\textit{consuntivo}_G$ hanno ecceduto quelli del $\textit{preventivo}_G$ facendo verificare il rischio di valutazione erronea delle ore assegnate; in particolare si rileva un eccesso nei ruoli di \emph{Amministratore} e di \emph{Programmatore} in quanto:
\begin{itemize}
    \item Per l'amministratore sono emerse problematiche riguardanti il funzionamento dell'automazione del glossario e un cambio in corsa per quanto riguarda l'automazione del sunto dei documenti;
    \item Per il ruolo di programmatore essendo $\textit{tecnologie}_G$ nuove per i componenti del gruppo si è sottostimata la loro applicazione e coesione nell'ambito di un progetto complesso.
\end{itemize}
\newpage
% Ottavo periodo
\subsubsection{Ottavo periodo (25/3/2024 - 7/4/2024 )}
\subsubsubsection{Planning}
\subsubsubsubsection*{Attività pianificate}
All'inizio del periodo ad ogni membro del gruppo sono stati assegnati ruoli specifici, di seguito riportati:
\begin{table}[H]
\centering
\begin{tabular}{|c|c|c|}
\hline
\textbf{Membro} & \textbf{Ruolo} \\
\hline
Samuele V. & Programmatore \\
\hline
Michele Z. & Amministratore \\
\hline
Leonardo B. & Programmatore \\
\hline
Riccardo Z. & Responsabile \\
\hline
Filippo T. & Programmatore \\
\hline
Davide B. & Responsabile \\
\hline
\end{tabular}
\caption{Ruoli assunti per ciascun membro del team all'inizio del periodo}
\end{table}
Gli obiettivi posti per lo $\textit{sprint}_G$ sono stati i seguenti:
\begin{itemize}
    \item Per quanto riguarda il \emph{PoC}:
    \begin{itemize}
        \item Raggiungere una prima implementazione della schermata di login (lato front-end);
        \item Raggiungere una prima implementazione della funzionalità di $\textit{prenotazione}_G$ (lato back-end e front-end);
        \item Raggiungere una prima implementazione della funzionalità di $\textit{ordinazione}_G$ (lato back-end e front-end) esplorando i \emph{socket} per la comunicazione bidirezionale.
    \end{itemize}
    \item Aggiornare il documento \emph{Norme di Progetto} apportando le seguenti modifiche:
    \begin{itemize}
        \item Specificare in modo maggiormente dettagliato i ruoli;
        \item Aggiungere le nuove $\textit{repository}_G$ create e la loro organizzazione;
        \item Aggiungere tra i $\textit{processi di supporto}_G$ le attività di verifica e di gestione della qualità;
        \item Specificare l'utilizzo delle nuove automazioni aggiunte recentemente.
    \end{itemize}
    \item Redigere una prima versione del glossario;
    \item Redigere una prima versione della sezione \emph{Qualità di Prodotto} e di \emph{Test di Sistema} nel documento \emph{Piano di Qualifica};
    \item Aggiornare i periodi precedenti nel \emph{Piano di Progetto} andando a definire un modo sistematico per calcolare le ore preventivate ed effettive;
    \item Rivedere i costi e le scadenze in fase di $\textit{candidatura}_G$.
\end{itemize}
\subsubsubsubsection*{Preventivo}
\begin{table}[H]
    \centering
\begin{spreadtab}{{tabular}{|c|c|c|c|c|c|c|c|}}
    \hline
    @\textbf{Membro} & @\textbf{Re} & @\textbf{Amm} & @\textbf{An} & @\textbf{Progr} & @\textbf{Proge} & @\textbf{Ve} & @\textbf{Totale} \\
    \hline
    @ Samuele V.   & 0          & 0          & 0        & 5          & 0     & 0     & sum(b2:g2) \\
    @ Leonardo B.  & 0         & 0          & 0        & 5        & 0     & 0    & sum(b3:g3) \\
    @ Riccardo Z.  & 4.5          & 0          & 0          & 0.17          & 0     & 1.5  & sum(b4:g4) \\
    @ Davide B.    & 4          & 0         & 0      & 0       & 0     & 0     & sum(b5:g5) \\
    @ Michele Z.   & 0          & 4          & 0         & 0          & 0     & 0.17     & sum(b6:g6) \\
    @ Filippo T.   & 0          & 0          & 0         & 5          & 0     & 0     & sum(b7:g7) \\
    \hline
    @\textbf{Ore totali} & sum(b2:b7) & sum(c2:c7) & sum(d2:d7) & sum(e2:e7) & sum(f2:f7) & sum(g2:g7) &  sum(b8:g8)\\
    \hline
    @\textbf{Costo totale} & 30*b8 & 20*c8 & 25*d8 & 15*e8 & 25*f8 & 15*g8 & sum(b9:g9)\\
    \hline
\end{spreadtab}
    \caption{Preventivo orario ed economico parziale per l'ottavo periodo, in base al ruolo}
    \label{tab:prev_rtb}
    \vspace{5mm}
    \textbf{Legenda:} \textit{Re} = Responsabile, \textit{Amm} = Amministratore, \textit{An} = Analista, \textit{Progr} = Programmatore, \textit{Proge} = Progettista, \textit{Ve} = Verificatore
\end{table}

\begin{figure}[H]
  \centering
  \includegraphics[width=0.6\linewidth]{grafici/8_periodo_torta.png}
  \caption{Ripartizione dei costi per ruolo nel $8^\circ$ periodo}
        \vspace{5mm}
  \includegraphics[width=0.7\linewidth]{grafici/8_periodo_istogramma.png}
  \caption{Ore preventivate per ciascuna persona nel $8^\circ$ periodo}
\end{figure}

\subsubsubsection{Review}
\subsubsubsubsection*{Attività svolte}
Le attività svolte in questo periodo sono state le seguenti:
\begin{itemize}
    \item Redatta ed approvata una prima versione del glossario tecnico;
    \item Redatte le sezioni \emph{Qualità di Prodotto} e \emph{Test di sistema} nel documento \emph{Piano di Qualifica};
    \item Aggiornato \emph{Norme di Progetto} con le attività di verifica, gestione della qualità e la procedura per la redazione del glossario;
    \item Rivisti i costi stabiliti in fase di $\textit{candidatura}_G$;
    \item Integrati i websocket nel $\textit{PoC}_G$, implementate le funzionalità di $\textit{prenotazione}_G$ e $\textit{ordinazione}_G$, non quella di login e raggiunta una configurazione soddisfacente;
    \item Richiesto un incontro con il prof. Cardin per la prima parte della revisione $\textit{RTB}_G$;
    \item Richiesto e fissato un incontro con il proponente in data 11 aprile per presentare il $\textit{PoC}_G$.
\end{itemize}
\subsubsubsubsection*{Consuntivo}
\begin{table}[H]
    \centering
\begin{spreadtab}{{tabular}{|c|c|c|c|c|c|c|c|}}
    \hline
    @\textbf{Membro} & @\textbf{Re} & @\textbf{Amm} & @\textbf{An} & @\textbf{Progr} & @\textbf{Proge} & @\textbf{Ve} & @\textbf{Totale} \\
    \hline
    @ Samuele V.   & 0          & 0          & 0         & 6          & 0     & 0     & sum(b2:g2) \\
    @ Leonardo B.  & 0         & 0          & 0        & 5.42        & 0     & 0    & sum(b3:g3) \\
    @ Riccardo Z.  & 6.75          & 0          & 0          & 0.17         & 0     & 2   & sum(b4:g4) \\
    @ Davide B.    & 5          & 0         & 0       & 0       & 0     & 0     & sum(b5:g5) \\
    @ Michele Z.   & 0          & 4.25          & 0         & 0          & 0     & 0.17     & sum(b6:g6) \\
    @ Filippo T.   & 0          & 0          & 0         & 8.5          & 0     & 0     & sum(b7:g7) \\
    \hline
    @\textbf{Ore totali} & sum(b2:b7) & sum(c2:c7) & sum(d2:d7) & sum(e2:e7) & sum(f2:f7) & sum(g2:g7) &  sum(b8:g8)\\
    \hline
    @\textbf{Costo totale} & 30*b8 & 20*c8 & 25*d8 & 15*e8 & 25*f8 & 15*g8 & sum(b9:g9)\\
    \hline
\end{spreadtab}
    \caption{Consuntivo orario ed economico parziale per l'ottavo periodo, in base al ruolo}
    \label{tab:prev_rtb}
    \vspace{5mm}
    \textbf{Legenda:} \textit{Re} = Responsabile, \textit{Amm} = Amministratore, \textit{An} = Analista, \textit{Progr} = Programmatore, \textit{Proge} = Progettista, \textit{Ve} = Verificatore
\end{table}
\subsubsubsection{Retrospective}
Dal presente periodo sono emersi i seguenti elementi:
\begin{itemize}
    \item E' stato riscontrato durante un meeting un errore sull'automazione per i $\textit{riferimenti}_G$ del glossario, si è dunque deciso di intervenire assegnando la task apposita ad uno dei membri del gruppo dunque ricorrendo ad un innalzamento dei costi. Per mitigare tale problema si è dovuta stabilire una procedura di inserimento delle parole del glossario nei documenti e verifica dei documenti da inserire nei prossimi periodi.
    \item La finalizzazione del documento \emph{Analisi dei Requisiti} si è rivelata più difficoltosa del previsto, portando ad un innalzamento dei costi e facendo emergere delle attività di verifica lacunose e da rivedere per le prossime stesure dei documenti. Per mitigare tale problema si è stabilito che debbano esistere delle procedure da seguire per poter segnare come "verificato" un documento, in particolare si aggiorneranno nei prossimi periodi le sezioni relative a ciascun documento.
\end{itemize}


%\subsubsubsubsection*{Rischi verificati}
%\subsubsubsubsection*{Analisi retrospettiva}
\newpage
% Nono periodo
\subsubsection{Nono periodo (8/4/2024 - 27/4/2024)}
\subsubsubsection{Planning}
\subsubsubsubsection*{Attività pianificate}
Gli obiettivi posti per lo $\textit{sprint}_G$ sono stati i seguenti:
\begin{itemize}
    \item Effettuare un incontro con il proponente fissato in data 10 Aprile in cui mostrare quanto fatto nel $\textit{PoC}_G$;
    \item Effettuare il colloquio con il prof. Cardin per la revisione $\textit{RTB}_G$ in data 12 Aprile per il quale si vuole:
    \begin{itemize}
        \item Preparare le diapositive per il colloquio;
        \item Preparare la lettera di presentazione per la fase $\textit{RTB}_G$.
    \end{itemize}
    \item Per quanto riguarda il $\textit{Piano di Qualifica}_G$ si vuole:
    \begin{itemize}
        \item Individuare le metriche da inserire nella sezione \emph{"Cruscotto di valutazione della qualità"};
        \item Stilare di conseguenza la sezione di \emph{"Modifiche migliorative"};
        \item Aggiornare la sezione \emph{Test di sistema} in base a quanto riportato nell'ultimo aggiornamento dell'Analisi dei Requisiti in particolare riguardo ai requisiti.
    \end{itemize}
    \item Effettuare delle correzioni (eventuali) all'Analisi dei Requisiti fornite dal prof. Cardin in fase di revisione $\textit{RTB}_G$.
    \item Aggiornare $\textit{Norme di Progetto}_G$, inserendo alcune sezioni mancanti come \emph{Validazione}, \emph{repository Proof of Concept} e inserendo come vengono gestite le procedure per alcuni ruoli.
\end{itemize}
\subsubsubsubsection*{Preventivo}
\begin{table}[H]
    \centering
\begin{spreadtab}{{tabular}{|c|c|c|c|c|c|c|c|}}
    \hline
    @\textbf{Membro} & @\textbf{Re} & @\textbf{Amm} & @\textbf{An} & @\textbf{Progr} & @\textbf{Proge} & @\textbf{Ve} & @\textbf{Totale} \\
    \hline
    @ Samuele V.   & 0          & 2.42          & 0         & 0.25          & 0     & 0.55     & sum(b2:g2) \\
    @ Leonardo B.  & 0         & 0          & 1        & 0        & 0     & 2.92    & sum(b3:g3) \\
    @ Riccardo Z.  & 1.5          & 2.5          & 0          & 0.67         & 0     & 0.25  & sum(b4:g4) \\
    @ Davide B.    & 2.5          & 2.5         & 0       & 0       & 0     & 0     & sum(b5:g5) \\
    @ Michele Z.   & 0          & 2.5          & 1.5         & 0          & 0     & 1.67     & sum(b6:g6) \\
    @ Filippo T.   & 0          & 1.83          & 2         & 0.33          & 0     & 0     & sum(b7:g7) \\
    \hline
    @\textbf{Ore totali} & sum(b2:b7) & sum(c2:c7) & sum(d2:d7) & sum(e2:e7) & sum(f2:f7) & sum(g2:g7) &  sum(b8:g8)\\
    \hline
    @\textbf{Costo totale} & 30*b8 & 20*c8 & 25*d8 & 15*e8 & 25*f8 & 15*g8 & sum(b9:g9)\\
    \hline
\end{spreadtab}
    \caption{Preventivo orario ed economico parziale per il nono periodo, in base al ruolo}
    \label{tab:prev_rtb}
    \vspace{5mm}
    \textbf{Legenda:} \textit{Re} = Responsabile, \textit{Amm} = Amministratore, \textit{An} = Analista, \textit{Progr} = Programmatore, \textit{Proge} = Progettista, \textit{Ve} = Verificatore
\end{table}
\subsubsubsection{Review}
\subsubsubsubsection*{Attività svolte}
Le attività svolte in questo periodo sono state le seguenti:
\begin{itemize}
    \item Effettuato un incontro con il proponente per mostrare quanto sviluppato per il $\textit{PoC}_G$ in cui:
    \begin{itemize}
        \item è emerso l'apprezzamento per la $\textit{feature}_G$ di $\textit{ordinazione}_G$ collaborativa;
        \item è emerso che in futuro si debba prendere in considerazione la paginazione per gestire una grande mole di dati in funzionalità quali la ricerca del ristorante;
        \item si è discusso di architetture come Kafka;
        \item si è parlato della possibilità di impostare un progetto implementando un'architettura a microservizi.
    \end{itemize}
    \item Effettuato il colloquio con il prof. Cardin per la revisione $\textit{RTB}_G$ presentando le relative diapositive, il quale è risultato in un semaforo verde con le relative correzioni da effettuare al documento di Analisi dei Requisiti.
    \item Aggiornato $\textit{Norme di Progetto}_G$ e inserite le sezioni mancanti;
    \item Aggiornato $\textit{Piano di Qualifica}_G$ e inserite le metriche nella sezione riguardante il cruscotto;
    \item Corretto il documento di Analisi dei Requisiti con le indicazioni del prof. Cardin;
    \item Aggiornato il presente documento con i relativi grafici per ogni periodo;
    \item Preparata la lettera di presentazione e decisa la data di consegna del progetto rivista.
\end{itemize}
\subsubsubsubsection*{Consuntivo}
\begin{table}[H]
    \centering
\begin{spreadtab}{{tabular}{|c|c|c|c|c|c|c|c|}}
    \hline
    @\textbf{Membro} & @\textbf{Re} & @\textbf{Amm} & @\textbf{An} & @\textbf{Progr} & @\textbf{Proge} & @\textbf{Ve} & @\textbf{Totale} \\
    \hline
    @ Samuele V.   & 0          & 2.08          & 0         & 0.17          & 0     & 0.58     & sum(b2:g2) \\
    @ Leonardo B.  & 0         & 0          & 1        & 0        & 0     & 2.67    & sum(b3:g3) \\
    @ Riccardo Z.  & 1.17          & 1.5          & 0          & 0.67         & 0     & 0.42   & sum(b4:g4) \\
    @ Davide B.    & 2          & 2         & 0       & 0       & 0     & 0     & sum(b5:g5) \\
    @ Michele Z.   & 0          & 1.5          & 0         & 0          & 0     & 0.83     & sum(b6:g6) \\
    @ Filippo T.   & 0          & 1.33          & 1.75         & 0.33          & 0     & 0     & sum(b7:g7) \\
    \hline
    @\textbf{Ore totali} & sum(b2:b7) & sum(c2:c7) & sum(d2:d7) & sum(e2:e7) & sum(f2:f7) & sum(g2:g7) &  sum(b8:g8)\\
    \hline
    @\textbf{Costo totale} & 30*b8 & 20*c8 & 25*d8 & 15*e8 & 25*f8 & 15*g8 & sum(b9:g9)\\
    \hline
\end{spreadtab}
    \caption{Consuntivo orario ed economico parziale per il nono periodo, in base al ruolo}
    \label{tab:prev_rtb}
    \vspace{5mm}
    \textbf{Legenda:} \textit{Re} = Responsabile, \textit{Amm} = Amministratore, \textit{An} = Analista, \textit{Progr} = Programmatore, \textit{Proge} = Progettista, \textit{Ve} = Verificatore
\end{table}

\begin{figure}[H]
  \centering
  \includegraphics[width=0.6\linewidth]{grafici/9_periodo_torta.png}
  \caption{Ripartizione dei costi per ruolo nel $9^\circ$ periodo}
        \vspace{5mm}
  \includegraphics[width=0.7\linewidth]{grafici/9_periodo_istogramma.png}
  \caption{Ore preventivate per ciascuna persona nel $9^\circ$ periodo}
\end{figure}

\subsubsubsection{Retrospective}
La stesura dei documenti si è rilevata più' dispendiosa del previsto e tali documenti sono stati approvati più' tardi rispetto a quanto stabilito all'inizio del periodo. In particolare:
\begin{itemize}
    \item Per l'Analisi dei Requisiti non era chiaro ad alcuni membri del gruppo come poter effettuare le correzioni indicate;
    \item E' emerso che in futuro si dovranno distribuire più' correttamente le attività in base alle disponibilità di ciascun membro.
\end{itemize}


\newpage
% Preventivi
\section{Preventivi}
\subsection{\textit{Preventivo}_G totale}
Per questa sezione per approfondimenti si rimanda al documento \emph{"Dichiarazione impegni v1.2"} citato nella sezione \ref{sec:rif_inf} \emph{"\textit{Riferimenti informativi}_G"}.\\
Come riportato dal sopracitato documento, i costi orari per ogni membro in base al ruolo sono i seguenti:
\begin{table}[htbp]
    \centering    
    \begin{spreadtab}{{tabular}{|c|c|c|c|c|c|c|c|}}
    \hline
    @\textbf{Membro} & @\textbf{Re} & @\textbf{Amm} & @\textbf{An} & @\textbf{Progr} & @\textbf{Proge} & @\textbf{Ve} & @\textbf{Totale} \\
    \hline
    @ Samuele V.   & 9          & 8          & 12         & 26          & 20     & 20     & sum(b2:g2) \\
    @ Leonardo B.  & 9         & 8          & 12         & 26          & 20     & 20     & sum(b3:g3) \\
    @ Riccardo Z.  & 9          & 8          & 12          & 26          & 20     & 20     & sum(b4:g4) \\
    @ Davide B.    & 9          & 8          & 12       & 26          & 20     & 20     & sum(b5:g5) \\
    @ Michele Z.   & 9          & 8          & 12         & 26          & 20     & 20     & sum(b6:g6) \\
    @ Filippo T.   & 9          & 8          & 12          & 26          & 20     & 20     & sum(b7:g7) \\
    \hline
    @\textbf{Ore totali} & sum(b2:b7) & sum(c2:c7) & sum(d2:d7) & sum(e2:e7) & sum(f2:f7) & sum(g2:g7) &  sum(b8:g8)\\
    \hline
    @\textbf{Costo totale} & 30*b8 & 20*c8 & 25*d8 & 15*e8 & 25*f8 & 15*g8 & sum(b9:g9)\\
    \hline
    \end{spreadtab}
    \caption{\textit{Preventivo}_G orario ed economico totale, in base al ruolo}
    \label{tab:prev_totale}
    \vspace{5mm}
    \textbf{Legenda:} \textit{Re} = Responsabile, \textit{Amm} = \textit{Amministratore}_G, \textit{An} = Analista, \textit{Progr} = Programmatore, \textit{Proge} = Progettista, \textit{Ve} = Verificatore
\end{table}

\subsection{\textit{Preventivo}_G \textit{RTB}_G}
In questa sezione è presente il preventivo orario ed economico parziale scelto per la fase \textit{RTB}_G.
\begin{table}[htbp]
    \centering
\begin{spreadtab}{{tabular}{|c|c|c|c|c|c|c|c|}}
    \hline
    @\textbf{Membro} & @\textbf{Re} & @\textbf{Amm} & @\textbf{An} & @\textbf{Progr} & @\textbf{Proge} & @\textbf{Ve} & @\textbf{Totale} \\
    \hline
    @ Samuele V.   & 4          & 4          & 10         & 6          & 4     & 8     & sum(b2:g2) \\
    @ Leonardo B.  & 4         & 4          & 10        & 6          & 4     & 8    & sum(b3:g3) \\
    @ Riccardo Z.  & 4          & 4          & 10          & 6          & 4     & 8    & sum(b4:g4) \\
    @ Davide B.    & 4          & 4          & 10       & 6          & 4     & 8     & sum(b5:g5) \\
    @ Michele Z.   & 4          & 4          & 10         & 6          & 4     & 8     & sum(b6:g6) \\
    @ Filippo T.   & 4          & 4          & 10         & 6          & 4     & 8     & sum(b7:g7) \\
    \hline
    @\textbf{Ore totali} & sum(b2:b7) & sum(c2:c7) & sum(d2:d7) & sum(e2:e7) & sum(f2:f7) & sum(g2:g7) &  sum(b8:g8)\\
    \hline
    @\textbf{Costo totale} & 30*b8 & 20*c8 & 25*d8 & 15*e8 & 25*f8 & 15*g8 & sum(b9:g9)\\
    \hline
\end{spreadtab}
    \caption{\textit{Preventivo}_G orario ed economico parziale per la fase \textit{RTB}_G, in base al ruolo}
    \label{tab:prev_rtb}
    \vspace{5mm}
    \textbf{Legenda:} \textit{Re} = Responsabile, \textit{Amm} = \textit{Amministratore}_G, \textit{An} = Analista, \textit{Progr} = Programmatore, \textit{Proge} = Progettista, \textit{Ve} = Verificatore
\end{table}
\\\\\\\\
\subsection{\textit{Preventivo}_G \textit{PB}_G}
In questa sezione è presente il preventivo orario ed economico parziale scelto per la fase \textit{PB}_G.
\begin{table}[htbp]
    \centering
\begin{spreadtab}{{tabular}{|c|c|c|c|c|c|c|c|}}
    \hline
    @\textbf{Membro} & @\textbf{Re} & @\textbf{Amm} & @\textbf{An} & @\textbf{Progr} & @\textbf{Proge} & @\textbf{Ve} & @\textbf{Totale} \\
    \hline
    @ Samuele V.   & 5          & 4          & 2         & 20          & 16     & 12     & sum(b2:g2) \\
    @ Leonardo B.  & 5         & 4          & 2        & 20          & 16     & 12    & sum(b3:g3) \\
    @ Riccardo Z.  & 5          & 4          & 2          & 20          & 16     & 12    & sum(b4:g4) \\
    @ Davide B.    & 5          & 4          & 2       & 20          & 16     & 12     & sum(b5:g5) \\
    @ Michele Z.   & 5          & 4          & 2         & 20          & 16     & 12     & sum(b6:g6) \\
    @ Filippo T.   & 5          & 4          & 2         & 20          & 16     & 12     & sum(b7:g7) \\
    \hline
    @\textbf{Ore totali} & sum(b2:b7) & sum(c2:c7) & sum(d2:d7) & sum(e2:e7) & sum(f2:f7) & sum(g2:g7) &  sum(b8:g8)\\
    \hline
    @\textbf{Costo totale} & 30*b8 & 20*c8 & 25*d8 & 15*e8 & 25*f8 & 15*g8 & sum(b9:g9)\\
    \hline
\end{spreadtab}
    \caption{\textit{Preventivo}_G orario ed economico parizale per la fase \textit{PB}_G, in base al ruolo}
    \label{tab:prev_pb}
    \vspace{5mm}
    \textbf{Legenda:} \textit{Re} = Responsabile, \textit{Amm} = \textit{Amministratore}_G, \textit{An} = Analista, \textit{Progr} = Programmatore, \textit{Proge} = Progettista, \textit{Ve} = Verificatore
\end{table}
\newpage
% Consuntivi
\section{Consuntivi}
\subsubsection{Excel, individuali}

\vspace{10 mm}
\begin{spreadtab}{{tabular}{|c|c|c|c|c|c|c|c|}}
    \hline
    @\textbf{Membro} & @\textbf{Re} & @\textbf{Amm} & @\textbf{An} & @\textbf{Pg} & @\textbf{Pr} & @\textbf{Ve} & @\textbf{Totale} \\
    \hline
    @ Samuele V.   & 15          & 0          & 4         & 0          & 0     & 0     & sum(b2:f2) \\
    @ Michele Z.   & 0          & 4          & 15         & 0          & 0     & 0     & sum(b3:f3) \\
    @ Leonardo B.  & 6         & 0          & 5         & 9           & 0     & 0     & sum(b4:f4) \\
    @ Riccardo Z.  & 0          & 0          & 9          & 18          & 0     & 0     & sum(b5:f5) \\
    @ Filippo T.   & 6          & 0          & 0          & 0          & 0     & 4     & sum(b6:f6) \\
    @ Davide B.    & 0          & 3          & 3       & 0          & 0     & 3     & sum(b7:f7) \\
    \hline
    @\textbf{Ore totali} & sum(b2:b7) & sum(c2:c7) & sum(d2:d7) & sum(e2:e7) & sum(f2:f7) & sum(g2:g7) &  sum(b8:g8)\\
    \hline
    @\textbf{Costo totale} & 30*b8 & 20*c8 & 25*d8 & 15*e8 & 25*f8 & 15*g8 & sum(b9:g9)\\
    \hline
\end{spreadtab}
\vspace{10 mm}

Pg = programmatore, Pr = progettista\\

\subsubsection{Excel, produttive}

\vspace{10 mm}
\begin{spreadtab}{{tabular}{|c|c|c|c|c|c|c|c|}}
    \hline
    @\textbf{Membro} & @\textbf{Re} & @\textbf{Amm} & @\textbf{An} & @\textbf{Pg} & @\textbf{Pr} & @\textbf{Ve} & @\textbf{Totale} \\
    \hline
    @ Samuele V.   & 3          & 0          & 1         & 0          & 0     & 0     & sum(b2:f2) \\
    @ Michele Z.   & 0          & 1          & 2         & 0          & 0     & 0     & sum(b3:f3) \\
    @ Leonardo B.  & 2         & 0          & 2         & 1          & 0     & 0     & sum(b4:f4) \\
    @ Riccardo Z.  & 0          & 0          & 3          & 3          & 0     & 0     & sum(b5:f5) \\
    @ Filippo T.   & 6          & 0          & 0          & 0          & 0     & 4     & sum(b6:f6) \\
    @ Davide B.    & 0          & 3          & 2       & 0          & 0     & 3     & sum(b7:f7) \\
    \hline
    @\textbf{Ore totali} & sum(b2:b7) & sum(c2:c7) & sum(d2:d7) & sum(e2:e7) & sum(f2:f7) & sum(g2:g7) &  sum(b8:g8)\\
    \hline
    @\textbf{Costo totale} & 30*b8 & 20*c8 & 25*d8 & 15*e8 & 25*f8 & 15*g8 & sum(b9:g9)\\
    \hline
\end{spreadtab}
\vspace{10 mm}

Pg = programmatore, Pr = progettista\\

\subsubsection{Jira, produttive}
\vspace{10 mm}
\begin{spreadtab}{{tabular}{|c|c|c|c|c|c|c|c|}}
    \hline
    @\textbf{Membro} & @\textbf{Re} & @\textbf{Amm} & @\textbf{An} & @\textbf{Pg} & @\textbf{Pr} & @\textbf{Ve} & @\textbf{Totale} \\
    \hline
    @ Samuele V.   & 0          & 0          & 6        & 0          & 0     & 0     & sum(b2:g2) \\
    @ Michele Z.   & 0          & 1          & 0         & 0          & 0     & 0     & sum(b3:g3) \\
    @ Leonardo B.  & 3         & 10          & 0         & 0          & 0     & 0     & sum(b4:g4) \\
    @ Riccardo Z.  & 0          & 0          & 8          & 4          & 0     & 0     & sum(b5:g5) \\
    @ Filippo T.   & 0          & 0          & 0          & 0          & 0     & 4     & sum(b6:g6) \\
    @ Davide B.    & 0          & 0          & 11       & 0          & 0     & 1     & sum(b7:g7) \\
    \hline
    @\textbf{Ore totali} & sum(b2:b7) & sum(c2:c7) & sum(d2:d7) & sum(e2:e7) & sum(f2:f7) & sum(g2:g7) &  sum(b8:g8)\\
    \hline
    @\textbf{Costo totale} & 30*b8 & 20*c8 & 25*d8 & 15*e8 & 25*f8 & 15*g8 & sum(b9:g9)\\
    \hline
\end{spreadtab}
\vspace{10 mm}

Pg = programmatore, Pr = progettista\\
\begin{comment}
\subsection{Consuntivo RTB}

\newcommand{\totOre}{84}
\newcommand{\totCosto}{1895}

\subsubsection{Ripartizione ruoli per ore}
\begin{figure}[h]
    \centering
    \edef\percentA{\fpeval{(14/\totOre)*100}}
    \edef\percentB{\fpeval{(15/\totOre)*100}}
    \edef\percentC{\fpeval{(35/\totOre)*100}}
    \edef\percentD{\fpeval{(8/\totOre)*100}}
    \edef\percentE{\fpeval{(12/\totOre)*100}}
    \begin{tikzpicture}
        \pie[text=legend, sum=auto, radius=1.5, color={blue, red, green, orange, yellow}, hide number]
            {\percentA/Responsabile (\pgfmathprintnumber{\percentA}\%), \percentB/Amministratore (\pgfmathprintnumber{\percentB}\%), \percentC/Analista (\pgfmathprintnumber{\percentC}\%), \percentD/Programmatore (\pgfmathprintnumber{\percentD}\%), \percentE/Verificatore (\pgfmathprintnumber{\percentE}\%)}
    \end{tikzpicture}
    \caption{Distribuzione delle ore}
\end{figure}

\subsubsection{Ripartizione ruoli per costo}
\begin{figure}[h]
    \centering
    \edef\percentA{\fpeval{(420/\totCosto)*100}}
    \edef\percentB{\fpeval{(300/\totCosto)*100}}
    \edef\percentC{\fpeval{(875/\totCosto)*100}}
    \edef\percentD{\fpeval{(120/\totCosto)*100}}
    \edef\percentE{\fpeval{(180/\totCosto)*100}}
    \begin{tikzpicture}
        \pie[text=legend, sum=auto, radius=1.5, color={blue, red, green, orange, yellow}, hide number]
            {\percentA/Responsabile (\pgfmathprintnumber{\percentA}\%), \percentB/Amministratore (\pgfmathprintnumber{\percentB}\%), \percentC/Analista (\pgfmathprintnumber{\percentC}\%), \percentD/Programmatore (\pgfmathprintnumber{\percentD}\%), \percentE/Verificatore (\pgfmathprintnumber{\percentE}\%)}
    \end{tikzpicture}
    \caption{Distribuzione dei costi}
\end{figure}
\subsubsection{Ruoli per persona}

\begin{tikzpicture}
    \begin{axis}[
        ybar stacked,
        bar width=0.5cm,
        ylabel={Ore},
        xtick=data,
        xticklabels={\hspace{0.5cm}\scriptsize Samuele V.,\hspace{0.5cm}\scriptsize Michele Z.,\hspace{0.5cm}\scriptsize Leonardo B.,\hspace{0.5cm}\scriptsize Riccardo Z.,\hspace{0.5cm}\scriptsize Filippo T.,\hspace{0.5cm}\scriptsize Davide B.,\hspace{0.5cm}\scriptsize Missing Member},
        yticklabel={\pgfmathprintnumber{\tick}},
        ymin=0,
        ymax=30,
        width=15cm,
        height=6cm,
        legend style={at={(0.5,-0.15)},
        anchor=north,legend columns=-1},
        enlarge x limits=0.2,
        ]
        \addplot coordinates {(0,3) (1,0) (2,5) (3,0) (4,6) (5,0)};
        \addplot coordinates {(0,0) (1,2) (2,10) (3,0) (4,0) (5,3)};
        \addplot coordinates {(0,7) (1,2) (2,2) (3,11) (4,0) (5,13)};
        \addplot coordinates {(0,0) (1,0) (2,1) (3,7) (4,0) (5,0)};
        \addplot coordinates {(0,0) (1,0) (2,0) (3,0) (4,8) (5,4)};
        \legend{Responsabile, Amministratore, Analista, Programmatore, Verificatore}
    \end{axis}
\end{tikzpicture}

\subsubsection{Distanza dal preventivo progetto}
\begin{figure}[h]
    \centering
    \begin{tikzpicture}
        \pgfplotsset{width=0.8\textwidth, height=6cm, symbolic x coords={Responsabile, Amministratore, Analista, Programmatore, Verificatore}, xtick=data}
        
        \begin{axis}[
            ybar stacked,
            bar width=0.5cm,
            enlarge x limits=0.2,
            legend style={at={(0.5,-0.15)}, anchor=north,legend columns=-1},
            ylabel={Ore},
            xlabel={Ruolo},
            symbolic x coords={Responsabile, Amministratore, Analista, Programmatore, Verificatore},
            xtick=data,
        ]
            \addplot coordinates {(Responsabile, 12) (Amministratore, 18) (Analista, 35) (Programmatore, 22) (Verificatore, 14)};
            \addplot coordinates {(Responsabile, 10) (Amministratore, 20) (Analista, 30) (Programmatore, 25) (Verificatore, 15)};
            \legend{Attese, Effettive}
        \end{axis}
    \end{tikzpicture}
    \caption{Confronto ore attese ed effettive per ruolo}
\end{figure}

\begin{figure}[h]
	\centering
	\begin{tikzpicture}
		\pgfplotsset{width=0.8\textwidth, height=8cm, symbolic y coords={Responsabile, Amministratore, Analista, Programmatore, Verificatore}, ytick=data}
		
		\begin{axis}[
			xbar,
			bar width=0.4cm,
			enlarge y limits=0.1,
			legend style={at={(0.5,-0.15)}, anchor=north,legend columns=-1},
			xlabel={Ore},
			ylabel={Ruolo},
			symbolic y coords={Responsabile, Amministratore, Analista, Programmatore, Verificatore},
			ytick=data,
			yticklabels={Responsabile, Amministratore, Analista, Programmatore, Verificatore},
			ymajorgrids=true,
			xtick={0,20,40,60,80,100,120,140,160},
			xmajorgrids=true,
			grid style=dashed,
			ytick distance=1,
			xlabel style={yshift=10pt}, % Aggiungi un offset verticale di 10pt all'etichetta x
		]
			\addplot coordinates {(54,Responsabile) (48,Amministratore) (72,Analista) (156,Programmatore) (14,Verificatore)};
			\addplot coordinates {(14,Responsabile) (0,Amministratore) (35,Analista) (0,Programmatore) (0,Verificatore)};
			\legend{Attese, Effettive}
		\end{axis}
	\end{tikzpicture}
	\caption{Confronto ore attese ed effettive per ruolo}
\end{figure}




\begin{figure}
	\centering
	\begin{tikzpicture}
		\begin{axis}[
				ybar,
				symbolic x coords={Samuele V., Michele Z., Leonardo B., Riccardo Z., Filippo T., Davide B.},
				xtick=data,
				ylabel={Ore rimanenti},
				legend style={at={(0.5,-0.15)},
						anchor=north,legend columns=-1},
				width=0.9\textwidth,
			]
			\addplot coordinates {(Samuele V., 20) (Michele Z., 25) (Leonardo B., 15) (Riccardo Z., 30) (Filippo T., 0) (Davide B., 0) (Missing Member, 0)};
			\addplot coordinates {(Samuele V., 15) (Michele Z., 10) (Leonardo B., 20) (Riccardo Z., 25) (Filippo T., 0) (Davide B., 0) (Missing Member, 0)};
			\addplot coordinates {(Samuele V., 10) (Michele Z., 15) (Leonardo B., 10) (Riccardo Z., 20) (Filippo T., 0) (Davide B., 0) (Missing Member, 0)};
			\legend{Responsabile, Amministratore, Altri ruoli}
		\end{axis}
	\end{tikzpicture}
	\caption{Ore rimanenti per ruolo per ogni membro del gruppo}
\end{figure}
\end{comment}

\end{document}