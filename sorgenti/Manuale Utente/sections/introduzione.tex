\section{Introduzione}
\subsection{Scopo del documento}
Il presente documento ha lo scopo di fornire un supporto completo agli utenti del $\textit{sistema}_G$ $\textit{Easy Meal}_G$ richiesto dall'azienda $\textit{Imola Informatica}_G$. Vengono dunque illustrate le istruzioni per l'utilizzo delle funzionalità fornite dall'applicazione oltre che i requisiti minimi necessari per il corretto funzionamento di $\textit{Easy Meal}_G$.\\
Eventuali termini tecnici sono definiti all'interno del documento "Glossario Tecnico".

\subsection{Scopo del prodotto}
Il prodotto finale, realizzato tramite un'$\textit{Applicazione Web Responsive}_G$, si propone di realizzare un $\textit{software}_G$ innovativo volto a semplificate il $\textit{processo}_G$ di $\textit{prenotazione}_G$ e $\textit{ordinazione}_G$ nei ristoranti, contribuendo a migliorare l'esperienza per clienti e ristoratori. In particolare, $\textit{Easy Meal}_G$ dovrà consentire agli utenti di personalizzare gli ordini in base alle proprie preferenze, allergie ed esigenze alimentari; interagire direttamente con lo staff del ristorante attraverso una chat integrata; consentire di dividere il conto tra i partecipanti al tavolo.
\subsection{Riferimenti}
\subsubsection{Riferimenti normativi}
\begin{enumerate}
    \item Presentazione del $\textit{capitolato}_G$ d'appalto C3 - Progetto $\textit{Easy Meal}_G$: \\ \url{https://www.math.unipd.it/~tullio/IS-1/2023/Progetto/C3.pdf}\\
    (Consultato: 2024-05-13)
    \item Regolamento del progetto didattico: \\ \url{https://www.math.unipd.it/~tullio/IS-1/2023/Dispense/PD2.pdf}\\
    (Consultato: 2024-05-13)
    \item $\textit{Norme di Progetto}_G$ v2.0.0.
\end{enumerate}
\subsubsection{Riferimenti informativi}
\begin{enumerate}
    \item Glossario v2.0.0.
\end{enumerate}