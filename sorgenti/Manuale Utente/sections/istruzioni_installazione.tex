\section{Istruzioni all'installazione}

\subsection{Download del progetto}
Per scaricare il progetto, clonare il $\textit{repository}_G$ presente su $\textit{Github}_G$.

\begin{lstlisting}[language=bash, caption=Download repository]
git clone https://github.com/RAMtastic6/EasyMeal.git
cd EasyMeal
\end{lstlisting}

\subsection{Avvio dell'applicazione}
L'applicazione si compone di quattro moduli separati che devono essere avviati; per fare ciò si utilizza $\textit{Docker}_G$.

\subsubsection{Docker}
Questo progetto utilizza $\textit{Docker}_G$ Compose per gestire l'avvio dei container $\textit{Docker}_G$; seguire le istruzioni di seguito per avviare i container e utilizzare l'applicazione.

\paragraph{Avvio dei container} Per avviare i container utilizzando $\textit{Docker}_G$ Compose, eseguire il seguente comando nella directory del progetto:

\begin{lstlisting}[language=bash, caption=Inizializzare i container]
docker-compose up
\end{lstlisting}
Una volta avviati i container, sarà possibile accedere all'applicazione utilizzando il browser o gli strumenti di sviluppo appropriati; per poter accedere al progetto $\textit{Next.js}_G$ collegarsi al link: \url{http://localhost:3000/}.

\paragraph{Modifica dei file Dockerfile e compose.yaml} Per applicare le modifiche e far partire il progetto utilizzare il seguente comando:

\begin{lstlisting}[language=bash, caption=Riconstruire e inizializzare i container]
docker-compose up --build
\end{lstlisting}

\paragraph{Ripetizione del processo} se si desidera cancellare completamente l'ambiente di esecuzione utilizzare il seguente comando:

\begin{lstlisting}[language=bash, caption=Rimozione dei container e dei volumi]
docker-compose down -v 
\end{lstlisting}
Per ulteriori informazioni su $\textit{Docker}_G$ Compose, consultare la documentazione ufficiale: \url{https://docs.docker.com/compose/} (Consultato: 2024-06-11)