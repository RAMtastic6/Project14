\subsection{Acquisizione}
\subsubsection{Descrizione}
Nel $\textit{processo}_G$ di acquisizione avviene la raccolta e la comprensione dei requisiti con lo scopo di identificare un $\textit{capitolato}_G$ adeguato per il gruppo da proporre per la candidatura.
\subsubsection{Valutazione capitolati}
Il gruppo ha analizzato la proposta dei proponenti valutandone, in base all'esperienza del gruuppo, la loro \textit{complessità}, tale analisi è stata determinante per la scelta definitiva.
\subsubsection{Appalto capitolati}
Il gruppo si è proposto per il $\textit{capitolato}_G$ C3 dell'azienda IMOLA INFORMATICA. Nonostante un  un primo riscontro negativo, a seguito di modifiche volte a sistemare lacune presenti  presenti nella $\textit{candidatura}_G$ iniziale, il gruppo è riuscito ad aggiudicarsi il $\textit{capitolato}_G$.

\subsection{Fornitura}
\subsubsection{Descrizione}
Il $\textit{processo}_G$ di $\textit{fornitura}_G$ consiste nel chiarire ogni dubbio legato al prodotto finale che il proponente desidera; in modo da evitare incomprensioni durante lo svolgimento del progetto, il gruppo 14 si impegna a comunicare con l'azienda, in modo da raggiungere i seguenti obiettivi:
\begin{enumerate}
    \item determinare i requisiti da soddisfirare nel prodotto finale;
    \item ottenere incontri di formazioni sulle tecnogie e strumenti consigliati dall'azienda per realizzare il prodotto;
    \item ricevere $\textit{feedback}_G$ in fase di sviluppo su quanto precedemente svolto.
\end{enumerate}
\subsubsection{Piano di progetto}
Il documento \textit{Piano di progetto} costituisce uno $\textit{strumento}_G$ di pianificazione per tutte le attività che dovranno essere svolte così da poter rispettare la data di consegna del progetto.
Nel dettaglio il $\textit{piano di progetto}_G$ è composto da:
\begin{itemize}
    \item Analisi dei rischi: analisi delle difficoltà che il gruppo potrebbe riscontrare durante lo svolgimento del progetto, in particolar modo a livello organizzativo e tecnologico;
    \item Modello di sviluppo;
    \item Pianificazione;
    \item Preventivo: che rispecchi quanto comunicato in fase di candidatura;
    \item Consuntivo: tracciamento dell'andamento del gruppo rispetto al $\textit{preventivo}_G$ fatto.
\end{itemize}
\subsubsection{Piano di qualifica}
Nel documento \textit{Piano di qualifica} vengono elencate tutte le attività svolte dal verificatore con l'obiettivo di garantire la qualità del prodotto finale.
E' formato dai seguenti componenti:
\begin{itemize}
    \item Qualità di processo
    \item Qualità di prodotto
    \item Test: eseguiti sul prodotto che assicurano che i requisiti siano stati rispettati
    \item Resoconto
\end{itemize}
\subsubsection{Rilascio}
Quando il prodotto verrà ultimato, verrà collaudato per garantire il suo corretto funzionamento. Se il prodotto supera il collaudo, verrà consegnato al committente insieme alla documentazione del progetto. 
Il gruppo non effettuerrà manutenzione una volta rilasciato il prodotto.

\subsection{Sviluppo}
\subsubsection{Descrizione}
Lo scopo del $\textit{processo}_G$ di sviluppo è quello di dichiarare le attività da svolgere per raggiungere i requisiti necessari del prodotto.
A tale scopo, il gruppo si dividerà in diversi ruoli, i quali avranno compiti precisi da svolgere.
\subsubsection{Analisi dei requisiti}
\textit{Analisi dei requisiti} è un documento fondamentale per lo sviluppo. Infatti, tale documento deve indicare i requisiti necessari del prodotto finale, che a loro volta andranno a rispecchiare le aspettative del proponente. \\
Inoltre, questo documento servirà anche come documentazione del prodotto, andandone infatti a contenere tutte le relative funzionalità.
\paragraph{Attori} 
Innanzitutto, verranno definiti gli attori e una panoramica dei vari casi d'uso a loro associati tramite un diagramma
\paragraph{Casi d'uso}
Successivamente, verranno descritti i vari casi d'uso, i quali hanno il compito di rappresentare le funzionalità che il prodotto finale dovrà rispettare. \\
I casi d'uso saranno ordinati per attore, ovvero verranno prima descritti tutti i casi d'uso associati a un particolare attore, per poi proseguire con la descrizione di tutti i casi d'uso riguardanti l'attore successivo, e a proseguire. \\
Ogni $\textit{caso d'uso}_G$ avrà un diagramma ad esso associato in cui verrà inserita la funzionalità che si vuole descrivere e le eventuali estensioni per le eccezioni. Verrano poi inseriti diagrammi che illustreranno i vari sottocasi d'uso, cercando di arrivare a casi d'uso cosiddetti "atomici". \\
I casi d'uso verranno inoltre descritti verbalmente, tramite una struttura standard da rispettare:
%I casi d'uso hanno il ruolo di rappresentare le funzionalità che il prodotto finale deve rispattare, quindi hanno una struttura standard da rispettare:
\begin{itemize}
    \item Attori
    \item Precondizioni
    \item Postcondizioni
    %\item $\textit{Scenario}_G$ primario
    %\item $\textit{Scenario}_G$ secondario (opzionale)
    \item $\textit{Scenario}_G$ primario
    \item Scenari alternativi (opzionale)
\end{itemize}
\begin{comment}
I sottocasi d'uso non verranno descritti individualmente, poiché sarà già tutto descritto a livello atomico nello $\textit{scenario}_G$ principale del relativo $\textit{caso d'uso}_G$ "padre".
\end{comment}
Gli scenari alternativi saranno presenti solo se il $\textit{caso d'uso}_G$ può generare eccezioni: poiché a più eccezioni corrispondono una singola modalità di gestione delle stesse, per ogni $\textit{scenario}_G$ alternativo ci potranno essere più primi punti, i quali verranno rappresentati nel formato "\textit{1.xa}", nel quale '\textit{x}' rappresenta il punto dello $\textit{scenario}_G$ principale dal quale viene generata l'eccezione, mentre '\textit{a}' rappresenta il tipo di eccezione. Verrà poi descritto dal punto "\textit{2}" a seguire lo $\textit{scenario}_G$ alternativo.
\paragraph{Requisiti} 
Verranno infine descritti i requisiti.
\subsubsection{Progettazione}
La progettazione, a carico della figura del progettista, definisce la struttura del progetto basandosi sull'analisi dei requisiti.
La progettazione avviene su più livelli:
\begin{enumerate}
    \item Progettazione architetturale: dove viene scelta la struttura del sistema
    \item Design: ovvero il $\textit{design}_G$ dell'interfaccia che deve avere il prodotto
    \item Progettazione dettagliata: le specifiche dei componenti del prodotto che comprendono le specifiche architetturali, i diagrammi delle classi e i $\textit{test}_G$ d'unità
\end{enumerate}
\subsubsection{Codifica}
I programmatori, dopo l'analisi e la progettazione, implementano le funzionalità che deve avere il prodotto finale basandosi sull'analisi dei requisiti e sui documenti di progettazione.\\
Inoltre, durante questa fase, si devono implementare $\textit{test}_G$ che assicurano il corretto funzionamento del prodotto.