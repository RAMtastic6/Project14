\paragraph{Ruoli di progetto}
In questa sezione viene riportata una breve descrizione dei ruoli e delle responsabilità dei membri di un gruppo dedicato allo sviluppo di un qualsiasi tipo di \textit{project}.
\subsection{Tutti i ruoli}
Ogni ruolo dovrà tracciare le ore svolte per completare i $\textit{ticket}_G$ definiti dal responsabile inserendo quanto tempo si è speso per completarla sul progetto di $\textit{jira}_G$.
Per la redazione di tutti i documenti eccetto il glossario tecnico si procede nel seguente modo:
\begin{itemize}
    \item Si modifica il documento su overleaf
    \item Segnalare eventuali parole da aggiungere al glossario
    \item Aggiornare il $\textit{changelog}_G$ riportando, in breve, la modifica apportata al documento
    \item Una volta scritto, spostare il $\textit{ticket}_G$ in "da verificare" su jira
\end{itemize}
\subsection{Responsabile di Progetto}
Il Responsabile di Progetto è la figura professionale, punto di riferimento sia per il committente sia per il fornitore, con lo scopo di
mediare tra le due parti. Assume la responsabilità delle decisioni del gruppo dopo averle approvate.\\
Le sue responsabilità includono:
\begin{itemize}
    \item Approvare l’emissione della documentazione;
    \item Approvare l’offerta economica sottoposta al committente;
    \item Pianificare e coordinare le attività di progetto;
    \item Gestire le risorse umane;
    \item Studiare e gestire i  $\textit{rischi}_G$.
    \item Chiedere l'approvazione dei verbali alle persone esterne che hanno partecipato
    \item Assegnare le attività, tramite la funzione di tracking issues fornito da $\textit{Jira}_G$, ai membri che le dovranno svolgere
    \item Gestire le \textit{milestone} e fissarne di nuove, o modificare quelle attuali in base all'andamento del team
\end{itemize}
\subsubsection{Gestione dello sprint}
Inoltre, il responsabile dovrà gestire le tasks tramite $\textit{jira}_G$, ovvero all'inizio di ogni $\textit{sprint}_G$ dovrà:
\begin{enumerate}
    \item Individuare delle attività da svolgere nel periodo di sprint
    \item Determinare la durata dello sprint
    \item Creare dei tickets, associati alle attività, su $\textit{jira}_G$ impostando un tempo previsto per svolgere l'attività (le attività di verifica vengono aggiunte anche durante il periodo di $\textit{sprint}_G$)
    \item Decidere i ruoli dei membri ed associarli alle attività che ognuno dovrà svolgere
    \item Stilare il $\textit{preventivo}_G$ sul file excel apposito facendo la somma delle ore stimate per concludere le attività
    \item Creare un nuovo periodo nel PDP nella sezione "Pianificazione" e nella sottosezione planning inserire le attività da svolgere e il $\textit{preventivo}_G$ (creato nel punto precedente tramite excel)
\end{enumerate}
Durante lo $\textit{sprint}_G$ il responsabile di progetto dovrà:
\begin{enumerate}
    \item Inserire nel PDP (nella sezione "Review") le attività completate
    \item Aggiornare il $\textit{consuntivo}_G$ nel file excel con le attività completate
    \item Inserire gli eventuali $\textit{rischi}_G$ nella sezione "Analisi dei $\textit{rischi}_G$"
\end{enumerate}
Infine, alla fine del periodo:
\begin{enumerate}
    \item Inserire il $\textit{consuntivo}_G$ nella sezione Review del Piano di Progetto
    \item Completare la sezione "Retrospective", analizzando le criticità e i $\textit{rischi}_G$ e su come si potrebbero risolvere
\end{enumerate}
\subsubsection{Verifica dei documenti}
Il responsabile, quando un documento viene verificato, lo inserirà nella $\textit{repository}_G$ \href{https://github.com/RAMtastic6/Project14}{Project14} nel $\textit{branch}_G$ develop, aggiornando il file README.
Quindi se il file che viene verificato è un documento (tranne il glossario) si deve:
\begin{enumerate}
    \item Scaricare il sorgente del documento tramite $\textit{overleaf}_G$ (si deve andare in progetto $>$ menù $>$ sorgente)
    \item Inserire la cartella in develop, eventualmente sostituendola a quella presente
    \item Utilizzare il seguente comando per far partire l'automazione: python3 entry\_point\_automazioni.py, si otterrà un file $\textit{latex}_G$ con i $\textit{riferimenti}_G$ al glossario
    \item Tramite LatexWorkshop, \href{https://github.com/James-Yu/LaTeX-Workshop/wiki/Install}{per l'installazione} , (o strumenti simili) generare il file $\textit{pdf}_G$ corrispondente
    \item Inserire il file $\textit{pdf}_G$ nella cartella documenti aggiornando il README
    \item Segnare sia il $\textit{ticket}_G$ della stesura del documento e sia della verifica del documento in "completato" tramite jira
\end{enumerate}
Se invence il documento approvato è il glossario allora si procede nel seguente modo:
\begin{enumerate}
    \item Si approva la $\textit{pull request}_G$ associata in develop
    \item Si inseriscono tutti i sorgenti dei documenti presenti su $\textit{overleaf}_G$ (se approvati) tranne i verbali
    \item Si utilizza l'automazione: python3 entry\_point\_automazioni.py
    cosicché vengono generati i file $\textit{latex}_G$ dei documenti e del glossario
    \item Utilizzando LatexWorkshop per generare i $\textit{pdf}_G$ e si inseriscono nella $\textit{repository}_G$ su develop
    \item Infine si aggiorna il README
\end{enumerate}
\subsubsection{Verifica del codice}
Invece, se invece l'attività riguardava al codice allora bisogna approvare o rifiutare la pull request
\subsection{Amministratore di Progetto}
L'Amministratore di Progetto è responsabile delle procedure di controllo e amministrazione dell’ambiente di
lavoro, con piena responsabilità sulla capacità operativa e sull’efficienza.\\
In particolare, si occupa di:
\begin{itemize}
    \item Ricercare, studiare e mettere in opera risorse per migliorare l’ambiente di lavoro, automatizzandolo quando possibile;
    \item Risolvere problemi legati alla gestione dei processi;
    \item Salvaguardare la documentazione di progetto;
    \item Effettuare il controllo di versioni e configurazioni del prodotto $\textit{software}_G$;
    \item Redigere e attuare i piani e le procedure per la gestione della qualità.
\end{itemize}
\subsection{Analista}
L'Analista è una figura con maggiori competenze riguardo al dominio applicativo del problema. \\
Le sue responsabilità includono:
\begin{itemize}
    \item Studiare il problema e il relativo contesto applicativo;
    \item Comprendere il problema e definire la complessità e i requisiti;
    \item Redigere il documento "$\textit{Analisi Dei Requisiti}_G$";
    \item Studiare i casi d'uso e redigere il loro relativo schema $\textit{UML}_G$.
    \item Aggiornare, in base alle attività definite dai tickets, i documenti Norme di progetto e Piano di qualifica
\end{itemize}
\subsection{Progettista}
Il Progettista gestisce gli aspetti tecnologici e tecnici del progetto. \\
In particolare, si occupa di:
\begin{itemize}
    \item Effettuare scelte riguardanti gli aspetti tecnici e tecnologici del progetto, favorendone l'efficacia e l'efficienza;
    \item Definire un'$\textit{architettura}_G$ del prodotto da sviluppare che miri all'economicità e alla manutenibilità a partire dal lavoro svolto dall'Analista;
    \item Redigere la "$\textit{specifica tecnica}_G$" e la parte pragmatica del "$\textit{Piano di Qualifica}_G$".
\end{itemize}

\subsection{Verificatore}
Il Verificatore è responsabile della sorveglianza sul lavoro svolto dagli altri componenti del gruppo, sulla base delle proprie competenze tecniche, esperienza e conoscenza delle norme. \\
In particolare, si occupa di:
\begin{itemize}
    \item Esaminare i prodotti in fase di revisione, con l'ausilio delle tecniche e degli strumenti definiti nel presente documento;
    \item Verificare la conformità dei prodotti ai requisiti funzionali e di qualità;
    \item Verificare i documenti segnalando eventuali errori.
    \item Caricare i verbali, dopo averli verificati, all'interno della $\textit{Repository}_G$ $\textit{Github}_G$ nel ramo \textit{develop}.
\end{itemize}
Si precisa che la verifica dei documenti deve essere svolta tramite controllo della sintassi e semantica del documento in questione e, inoltre, deve assicurarsi che quanto definito dal documento sia corretto e congruente con altri eventuali documenti.
Infine il verificatore può segnalare elementi da aggiungere nel $\textit{Piano di Progetto}_G$ che possono tornare utili agli altri verificatori. \\
\subsubsection{Verifica dei documenti}
Per completare la verifica di un documento, tranne il glossario, si procede nel seguente modo:
\begin{itemize}
    \item Verificare il documento su overleaf
    \item Controllare eventuali $\textit{riferimenti}_G$ al glossario (si ricorda che le parole del glossario sono case sensitive)
    \item Se il documento viene approvato allora bisogna segnare come "verificato" sia il $\textit{ticket}_G$ relativo alla verifica del documento sia il $\textit{ticket}_G$ della stesura del documento presenti su jira
\end{itemize}
Per la verifica del glossario si faccia riferimento alla sezione 3.1.12 .
\subsubsection{Verifica del codice}
Il verificatore \emph{NON} deve spostare i $\textit{ticket}_G$ in completato che riguardano ai programmatori in quanto è compito del responsabile una volta approvata la $\textit{pull request}_G$.
\subsection{Programmatore}
Il Programmatore è incaricato della $\textit{codifica}_G$ del progetto e delle componenti di supporto che verranno utilizzate per eseguire prove di verifica e di validazione del prodotto. \\
In particolare, si occupa di:
\begin{itemize}
    \item Implementare la "$\textit{specifica tecnica}_G$" redatta dal Progettista;
    \item Scrivere un codice pulito e facilmente mantenibile che rispetti le norme definite nel presente documento;
    \item Realizzare gli strumenti per la verifica e la validazione del $\textit{software}_G$;
    \item Redigere il "$\textit{Manuale Utente}_G$" relativo alla propria $\textit{codifica}_G$;
    \item Redigere i verbali delle riunioni interne ed esterne del team.
\end{itemize}
\subsubsection{Gestione del repository}
Inoltre, di seguito viene specificato come gestire la $\textit{repository}_G$ per l'implementazione di nuove $\textit{feature}_G$:
\begin{enumerate}
    \item Tramite $\textit{jira}_G$ creare un nuovo $\textit{branch}_G$ associato al $\textit{ticket}_G$, come viene descritto nella sezione gestione del repository
    \item Eseguire il $\textit{push}_G$ una volta completata l'attività
    \item Creare una $\textit{pull request}_G$ dal nuovo $\textit{branch}_G$ verso il $\textit{branch}_G$ develop (tramite jira/github)
    \item Una volta eseguito si sposta il $\textit{ticket}_G$ nella sezione "da verificare"
\end{enumerate}
\subsection{Gestione di progetto}
\subsubsection{Allineamento organizzativo}
Per l'allineamento e il coordinamento delle attività sono stati scelti due tipi di comunicazione:
\begin{itemize}
    \item Interne: coinvolge tutti i membri del gruppo;
    \item Esterne: coinvolge proponenti e committenti.
\end{itemize}

\subsubsection{Comunicazioni interne}
Le comunicazioni interne avvengono tramite le applicazioni:
\begin{itemize}
    \item \textit{Telegram}: permette una comunicazione veloce tra i membri del gruppo e viene utilizzato principalmente per organizzare incontri interni o discutere di eventuali quesiti.
    \item \textit{Discord}: viene utilizzato dai membri del gruppo per tenere incontri interni, in quanto dispone di un canale vocale; inoltre, permette la condivisione di contenuti multimediali in streaming video.
\end{itemize}

\subsubsection{Comunicazioni esterne}
Le comunicazioni esterne vengono gestite dal Responsabile di Progetto tramite l’utilizzo dei seguenti
canali:
\begin{itemize}
    \item  $\textit{Posta elettronica}_G$: tramite l’indirizzo e-mail del gruppo \textit{ramtastic6@gmail.com}.
    \item  \textit{Telegram}: permette di instaurare un canale veloce di comunicazione con la proponente.
\end{itemize}

\subsection{Gestione organizzazione del lavoro}
\subsubsection{Modello di sviluppo}
Il gruppo, al fine di minimizzare ritardi e massimizzare lo svolgimento delle proprie attività, ha deciso
di implementare il $\textit{framework}_G$ \textit{Scrum}, che permette di dividere il tempo di lavoro in intervalli piccoli, frammentati all’interno di un periodo di circa due settimane, definito come "\textit{sprint}". In particolare, all’interno di
questo, vengono svolte un numero di attività commisurate che devono essere completate entro i suoi termini. Individuiamo dunque una serie di fasi:
\begin{itemize}
    \item \textbf{sprint planning}: in questa fase, si discutono gli obbiettivi da raggiungere all'interno dello \textit{sprint}; vengono quindi pianificate le attivitá da svolgere durante lo \textit{sprint} in funzione di questi, facendo attenzione alle disponibilitá di ogni membro e preparando un $\textit{preventivo}_G$ per il periodo; è in questa fase, inoltre, che viene svolto il cambio dei ruoli.
    Lo \textit{sprint planning} consiste in una riunione tenuta tramite \textit{Discord} che si svolge all'inizio dello \textit{sprint} e richiede la partecipazione di tutti i membri del gruppo.
    
    \item \textbf{sprint review}: in questa fase, si discutono gli obiettivi da raggiunti nello \textit{sprint}. Alla fine, dovrebbe essere prodotto almeno un incremento, cioè un $\textit{software}_G$ utilizzabile. In particolare, in questa fase vengono definiti a $\textit{consuntivo}_G$ le risorse impiegate commisurate agli obiettivi, questi ultimi suddivisi in raggiunti e non raggiunti, capendo cosa va migliorato e cosa non è stato fatto per poter aggiungere ulteriori obiettivi per lo \textit{sprint} successivo;
    
    \item \textbf{sprint retrospective}: fase che conclude definitivamente lo \textit{sprint} appena svolto, avente l'obiettivo di valutarne l'andamento generale e cercando di capire cosa è stato fatto bene e cosa può essere migliorato. In questo modo, è meglio definire come ripianificare le attività, decidendo come iniziare/continuare/concludere attività presenti o future da realizzare.
\end{itemize}

\subsubsection{Rotazione dei ruoli}
\'E prevista una rotazione dei ruoli all'interno del gruppo a cadenza periodica.
L’attribuzione dei ruoli viene svolta secondo i seguenti criteri:
\begin{itemize}
    \item Equità;
    \item Disponibilità;
    \item Assenza di conflitti.
\end{itemize}


\subsubsection{Gestione delle attività}
Per gestire le attività da istanziare all'interno del $\textit{framework}_G$ \textit{Scrum}, è stato deciso di utilizzare \textit{Jira}. Un'attività, definita come \textit{ticket} all'interno di \textit{Jira}, ha un ciclo di vita; esso si compone di quattro stati:

\begin{itemize}
    \item \textbf{Da completare}: l'attività è stata istanziata e assegnata ad un membro del gruppo; è stata inoltre definita una stima in termini temporali associata ad essa.
    \item \textbf{In corso}: l'attività è entrata nella fase di svolgimento; all'interno di questa fase, il membro al quale è stata assegnata ha il compito di dover inserire le informazioni di tracciamento temporale associate al \textit{ticket}; per fare ciò, basta selezionare un \textit{ticket} e andare a modificare il campo \textit{Tracciamento temporale}, inserendo il tempo impiegato fino ad ora per svolgere l'attività.
    \item \textbf{In fase di verifica}: durante questa fase, il Verificatore si assicura che l'attività sia stata svolta correttamente e secondo dei criteri di qualità attesa (al momento, ancora da specificare).
    \item \textbf{Completata}: l'attività è stata verificata e approvata dal Responsabile di Progetto.
\end{itemize}
Il passaggio di un \textit{ticket} da uno stato all'altro è responsabilità del membro al quale è stato assegnato, eccetto per lo stato di completamento; infatti, spetta al Responsabile di Progetto decidere quando un'attività risulti o meno completata.

\subsection{Infrastruttura}
Strumenti per la gestione delle attività di progetto e procedure legate ad essi.
\subsection{Miglioramento}
Durante lo svolgimento delle attività e successiva stesura dei documenti, il gruppo si impegna ad operare secondo il \textit{principio di miglioramento continuo}, al fine di individuare facilmente attività, ruoli e possibili miglioramenti, cercando nuove o diverse soluzioni alle problematiche insorte.
\subsection{Formazione}
Al fine di operare continuativamente un miglioramento sulle attività svolte e proseguire nel mantenimento corretto delle attività in modo asincrono, occorre da parte di tutti i membri del gruppo lo studio in autonomia delle $\textit{tecnologie}_G$ e delle modalità operative presenti, al fine di velocizzare l'apprendimento degli strumenti utilizzati. Si listano, al fine di
completezza, alcune documentazioni utilizzate durante lo sviluppo, sia a livello documentale che organizzativo:

\begin{itemize}
    \item \textit{Github}: https://docs.github.com/
    \item \textit{Jira}: https://confluence.atlassian.com/jira
    \item \textit{Git}: https://docs.github.com/en/get-started/using-git/about-git
    \item \textit{Scrum}: https://scrumguides.org/scrum-guide.html
\end{itemize}