\documentclass[12pt, oneside]{article} 
\usepackage{amsmath, amsthm, amssymb, calrsfs, wasysym, verbatim, bbm, color, graphicx, geometry, fancyhdr, url, multirow, hyperref}
\usepackage[italian]{babel}

\geometry{tmargin=.75in, bmargin=.75in, lmargin=.75in, rmargin = .75in}


\author{RAMtastic6}

%Intestazione
\pagestyle{fancy}
\fancyhf{}
\fancyhead[R]{Gruppo 14 RAMtastic6\\ramtastic6@gmail.com}
\fancyfoot[C]{\thepage}

% Linea intestazione
\renewcommand{\headrulewidth}{0pt} 

% Intestazione documento
\begin{document}
% Salta la prima pagina per l'intestazione
\thispagestyle{empty}
\title{Norme Di Progetto}
\maketitle
\begin{figure}[h]
  \centering
  \includegraphics[scale=0.3]{logo.png}
\end{figure}
\begin{center}
    email: ramtastic6@gmail.com
\end{center}

% Informazioni sul documento
\section*{Informazioni sul documento}
\begin{tabular}{ll}
Versione: & 0.8.0 \\
Redattori: &  Visentin S. Basso L. Tonietto F. Zambon M. \\
Verificatori: & Tonietto F. Brotto D. Zambon M. Basso L. Zaupa R. \\
Destinatari: & T. Vardanega, R. Cardin, $\textit{Imola Informatica}_G$ \\
Uso: & Interno
\end{tabular}
\newpage

% Registro dei cambiamenti
\section*{Registro dei Cambiamenti - Changelog}
\begin{tabular}{|c|c|c|p{3cm}|p{6cm}|}
\hline
\textbf{Versione} & \textbf{Data} & \textbf{Autore} & \textbf{Verificatore} & \textbf{Dettaglio} \\
\hline
v.0.8.0 & 2024-03-28 & Zambon M. & Zaupa R. & Modifica della sottosezione 3.1.12 (Glossario tecnico): ridefinita la parte di redazione del glossario e definita la sua verifica; stesura sottosezione 1.1 (scopo del documento) e 1.2 (scopo del prodotto)\\
\hline
v.0.7.2 & 2024-03-27 & Zambon M. & Zaupa R. & Stesura della sottosezione 3.4 (Verifica)\\
\hline
v.0.7.1 & 2024-03-26 & Zambon M. & Zaupa R. & Stesura della sottosezione 3.3 (Gestione della qualità)\\
\hline
v.0.7.0 & 2024-03-21 & Basso L. & Brotto D. & Stesura delle sottosezioni 3.1.2, 3.1.3 (relative ai $\textit{repository}_G$ per la documentazione) e 3.1.9, 3.1.11, 3.1.12 (relative alle procedure per produrre documentazione)\\
\hline

v.0.6.0 & 2024-01-18 & Basso L. & Tonietto F. & Stesura delle sezione 4 (processi organizzativi) con relative sottosezioni. \\
\hline
v.0.5.0 & 2024-01-09 & Zambon M. & Tonietto F. & Stesura delle sezione 2.4.2 (Analisi dei requisiti)\\
\hline
v.0.4.2 & 2023-12-12 & Visentin S. & Tonietto F. & Prima stesura della sezione 2 Processi primari \\
\hline
v.0.4.1 & 2023-12-09 & Visentin S. & Tonietto F. & Piccole modifiche nella sezione 4. \\
\hline
v.0.4.0 & 2023-11-19 & Tonietto F. & Zaupa R. & Stesura della sotto-sottosezione 3.2.4 (relativa al controllo del flusso, con approfondimento sul flusso della documentazione prodotta) e della sottosezione 4.1 (breve analisi e descrizioni dei ruoli di progetto). \\
\hline 
v.0.3.0 & 2023-11-12 & Basso L. & \multirow{3}{3cm}{Zambon M. \\  Visentin S. \\ Zaupa R.} & Stesura della sottosezione 3.2 della sezione relativa alla documentazione e modifiche relative al ciclo di vita di un documento (sottosezione 3.1, parte 3.1.3). \\
\hline
v.0.2.0 & 2023-11-12 & Basso L. & Zaupa R. & Stesura della sezione 3.1 (Documentazione) e delle sottosezioni relative ad essa \\
\hline
v.0.1.0 & 2023-10-30 & Visentin S. & Tonietto F. & Prima versione \\
\hline
\end{tabular}
\newpage

% Sommario
\tableofcontents
\newpage
\section{Introduzione}
\section{Introduzione}
\subsection{Scopo del documento}
Il presente documento si propone di definire le metriche e le metodologie di controllo e misurazione necessarie per garantire la qualità del prodotto e del $\textit{processo}_G$. In particolare, le metriche di valutazione del prodotto sono correlate ai requisiti e alle aspettative del fornitore.
Il $\textit{Piano di Qualifica}_G$ è concepito per essere dinamico ed incrementale, in particolar modo per quanto riguarda le metriche descritte e mira a fornire una valutazione il più obiettiva possibile di ciò che è stato realizzato.\\
Le procedure del $\textit{Way Of Working}_G$ devono essere costantemente osservate e migliorate, al fine di garantire che il prodotto soddisfi le aspettative del cliente e mantenga gli standard di qualità richiesti. Eventuali termini tecnici sono definiti all'interno del documento "Glossario Tecnico".
\subsection{Scopo del prodotto}
Il prodotto finale, realizzato tramite un'$\textit{Applicazione Web Responsive}_G$, si propone di realizzare un $\textit{software}_G$ innovativo volto a semplificate il $\textit{processo}_G$ di $\textit{prenotazione}_G$ e $\textit{ordinazione}_G$ nei ristoranti, contribuendo a migliorare l'esperienza per clienti e ristoratori. In particolare, \textit{Easy Meal} dovrà consentire agli utenti di personalizzare gli ordini in base alle proprie preferenze, allergie ed esigenze alimentari; interagire direttamente con lo staff del ristorante attraverso una chat integrata e in ultimo, consentire di dividere il conto tra i partecipanti al tavolo.
\subsection{Riferimenti}
\subsubsection{Riferimenti normativi}
\begin{enumerate}
    \item $\textit{Norme di Progetto}_G$ v2.0.0
    \item Presentazione del $\textit{capitolato}_G$ d'appalto C3 - Progetto $\textit{Easy Meal}_G$: \\ 
    \url{https://www.math.unipd.it/~tullio/IS-1/2023/Progetto/C3.pdf}
    \item Regolamento del progetto didattico: \\ 
    \url{https://www.math.unipd.it/~tullio/IS-1/2023/Dispense/PD2.pdf}
\end{enumerate}
\subsubsection{Riferimenti informativi}
\label{sec:rif_inf}
\begin{enumerate}
    \item Lezione \emph{"Progettazione $\textit{software}_G$ (T6)"} del corso di $\textit{Ingegneria del $\textit{software}_G$}_G$: \\
    \url{https://www.math.unipd.it/~tullio/IS-1/2023/Dispense/T6.pdf}
    \item Lezione \emph{"Qualità del $\textit{software}_G$ (T7)"} del corso di $\textit{Ingegneria del $\textit{software}_G$}_G$: \\
    \url{https://www.math.unipd.it/~tullio/IS-1/2023/Dispense/T7.pdf}
    \item Lezione \emph{"Qualità di $\textit{processo}_G$ (T8)"} del corso di $\textit{Ingegneria del software}_G$: \\
    \url{https://www.math.unipd.it/~tullio/IS-1/2023/Dispense/T8.pdf}
    \item Lezione \emph{"Verifica e validazione: introduzione (T9)"} del corso di $\textit{Ingegneria del software}_G$: \\
    \url{https://www.math.unipd.it/~tullio/IS-1/2023/Dispense/T9.pdf}
    \item Lezione \emph{"Verifica e validazione: analisi statica (T10)"} del corso di $\textit{Ingegneria del software}_G$: \\
    \url{https://www.math.unipd.it/~tullio/IS-1/2023/Dispense/T10.pdf}
    \item Lezione \emph{"Verifica e validazione: analisi dinamica (T11)"} del corso di $\textit{Ingegneria del software}_G$: \\
    \url{https://www.math.unipd.it/~tullio/IS-1/2023/Dispense/T11.pdf}
     \item Documento \emph{"Dichiarazione impegni v1.2"}: \\ \url{https://github.com/RAMtastic6/Project14/blob/main/documenti/CANDIDATURA/documento_impegni_v1.2.pdf}
     \item Metriche di progetto (\emph{Earned Value Analysis}):\\
     \url{https://it.wikipedia.org/wiki/Metriche_di_progetto}
     \item Glossario v2.0.0;
     \item Analisi dei Requisiti v3.0.0.
\end{enumerate}
\subsection{Codifica delle metriche}
In questa sottosezione verranno definite le metriche che utilizzeremo, utilizzando un codice standardizzato.

Una metrica è identificata dal seguente formato di codice:
\[
\text{M[Tipo][Id]-[Acronimo]}
\]

Dove:
\begin{itemize}
    \item \textbf{M} sta per "Metrica"
    \item \textbf{Tipo} può essere PC (per un $\textit{processo}_G$) o PD (per un prodotto)
    \item \textbf{Id} rappresenta un identificativo all'interno di una metrica di un certo tipo
    \item \textbf{Acronimo} indica l'acronimo del nome della metrica utilizzata
\end{itemize}

Per ciascuna metrica verranno fornite descrizioni, valori accettabili e valori preferibili.
\subsection{Codifica dei test}
In questa sottosezione verranno definiti i $\textit{test}_G$ che utilizzeremo, utilizzando un codice standardizzato.

Un $\textit{test}_G$ è identificato dal seguente formato di codice:
\[
\text{T[Tipo]-[Id]}
\]

Dove:
\begin{itemize}
    \item \textbf{T} sta per "$\textit{Test}_G$"
    \item \textbf{Tipo} può essere S (di $\textit{sistema}_G$) o I (di integrazione) o U (di unità) oppure A (di accettazione)
    \item \textbf{Id} rappresenta un identificativo all'interno di un $\textit{test}_G$ di un certo tipo
\end{itemize}

Per ciascun $\textit{test}_G$ verranno fornite descrizioni e il loro stato se implementato o meno oltre che un loro tracciamento.
\newpage

\section{Processi primari}
\subsection{Descrizione}
Con la locuzione "processi primari" si intendono "tutti i processi che hanno come clienti soggetti esterni al gruppo". \\
\subsection{Acquisizione}
Nel $\textit{processo}_G$ di acquisizione avviene la raccolta e la comprensione dei requisiti con lo scopo di identificare un $\textit{capitolato}_G$ adeguato per il gruppo da proporre per la $\textit{candidatura}_G$.
\subsubsection{Valutazione capitolati}
Il gruppo ha analizzato la proposta dei proponenti valutandone, in base all'esperienza del gruppo, la loro \textit{complessità}; tale analisi è stata determinante per la scelta definitiva.
\subsubsection{Appalto capitolati}
Il gruppo si è proposto per il $\textit{capitolato}_G$ C3 dell'azienda $\textit{Imola Informatica}_G$. Nonostante un  un primo riscontro negativo, a seguito di modifiche volte a sistemare lacune presenti  presenti nella $\textit{candidatura}_G$ iniziale, il gruppo è riuscito ad aggiudicarsi il $\textit{capitolato}_G$.

\subsection{Fornitura}
Il $\textit{processo}_G$ di $\textit{fornitura}_G$ consiste nel chiarire ogni dubbio legato al prodotto finale che il proponente desidera; in modo da evitare incomprensioni durante lo svolgimento del progetto, il gruppo RAMtastic6 si impegna a comunicare con l'azienda, in modo da raggiungere i seguenti obiettivi:
\begin{enumerate}
    \item Determinare i requisiti da soddisfare nel prodotto finale;
    \item Ottenere incontri di formazioni sulle $\textit{tecnologie}_G$ e strumenti consigliati dall'azienda per realizzare il prodotto;
    \item Ricevere $\textit{feedback}_G$ in fase di sviluppo su quanto precedentemente svolto.
\end{enumerate}

\subsubsection{Lettera di Presentazione}
La \textit{Lettera di Presentazione} è il documento che fornisce la presentazione della documentazione e del prodotto durante le varie fasi di revisione di progetto. 
\\
Questo documento riporta l’insieme della documentazione richiesta che sarà consegnata ai committenti, il Prof. Tullio Vardanega e il Prof. Riccardo Cardin.

\subsubsection{Analisi dei Requisiti v3.0.0}
L'\textit{Analisi dei Requisiti} stabilisce una comprensione comune tra gli \textit{stakeholder} riguardo alle caratteristiche e alle prestazioni del $\textit{software}_G$ da sviluppare. Questo documento elimina le ambiguità che potrebbero sorgere durante la comunicazione e fornisce una base chiara per il progetto.
\\
Il documento di \textit{Analisi dei Requisiti} è strutturato in diverse sezioni chiave:
\begin{itemize}
\item \textbf{Introduzione}: Fornisce una panoramica generale del prodotto e delle sue funzionalità principali. Questa sezione stabilisce il contesto per il resto del documento e fornisce una visione d'insieme del progetto.

\item \textbf{Casi d'uso}: Identifica e descrive tutti i possibili scenari di utilizzo del $\textit{software}_G$ da parte degli utenti. Ogni $\textit{caso d'uso}_G$ descrive un'interazione specifica tra l'utente e il $\textit{sistema}_G$, illustrando come il $\textit{software}_G$ dovrebbe essere utilizzato per soddisfare le esigenze degli utenti.

\item \textbf{Requisiti}: Raccoglie tutte le richieste e i vincoli definiti dal cliente o emersi durante le discussioni con il team di sviluppo. Ogni $\textit{requisito}_G$ è accompagnato da una descrizione dettagliata che specifica ciò che il $\textit{sistema}_G$ deve fare o quale comportamento deve avere. Questi requisiti costituiscono la base per la progettazione e lo sviluppo del $\textit{software}_G$.
\end{itemize}

\subsubsection{Specifica Tecnica v1.0.0}
Il documento \textit{Specifica Tecnica} descrive in modo dettagliato le scelte progettuali effettuate dal gruppo per la realizzazione del $\textit{sistema}_G$ richiesto dal proponente. 
\\Viene compresa l’architettura logica e l’architettura di \textit{deployment} oltre che la lista
delle $\textit{tecnologie}_G$ utilizzate e i $\textit{design}_G$ pattern adottati.
Inoltre, viene fornita una sezione relativa al tracciamento dei requisiti soddisfatti in linea con il documento di \textit{Analisi dei Requisiti}.

\subsubsection{Manuale Utente v1.0.0}
Il \textit{Manuale Utente} ha lo scopo di fornire all'utilizzatore del prodotto un orientamento verso quelli che sono tutti gli scenari in cui potrà navigare l'utente una volta messo in atto. \\
Comprende una guida di navigazione completa per eliminare tutti i dubbi che potrebbe avere durante l'utilizzo di tutte le funzionalità che il prodotto ha da offrire.

\subsubsection{Glossario v2.0.0}
Il \textit{Glossario} è il documento alla base di ogni altro documento di supporto. Il suo scopo è quello di esplicitare il significato dei termini comunemente usati all'interno di tutta la documentazione prodotta, al fine di evitare ambiguità o incomprensioni. \\
E' rivolto sia ai componenti del gruppo che ai committenti e ai proponenti. 

\subsubsection{Piano di Progetto v2.0.0}
Il documento \textit{Piano di Progetto} costituisce uno $\textit{strumento}_G$ di pianificazione per tutte le attività che dovranno essere svolte così da poter rispettare la data di consegna del progetto.
Nel dettaglio il $\textit{Piano di Progetto}_G$ è composto da:
\begin{itemize}
    \item $\textit{Analisi Dei Rischi}_G$: analisi delle difficoltà che il gruppo potrebbe riscontrare durante lo svolgimento del progetto, in particolar modo a livello organizzativo e tecnologico;
    \item Modello di sviluppo;
    \item Pianificazione;
    \item $\textit{Preventivo}_G$: che rispecchi quanto comunicato in fase di $\textit{candidatura}_G$;
    \item $\textit{Consuntivo}_G$: tracciamento dell'andamento del gruppo rispetto al $\textit{preventivo}_G$ fatto.
\end{itemize}
\subsubsection{Piano di Qualifica v2.0.0}
Nel documento \textit{Piano di Qualifica} vengono elencate tutte le attività svolte dal verificatore con l'obiettivo di garantire la qualità del prodotto finale.
E' formato dai seguenti componenti:
\begin{itemize}
    \item Qualità di processo
    \item Qualità di prodotto
    \item $\textit{Test}_G$: eseguiti sul prodotto che assicurano che i requisiti siano stati rispettati
    \item Resoconto
\end{itemize}
\subsubsection{Rilascio}
Quando il prodotto verrà ultimato, verrà collaudato per garantire il suo corretto funzionamento. Se il prodotto supera il collaudo, verrà consegnato al committente insieme alla documentazione del progetto. 
Il gruppo non effettuerà $\textit{manutenzione}_G$ una volta rilasciato il prodotto.
\subsubsection{Strumenti}
Per il $\textit{processo}_G$ di $\textit{fornitura}_G$ vengono utilizzati i seguenti strumenti:
\begin{itemize}
    \item \emph{\textbf{Microsoft Teams}}: servizio di videoconferenza utilizzato per gli incontri con il proponente;
    \item \emph{\textbf{Telegram}}: servizio di messaggistica utilizzato per le comunicazioni con il proponente;
    \item \emph{\textbf{Share Point}}: servizio che permette di condividere e gestire contenuti Office tramite browser, utilizzato per creare le presentazioni per i diari di bordo.
\end{itemize}

\subsection{Sviluppo}
Lo scopo del $\textit{processo}_G$ di sviluppo è quello di dichiarare le attività da svolgere per raggiungere i requisiti necessari del prodotto.
A tale scopo, il gruppo si dividerà in diversi ruoli, i quali avranno compiti precisi da svolgere.
\subsubsection{Analisi Dei Requisiti}
\textit{Analisi Dei Requisiti} è un documento fondamentale per lo sviluppo. Infatti, tale documento deve indicare i requisiti necessari del prodotto finale, che a loro volta andranno a rispecchiare le aspettative del proponente. \\
Inoltre, questo documento servirà anche come documentazione del prodotto, andandone infatti a contenere tutte le relative Funzionalità.
\subsubsubsection{Attori} 
Innanzitutto, verranno definiti gli attori e una panoramica dei vari casi d'uso a loro associati tramite un diagramma
\subsubsubsection{Casi d'uso}
Successivamente, verranno descritti i vari casi d'uso, i quali hanno il compito di rappresentare le Funzionalità che il prodotto finale dovrà rispettare. \\
I casi d'uso saranno ordinati per $\textit{attore}_G$, ovvero verranno prima descritti tutti i casi d'uso associati a un particolare $\textit{attore}_G$, per poi proseguire con la descrizione di tutti i casi d'uso riguardanti l'$\textit{attore}_G$ successivo, e a proseguire. \\
Ogni Caso d'Uso avrà un diagramma ad esso associato e la sua descrizione $\textit{verbale}_G$. Gli use case cosiddetti "atomici" saranno inseriti nello $\textit{scenario}_G$ principale, il quale rappresenterà le azioni compiute in modo consecutivo dall'$\textit{attore}_G$ nello $\textit{scenario}_G$ più comune. \\
La descrizione $\textit{verbale}_G$ di ogni $\textit{caso d'uso}_G$ seguirà la seguente struttura standard:
%I casi d'uso hanno il ruolo di rappresentare le Funzionalità che il prodotto finale deve rispattare, quindi hanno una struttura standard da rispettare:
\begin{itemize}
    \item Attori
    \item Precondizioni
    \item Postcondizioni
    %\item $\textit{Scenario}_G$ primario
    %\item $\textit{Scenario}_G$ secondario (opzionale)
    \item $\textit{Scenario}_G$ primario
    \item Scenari alternativi (opzionale)
\end{itemize}
\begin{comment}
I $\textit{sottocasi d'uso}_G$ non verranno descritti individualmente, poiché sarà già tutto descritto a livello atomico nello $\textit{Scenario}_G$ principale del relativo Caso d'Uso "padre".
\end{comment}
Gli scenari alternativi saranno presenti solo se il Caso d'Uso può generare eccezioni: poiché a più eccezioni corrispondono una singola modalità di gestione delle stesse, per ogni $\textit{scenario}_G$ alternativo ci potranno essere più primi punti, i quali verranno rappresentati nel formato "\textit{1.xa}", nel quale '\textit{x}' rappresenta il punto dello $\textit{scenario}_G$ principale dal quale viene generata l'$\textit{eccezione}_G$, mentre '\textit{a}' rappresenta il tipo di $\textit{eccezione}_G$. Verrà poi descritto dal punto "\textit{2}" a seguire lo $\textit{scenario}_G$ alternativo.
\subsubsubsection{Requisiti} 
Verranno infine descritti i requisiti, che potranno essere di tre tipi: 
\begin{itemize}
    \item \textbf{Funzionali}: requisiti che delineano le varie funzionalità del $\textit{sistema}_G$, ovvero le azioni eseguibili dallo stesso e le informazioni che il $\textit{sistema}_G$ è in grado di fornire; 
    \item \textbf{Di qualità}: requisiti che delineano le specifiche qualitative che devono essere rispettate al fine di garantire la qualità del $\textit{sistema}_G$; 
    \item \textbf{Requisiti di vincolo}: requisiti che delineano i limiti e le restrizioni di cui il $\textit{sistema}_G$ deve tener conto per adempiere alle esigenze del proponente. 
\end{itemize}
L'elenco di tutti i tipi di requisiti presenterà la seguente struttura: 
\begin{itemize}
    \item \textit{Codice}: un codice identificativo univoco per ogni $\textit{requisito}_G$; 
    \item \textit{Descrizione}: una descrizione $\textit{verbale}_G$ del $\textit{requisito}_G$; 
    \item \textit{Fonte}: l'elemento che ha generato la necessità di produrre il $\textit{requisito}_G$ preso in esame; può essere uno o più casi d'uso, come pure direttamente il testo del $\textit{capitolato}_G$ d'appalto. 
\end{itemize}
\subsubsubsection{Strumenti}
\begin{itemize}
    \item \emph{\textbf{LucidChart}}: applicazione $\textit{software}_G$ utilizzata dal team per la realizzazione dei diagrammi dei casi d'uso.
\end{itemize}
\subsubsection{Progettazione}
La progettazione, a carico della figura del progettista, definisce la struttura del progetto basandosi sull'\textit{Analisi Dei Requisiti}.
La progettazione avviene su più livelli:
\begin{enumerate}
    \item $\textit{Progettazione architetturale}_G$: dove viene scelta la struttura del sistema
    \item $\textit{Design}_G$: ovvero il design dell'interfaccia che deve avere il prodotto
    \item $\textit{Progettazione dettagliata}_G$: le specifiche dei componenti del prodotto che comprendono le specifiche architetturali, i diagrammi delle classi e i $\textit{test}_G$ d'unità
\end{enumerate}

\subsubsubsection{Obiettivi}
Nella fase di progettazione di un prodotto $\textit{software}_G$, l'obiettivo principale è garantire che i requisiti siano soddisfatti attraverso un $\textit{sistema}_G$ di qualità definito dall'$\textit{architettura}_G$ del prodotto. 
\\È essenziale organizzare il $\textit{sistema}_G$ in modo da facilitare futuri adattamenti e gestire la complessità mediante una $\textit{progettazione dettagliata}_G$ che suddivide il $\textit{sistema}_G$ in unità architetturali, rendendo più semplice la $\textit{codifica}_G$ di ogni parte e assicurando che sia gestibile, veloce e verificabile.
\\Il team di progettazione inizia con un'analisi approfondita per selezionare le $\textit{tecnologie}_G$ più appropriate, valutandone i vantaggi, le debolezze e le potenziali criticità. 
\\Una volta scelte le $\textit{tecnologie}_G$, si sviluppa un'$\textit{architettura}_G$ di alto livello per delineare la struttura generale del prodotto, creando una base solida per la realizzazione di un \textit{Proof of Concept} ($\textit{PoC}_G$). 
\\Questa $\textit{architettura}_G$ fornisce una visione d'insieme del $\textit{sistema}_G$, identificando i principali componenti, i flussi di dati e le loro interazioni, con un'attenzione particolare alla flessibilità per future modifiche. 
\\Il $\textit{PoC}_G$ serve a valutare le decisioni architetturali e tecnologiche e a verificarne la conformità agli obiettivi e alle specifiche del progetto. Dopo lo sviluppo e l'analisi del $\textit{PoC}_G$, si procede con iterazioni successive per migliorare, aggiustare e completare il $\textit{design}_G$, fino a ottenere un \textit{Minimum Vaulable Product} ($\textit{MVP}_G$), che rappresenta una versione funzionale ed essenziale del prodotto, integrata nella \textit{Product Baseline}.

\subsubsubsection{Documentazione}
\textbf{Specifica Tecnica}
\\
Il documento di \textit{Specifica Tecnica} descrive dettagliatamente il $\textit{design}_G$ finale del prodotto e fornisce istruzioni precise agli sviluppatori per guidarli nella corretta implementazione della soluzione $\textit{software}_G$ in linea con i requisiti e le specifiche indicate. Questo documento è fondamentale per ridurre la complessità e le ambiguità nel $\textit{processo}_G$ di sviluppo, assicurando che il prodotto finale soddisfi le aspettative del cliente e funzioni in modo ottimale.
\\
Il documento di \textit{Specifica Tecnica} include vari elementi essenziali:
\begin{itemize}
\item $\textit{Architettura}_G$ logica: definisce i componenti, i ruoli, le connessioni e le interazioni all'interno del $\textit{sistema}_G$.
\item $\textit{Architettura}_G$ di \textit{deployment}: descrive come i componenti architetturali vengono allocati e distribuiti nel $\textit{sistema}_G$ in esecuzione.
\item $\textit{Design}_G$ pattern: spiega i design pattern architetturali adottati e quelli influenzati dalle $\textit{tecnologie}_G$ utilizzate.
\item Procedure di \textit{testing}: indica i processi per testare e verificare che il $\textit{software}_G$ soddisfi i requisiti specificati.
\item Requisiti tecnici: dettaglia i requisiti prestazionali, di sicurezza, di scalabilità e di compatibilità con determinate piattaforme che il $\textit{software}_G$ deve soddisfare.
\end{itemize}
Attraverso una documentazione accurata e completa, la \textit{Specifica Tecnica} funge da riferimento chiave per il team di sviluppo, facilitando una comprensione chiara e condivisa del progetto e contribuendo a un prodotto finale di alta qualità.

\subsubsubsection{Qualità dell'architettura}
La qualità dell'$\textit{architettura}_G$ di un $\textit{software}_G$ è fondamentale per assicurare che il $\textit{sistema}_G$ non solo soddisfi i requisiti funzionali e non funzionali, ma che lo faccia in maniera efficiente e sostenibile. Gli aspetti chiave da considerare includono:
\begin{itemize}
\item \textbf{Adeguatezza}: l'$\textit{architettura}_G$ deve rispondere ai requisiti funzionali e non funzionali definiti, evitando sia sovra-dimensionamenti che sotto-dimensionamenti.
\item \textbf{Chiarezza}: l'$\textit{architettura}_G$ deve essere facilmente comprensibile, permettendo a sviluppatori e \textit{stakeholder} di afferrare rapidamente il funzionamento del $\textit{sistema}_G$.
\item \textbf{Modularità}: l'$\textit{architettura}_G$ deve essere suddivisa in moduli o componenti ben definiti, facilitando la separazione delle responsabilità, la $\textit{manutenzione}_G$, l'aggiornamento e lo sviluppo parallelo.
\item \textbf{Disponibilità}: il $\textit{sistema}_G$ deve essere accessibile e operativo quando richiesto, minimizzando i tempi di inattività non pianificati.
\item \textbf{Flessibilità}: l'$\textit{architettura}_G$ deve permettere adattamenti o estensioni per nuovi requisiti o cambiamenti, senza necessitare di modifiche radicali.
\item \textbf{Semplicità}: l'$\textit{architettura}_G$ deve essere il più semplice possibile, senza compromettere funzionalità o efficacia.
\item \textbf{Efficienza}: il $\textit{sistema}_G$ deve utilizzare in modo ottimale le risorse disponibili, come memoria e CPU, eseguendo le sue funzioni nel minor tempo possibile.
\item \textbf{Basso accoppiamento}: le interazioni o dipendenze tra i diversi moduli o componenti devono essere minimizzate per migliorare la manutenibilità e la flessibilità del $\textit{sistema}_G$.
\item \textbf{Robustezza}: il $\textit{sistema}_G$ deve essere in grado di gestire situazioni anomale o errori senza causare gravi interruzioni o perdite di dati, mantenendo un funzionamento accettabile.
\item \textbf{Sicurezza operativa (Safety)}: deve prevenire danni fisici o danni a persone e beni causati da malfunzionamenti del $\textit{software}_G$.
\item \textbf{Sicurezza contro intrusioni (Security)}: il $\textit{software}_G$ deve essere protetto da accessi non autorizzati, manipolazioni e intrusioni esterne.
\item \textbf{Riusabilità}: i componenti del $\textit{software}_G$ devono poter essere riutilizzati in contesti diversi, riducendo il carico di sviluppo e migliorando l'efficienza.
\item \textbf{Incapsulamento}: i dettagli implementativi devono essere nascosti all'esterno di un componente, accessibili solo attraverso un'interfaccia definita.
\item \textbf{Coesione}: i componenti all'interno di un modulo devono lavorare insieme per un obiettivo comune, evitando eccessive interdipendenze.
\item \textbf{Affidabilità}: il $\textit{software}_G$ deve funzionare correttamente e in modo coerente nel tempo, garantendo una prestazione prevedibile.
\end{itemize}
Questi aspetti sono cruciali per creare un'$\textit{architettura}_G$ $\textit{software}_G$ robusta, flessibile e manutenibile, in grado di soddisfare le esigenze presenti e future del progetto.

\subsubsubsection{Diagrammi delle classi}
I diagrammi delle classi rappresentano le proprietà e le relazioni dei componenti gli uni con gli altri.
Dei rettangoli rappresentano graficamente le classi, mentre invece, diversi tipi di frecce rappresentano la relazioni tra le classi.
I diversi tipi di relazioni che si incontrano sono:

\begin{itemize}
    \item \textbf{Dipendenza}
    \\La relazione di dipendenza tra due classi, A e B, indica che la classe A dipende dalla classe B per la sua specifica, implementazione o funzionamento. Questo tipo di relazione è rappresentata da una freccia tratteggiata che punta da A (il cliente) verso B (il fornitore). Ad esempio, se la classe A utilizza i metodi o gli attributi della classe B, un cambiamento nella classe B potrebbe influire sul corretto funzionamento della classe A. 
    \begin{figure}[h]
        \centering
        \includegraphics[width=0.5\linewidth]{img/dipendenza.PNG}
        \caption{Diagramma UML della relazione Dipendenza}
    \end{figure}
    
    \newpage
    
    \item \textbf{Composizione}
    \\La relazione di composizione tra due classi, A e B, implica che l'oggetto della classe A è composto da oggetti della classe B, e che l'esistenza degli oggetti della classe B dipende direttamente dall'esistenza degli oggetti della classe A. Questa relazione è rappresentata da una freccia solida con un rombo pieno alla fine, che punta dalla classe A alla classe B. 
    \begin{figure}[h]
        \centering
        \includegraphics[width=0.5\linewidth]{img/composizione.PNG}
        \caption{Diagramma UML della relazione Composizione}
    \end{figure}
    
    
    
    \item \textbf{Aggregazione}
    \\La relazione di aggregazione tra due classi, A e B, implica che l'oggetto della classe A "aggrega" gli oggetti della classe B, ma questi ultimi possono esistere anche al di fuori del contesto della classe A. È rappresentata da una freccia con un rombo vuoto alla fine, che punta dalla classe A alla classe B. 
    \begin{figure}[h]
        \centering
        \includegraphics[width=0.5\linewidth]{img/aggregazione.PNG}
        \caption{Diagramma UML della relazione Aggregazione}
    \end{figure}

    
    \item \textbf{Relazione con interfaccia}
    \\La relazione con interfaccia tra due classi, A e B, è indicata da una linea non direzionata che collega la classe A a un cerchio vuoto che rappresenta l'interfaccia fornita dalla classe B. Questa relazione denota che la classe A dipende dall'interfaccia definita dalla classe B per specificare le sue operazioni, senza specificare l'implementazione concreta di tali operazioni. 
    \begin{figure}[h]
        \centering
        \includegraphics[width=0.5\linewidth]{img/interfaccia.PNG}
        \caption{Diagramma UML della relazione Interfaccia}
    \end{figure}
    
    
\end{itemize}

\subsubsubsection{Design Pattern}
I \textit{design patterns} sono soluzioni consolidate a problemi di progettazione che si presentano in modo ricorrente in diversi contesti. 
\\Questi modelli offrono un approccio riusabile alla progettazione, garantendo qualità nella soluzione e velocità nell'implementazione. 
\\L'adozione di un \textit{design pattern} avviene quando una soluzione si è dimostrata efficace in un contesto specifico, fornendo una guida affidabile per affrontare problemi simili in futuro. 
\\Di solito, vengono fornite dettagliate linee guida sull'applicazione dei pattern, insieme a una rappresentazione grafica e una spiegazione testuale della loro logica e utilità all'interno dell'$\textit{architettura}_G$ complessiva. 
\\Questa documentazione gioca un ruolo essenziale nel favorire la comprensione dell'integrazione del $\textit{design}_G$ pattern nell'$\textit{architettura}_G$ generale e nella prevenzione di errori di progettazione, assicurando coerenza e coesione nel $\textit{sistema}_G$ $\textit{software}_G$.

\subsubsubsection{Metriche}
\begin{longtable}{|>{\centering\arraybackslash}p{4cm}|p{7cm}|}
  \hline
  \rowcolor{gray!30}
  \textbf{Metrica} & \textbf{Nome} \\
  \hline
  \endfirsthead
  
  \rowcolor{gray!10}
    \begin{tabular}[c]{@{}c@{}}
        \textbf{MPD09-BS} \\
    \end{tabular}
  & \texttt{Browser supportati} \\
  \hline
  \rowcolor{gray!10}
    \begin{tabular}[c]{@{}c@{}}
        \textbf{MPD04-TA}
    \end{tabular}
     & \texttt{Tempo di apprendimento} \\
  \hline
  \rowcolor{gray!10}
    \begin{tabular}[c]{@{}c@{}}
        \textbf{MPD05-RO}
    \end{tabular}
     & \texttt{Raggiunta dell’obbiettivo} \\
  \hline
  \rowcolor{gray!10}
    \begin{tabular}[c]{@{}c@{}}
        \textbf{MPD06-EU}
    \end{tabular}
     & \texttt{Errori dell’utente} \\
  \hline
  \rowcolor{gray!10}
    \begin{tabular}[c]{@{}c@{}}
        \textbf{MPD03-TM}
    \end{tabular}
     & \texttt{Tempo di risposta medio} \\
  \hline
  \end{longtable}

\subsubsubsection{Strumenti}
\begin{itemize}
    \item \emph{\textbf{LucidChart}}: applicazione $\textit{software}_G$ utilizzata dal team per la realizzazione dei diagrammi riguardanti l'$\textit{architettura}_G$ logica.
    \item \emph{\textbf{StarUML}}: applicazione $\textit{software}_G$ utilizzata dal team per la realizzazione dei diagrammi delle classi.
\end{itemize}








\subsubsection{Codifica}
I programmatori, dopo l'analisi e la progettazione, implementano le funzionalità che deve avere il prodotto finale basandosi sull'\textit{Analisi Dei Requisiti} e sui documenti di progettazione.\\
Le aspettative riguardanti tale attività nel caso della $\textit{codifica}_G$ dell'$\textit{MVP}_G$ sono le seguenti:
\begin{itemize}
    \item  Le modifiche attuate dal programmatore devono essere coerenti con i requisiti e i casi d'uso presenti nel documento \textit{Analisi dei Requisiti}.
    Nel caso in cui fosse necessario rivedere alcuni sezioni si discute con l'analista su come procedere.
    \item Le modifiche attuate dal programmatore devono essere coerenti con l'$\textit{architettura}_G$ e la progettazione del documento \textit{Specifica Tecnica}. Nel caso in cui fosse necessario rivedere alcuni sezioni si discute con il progettista su come procedere.
    \item Rispetto a quanto implementato si devono inserire $\textit{test}_G$ di unità che assicurano il corretto funzionamento del prodotto.
\end{itemize}

\subsubsubsection{Norme di codifica}
Per la stesura del codice si sono adottate delle norme comuni tra cui:
\begin{itemize}
    \item \textbf{Convenzioni di denominazione}: Usare nomi significativi per variabili, funzioni, classi, ecc. usando opportunamente lettere maiuscole e minuscole per
    identificare le parole chiave.
    
    \item \textbf{Indentazione}: Usa l'indentazione in modo consistente per evidenziare la struttura del codice. 
    
    \item \textbf{Commenti}: Aggiungere commenti significativi per spiegare parti complesse del codice, ma evitare commenti ovvi o ridondanti. 
    
    \item \textbf{Lunghezza delle righe}: Limitare la lunghezza delle righe per una facile lettura e revisione del codice. 
    
    \item \textbf{Struttura del codice}: Organizzare il codice in blocchi logici e funzioni coese. Utilizzare funzioni e classi per evitare la duplicazione del codice.
    
    \item \textbf{Gestione delle eccezioni}: Trattare le eccezioni in modo appropriato, usando blocchi try\-catch quando necessario. Non ignorare mai le eccezioni senza gestirle.

     \item \textbf{Separare $\textit{logica di business}_G$ e logica di interfaccia}: Mantenere separata la $\textit{logica di business}_G$ dal codice di interfaccia utente o di accesso ai dati. Questo rende il codice più modulare e facile da testare.
\end{itemize}

\subsubsubsection{Metriche}
\begin{longtable}{|>{\centering\arraybackslash}p{4cm}|p{7cm}|}
  \hline
  \rowcolor{gray!30}
  \textbf{Metrica} & \textbf{Nome} \\
  \hline
  \endfirsthead
  
  \rowcolor{gray!10}
    \begin{tabular}[c]{@{}c@{}}
        \textbf{MPD08-CC} \\
    \end{tabular}
  & \texttt{Complessità ciclomatica} \\
  \hline
  \rowcolor{gray!10}
    \begin{tabular}[c]{@{}c@{}}
        \textbf{MPD07-FD}
    \end{tabular}
     & \texttt{Failure density} \\
  \hline
  \rowcolor{gray!10}
    \begin{tabular}[c]{@{}c@{}}
        \textbf{MPC15-PCTS}
    \end{tabular}
     & \texttt{Percentuale dei $\textit{test}_G$ superati} \\
  \hline
  \rowcolor{gray!10}
    \begin{tabular}[c]{@{}c@{}}
        \textbf{MPC16-SC}
    \end{tabular}
     & \texttt{Statement coverage} \\
  \hline
  \rowcolor{gray!10}
    \begin{tabular}[c]{@{}c@{}}
        \textbf{MPC17-BC}
    \end{tabular}
     & \texttt{Branch coverage} \\
  \hline
  \rowcolor{gray!10}
    \begin{tabular}[c]{@{}c@{}}
        \textbf{MPC18-CC}
    \end{tabular}
     & \texttt{Condition coverage} \\
  \hline
  \rowcolor{gray!10}
    \begin{tabular}[c]{@{}c@{}}
        \textbf{MPC12-LOC}
    \end{tabular}
     & \texttt{Linee di codice} \\
  \hline
  \end{longtable}



\subsubsubsection{Strumenti}
\begin{itemize}
    \item \emph{\textbf{Visual Studio Code}}: IDE utilizzato dal gruppo per la $\textit{codifica}_G$ del prodotto $\textit{software}_G$.
\end{itemize}
\newpage

\section{Processi di supporto}
\subsection{Documentazione}
\subsubsection{Obiettivi}

\subsubsection{Scopo e struttura Repository Project 14}
Questo $\textit{repository}_G$ ha due funzioni:

\begin{itemize}
    \item mantenere una versione aggiornata dei sorgenti atti a produrre documentazione (file .tex)
    \item disporre di un $\textit{sistema}_G$ per produrre automaticamente documentazione (file in formato .pdf) a partire dai sorgenti.
\end{itemize}

In questo repository, infatti vi sono due cartelle principali:
\begin{itemize}
    \item sorgenti: contiene i file sorgente della documentazione (.tex) 
    \item documenti: contiene i risultati della compilazione dei file sorgente (file .pdf)
\end{itemize}

La cartella documenti deve avere la seguente struttura:
\begin{itemize}
    \item n\_periodo - PERIODO/
    \begin{itemize}
        \item documenti relativi al periodo
        \item verbali/
        \begin{itemize}
            \item verbali\_interni/    
            \item verbali\_esterni/
        \end{itemize}
    \end{itemize}
\end{itemize}

Un $\textit{verbale}_G$ deve essere nominato nel modo seguente: $\textit{verbale}_G$\_AAAA\_MM\_GG.

\subsubsection{Scopo e struttura Repository RAMtastic6.github.io}
Questo $\textit{repository}_G$ ha una funzione principale, ovvero quella di permettere una navigazione intuitiva all'interno della documentazione del gruppo; essa non è altro che una copia della cartella documenti del repo Project 14.

\subsubsection{Tipologie di documenti}
I documenti prodotti possono essere classificati in due classi principali: ad uso interno e ad uso esterno; la prima categoria comprende:
\begin{itemize}
    \item $\textit{Verbali interni}_G$ (i quali non necessitano di versionamento)
    \item Norme di progetto
\end{itemize}
La seconda categoria di documenti comprende:
\begin{itemize}
    \item Verbali esterni
    \item Piano di qualifica
    \item Piano di progetto
    \item Analisi dei requisiti
\end{itemize}
\subsubsection{Ciclo di vita di un documento}
Un documento segue le seguenti fasi di produzione:
\begin{itemize}
    \item Stesura: uno o più redattori si occupano di redigere il contenuto del documento.
    \item Verifica: ad uno o più membri del gruppo, diversi da quelli che hanno redatto il documento, viene assegnato il compito di verifica del documento.
È importante sottolineare che tutti i documenti sopracitati sono ufficiali e devono essere, quindi, preventivamente approvati da verificatori designati.
    \item Approvazione: durante questa fase, il responsabile di progetto può decidere se approvare l'inclusione di un particolare documento all'interno del repository. Nel caso in cui il documento non venga approvato, si ritorna alla fase di stesura.
    Se quest'ultima fase va a buon fine, vengono aggiunte informazioni di $\textit{versionamento}_G$ secondo quanto riportato nell'apposita sezione; infine viene caricato il documento all'interno del $\textit{repository}_G$ nel $\textit{branch}_G$ develop.
\end{itemize}
\subsubsection{Template}
Il gruppo ha scelto di utilizzare $\textit{template}_G$ LaTeX per la produzione della documentazione. Per visualizzare la struttura e utilizzare i $\textit{template}_G$, è sufficiente cercarli su overleaf.

\subsubsection{Struttura di un documento}
Un documento all'interno del nostro contesto segue una struttura ben definita, le sue sezioni principali includono:
\begin{itemize}
    \item Prima pagina: contiene il nome del gruppo e informazioni in merito al documento: uso, destinatari, redattori, verificatori, versione
    \item Indice: elenco strutturato dei contenuti del documento
    \item Registro dei cambiamenti: una tabella contente informazioni di $\textit{versionamento}_G$ relative al documento attuale; queste includono: la versione, la data, l'autore, il verificatore e una breve descrizione in merito alle modifiche apportate al documento.
    \item Intestazione: all'interno di essa vi sono il nome e l'indirizzo email del gruppo.
\end{itemize}

\subsubsection{Strumenti}
Per la creazione e la gestione della struttura dei documenti è stato deciso di utilizzare Overleaf, un $\textit{editor}_G$ LaTeX online che permette la stesura collaborativa dei documenti.

\subsubsection{Struttura di un progetto in overleaf}
Affinché le automazioni per produrre $\textit{riferimenti}_G$ al glossario funzionino, un progetto in $\textit{overleaf}_G$ deve avere la struttura seguente:

\begin{itemize}
    \item main.tex : file contenente il contentuto principale o, se si vuole, tutto il contenuto del file
    \item CONTENTS/ : cartella contenente ulteriori file (.tex) che il main.tex usa per poter costruire un documento unico
\end{itemize}


\subsubsection{Versionamento}

Il $\textit{versionamento}_G$ scelto per tenere traccia dei documenti è una tripletta di numeri: x.y.z.

\begin{itemize}
    \item \textit{x} è un numero intero, che fino alla release sarà $<$ 1, e indica la versione del progetto a cui il documento fa riferimento;
    \item \textit{y} è un numero intero positivo, e rappresenta lo stato di verifica del documento;
    \item \textit{z} è un numero intero positivo, e rappresenta il singolo cambiamento apportato al file.
\end{itemize}


\subsubsection{Procedure branch develop }

Il flusso di lavoro attuale per produrre la documentazione relativa al progetto è il seguente:
\begin{enumerate}
    \item modificare i documenti in overleaf
    \item scaricare i sorgenti dei documenti scritti in $\textit{overleaf}_G$ e inserirli all'interno del $\textit{repository}_G$ locale Project14, nella cartella sorgenti.
    \item posizionarsi nella cartella Project 14 ed eseguire le automazioni mediante il comando:
        \begin{quote}
            \emph{  python3 entry\_point\_automazioni.py} 
        \end{quote}
    Il risultato di questa esecuzione produrrà dei nuovi sorgenti (.tex): in particolare:
    \begin{enumerate}
        \item Una versione aggiornata del glossario tecnico
        \item Gli stessi file che contenevano $\textit{riferimenti}_G$ al glossario tecnico dove ve ne siano.
    \end{enumerate}
\end{enumerate}

\subsubsection{Glossario Tecnico}
Il \emph{Glossario tecnico} è un documento di supporto concepito per evitare ambiguità o incomprensioni riguardanti la terminologia utilizzata in tutta la documentazione per ogni fase del progetto ed è rivolto sia ai componenti del gruppo che a committenti e ai proponenti.
Si tratta dell'unico documento da non modificare all'interno di overleaf. Infatti, il glossario tecnico viene costruito a partire da un file \emph{json} contenuto all'interno della cartella \emph{sorgenti/Glossario} del $\textit{repository}_G$ \emph{Project14}.\\
Tale file, denominato \emph{glossario.json}, è costituito da un array di oggetti; ogni oggetto è formato da un insieme di due coppie chiave-valore, in particolare vi sono:
\begin{itemize}
    \item La chiave termine: il termine che necessita di essere definito all'interno del dominio di progetto;
    \item Il valore definizione: la definizione del termine stesso.
\end{itemize}
\paragraph{Redazione}
Per poter inserire o modificare un termine nel glossario tecnico, bisogna seguire i seguenti passaggi:
\begin{itemize}
    \item Creare un $\textit{branch}_G$ nel $\textit{repository}_G$ \emph{Project14} associato al $\textit{ticket}_G$ che indica le parole da inserire o modificare all'interno del glossario;
    \item Aggiungere o modificare le parole e definizioni nel file \emph{glossario.json};
    \item modificare il $\textit{changelog}_G$ nel file \emph{build\_glossary.py} nella cartella GLOSSARY\_AUTOMATIONS;
    \item Eseguire l'automazione tramite il comando:
        \begin{quote}
            \emph{  python3 entry\_point\_automazioni.py} 
        \end{quote}
    \item Effettuare il $\textit{push}_G$ in remoto;
    \item Aprire una $\textit{pull request}_G$ dal $\textit{branch}_G$ creato al $\textit{branch}_G$ develop;
    \item Spostare il $\textit{ticket}_G$ corrispondente nella sezione "da verificare" all'interno di \emph{Jira}.
\end{itemize}
Nel momento in cui viene inserita una parola nuova all'interno del glossario bisogna segnalare al responsabile se sono presenti discrepanze tra il modo in cui è stato scritto il termine all'interno del file \emph{glossario.json} e il modo in cui è stato scritto all'interno dei documenti. Viene riportato il seguente esempio: se all'interno del glossario viene riportata la parola "\emph{Analisi dei Requisiti}" allora all'interno dei documenti tale parola deve essere riportata con le stesse lettere maiuscole e minuscole ad $\textit{eccezione}_G$ della lettera iniziale (va bene "\emph{analisi dei Requisiti}" ma non "\emph{Analisi dei requisiti}").
\paragraph{Verifica} Per poter verificare il glossario, si seguono le seguenti azioni:
\begin{itemize}
    \item Utilizzare il $\textit{branch}_G$ creato dal redattore;
    \item Verificare i termini e definizioni nel file \emph{glossario.json};
    \item Modificare il $\textit{changelog}_G$ nel file \emph{build\_glossary.py} nella cartella GLOSSARY\_AUTOMATIONS;
    \item Eseguire l'automazione tramite il comando:
        \begin{quote}
            \emph{  python3 entry\_point\_automazioni.py} 
        \end{quote}
    \item Effettuare il $\textit{push}_G$ in remoto;
    \item Spostare il $\textit{ticket}_G$ nella sezione "completato" all'interno di \emph{Jira}.
\end{itemize}
Infine, il responsabile approva la $\textit{pull request}_G$ associata.
%Per poter modificare un termine del glossario tecnico, basta modificare il contenuto del valore %"definizione"; per poter aggiungere un termine del glossario tecnico, basta aggiungere un oggetto %avente la struttura sopracitata dell'array.

\subsection{Controllo di configurazione}
\subsubsection{Versionamento}
Capire come gestire i numeri di versione.
\subsubsection{Git e Github}
Il gruppo RAMtastic6 ha scelto di utilizzare come $\textit{strumento}_G$ di $\textit{versionamento}_G$ \emph{GitHub} e di utilizzare \emph{Git} come $\textit{strumento}_G$ per collegarsi alla $\textit{repository}_G$ GitHub.
Inoltre si è scelto di utilizzare $\textit{gitflow}_G$ come flusso di lavoro il quale verrà discusso in modo dettagliato in seguito
(\href{https://git-scm.com/downloads}{Link per il download dell'installer di Git}).\\
Inoltre, a questo \href{https://rogerdudler.github.io/git-guide/index.it.html}{link} si troverà una breve guida su come utilizzare git.
In sintesi si elencano i pricipali comandi:
\begin{itemize}
    \item $\textit{git}_G$ clone \emph{link repo}\\
    questo comando copierà la $\textit{repository}_G$ di $\textit{github}_G$ in locale
    \item $\textit{git}_G$ add \emph{nome file} (oppure "." per includere tutti i file)\\
    \emph{git add} aggiunge le modifice apportate ai files del $\textit{repository}_G$, senza eseguire questo comando un file aggiunto, eliminato o modificato non verrà salvato nella $\textit{repository}_G$ remota tramite il comando \emph{git push}.
    \item $\textit{git}_G$ commit -m "messaggio" \\
    salva le modifche apportate ai files in locale associando a quello stato un messaggio
    \item $\textit{git}_G$ $\textit{push}_G$ origin \emph{origine} \\
    salva le modifiche in remoto nel $\textit{branch}_G$ specificato
    \item $\textit{git}_G$ pull \\
    permette di aggiornare la repo in locale e in caso di necessità esegue il merge
\end{itemize}
\subsubsection{Struttura del repository}
La strutta della $\textit{repository}_G$ per i documenti deve essere:
\begin{itemize}
    \item documenti
    \begin{itemize}
        \item CANDIDATURA
        \item RTB
        \item PB
    \end{itemize}
    \item diari\_di\_bordo
    \item documenti\_interni
\end{itemize}
\subsubsection{Controllo di Flusso}
Il gruppo RAMtastic6 ha deciso di dotarsi di $\textit{Gitflow}_G$ come $\textit{sistema}_G$ di controllo del flusso di lavoro, motivato dalla sua facilità d'uso e dalle potenzialità di gestione offerte per il repository. Con una lieve modifica nei comandi per l'esecuzione dei commit, come illustrato in questa \href{https://www.atlassian.com/git/tutorials/comparing-workflows/gitflow-workflow}{guida su $\textit{Gitflow}_G$}, è possibile automatizzare il $\textit{processo}_G$ di creazione, gestione e chiusura di una $\textit{feature}_G$. Ulteriori dettagli su come gestire le $\textit{feature}_G$ sono disponibili a questo \href{http://danielkummer.github.io/git-flow-cheatsheet/}{link}.
\paragraph{Gestione dei Documenti}
Unua particolare attenzione in tal senso è rivolta alla documentazione. Al fine di mantenere nel $\textit{repository}_G$ solamente i $\textit{PDF}_G$ dei documenti prodotti, è stato deciso di adottare la piattaforma \href{https://www.overleaf.com/project}{Overleaf} per la stesura in LaTeX dei documenti e la successiva verifica. Ogni volta che un documento viene redatto o aggiornato, verificato e portato alla versione corretta come precedentemente indicato, può essere comodamente convertito in formato $\textit{PDF}_G$ tramite Overleaf. Successivamete, il documento può essere caricato nella $\textit{repository}_G$, con il $\textit{push}_G$ diretto sul $\textit{branch}_G$ \textit{develop}, soprattutto quando si parla di documentazione importante e la cui stesura è in itinere.
\subsection{Gestione della qualità}
\subsubsection{Descrizione}
La gestione della qualità è un insieme di processi che hanno lo scopo di garantire che il software, gli artefatti e i processi nel ciclo di vita del progetto aderiscano degli standard di qualità rispetto a requisiti specificati al fine di soddisfare le aspettative del proponente e degli utenti finali.
\subsubsection{Obiettivi}
La gestione della qualità si propone di raggiungere i seguenti obiettivi:
\begin{itemize}
    \item Realizzare un prodotto di qualità, in linea con le richieste del proponente;
    \item Ridurre al minimo i $\textit{rischi}_G$ che potrebbero influire sulla qualità del prodotto;
    \item Rispettare il budget preventivato del progetto.
\end{itemize}
Gli strumenti utilizzati, per la gestione della qualità dei processi e del prodotto e per valutare il lavoro svolto, sono delle metriche definite nel documento di \emph{Piano di Qualifica}. 
\subsubsection{Codifica delle metriche}
Ogni metrica è identificata dal seguente formato di codice:
\[
\text{M[Tipo][Id]-[Acronimo]}
\]

Dove:
\begin{itemize}
    \item \textbf{M} sta per "Metrica"
    \item \textbf{Tipo} può essere PC (per un processo) o PD (per un prodotto)
    \item \textbf{Id} rappresenta un identificativo all'interno di una metrica di un certo tipo
    \item \textbf{Acronimo} indica l'acronimo del nome della metrica utilizzata
\end{itemize}
Per ciascuna metrica vengono fornite delle descrizioni; inoltre per ogni tipo di $\textit{processo}_G$ viene fornita una tabella avente: il codice della metrica, il nome della metrica, valori accettabili e valori preferibili.

\subsection{Verifica}
\subsubsection{Descrizione}
La verifica del $\textit{software}_G$ è un $\textit{processo}_G$ che valuta il prodotto durante le varie fasi del progetto, dalla progettazione alla manutenzione. Essa mira a garantire che il $\textit{software}_G$ sia conforme alle aspettative e ai requisiti specificati fondandosi su criteri come coerenza, completezza e correttezza dei risultati.
\subsubsection{Obiettivi}
Il $\textit{processo}_G$ di verifica si propone di raggiungere i seguenti obiettivi:
\begin{itemize}
    \item Assicurarsi che il prodotto mantenga una buona qualità nel corso del suo sviluppo;
    \item Individuare errori e anomalie prima di proseguire con lo sviluppo del progetto.
\end{itemize}
Nel documento \emph{"Piano di Qualifica"} vengono definiti gli obiettivi da raggiungere e i criteri di accettazione  che saranno impiegati per condurre il $\textit{processo}_G$ di verifica in modo accurato ed efficiente. 
\subsubsection{Analisi statica}
L'analisi statica è una metodologia di verifica che prescinde dall'esecuzione del prodotto e che si basa su una revisione del codice e della documentazione. Lo scopo principale di questa analisi è quello di verificare l'assenza di difetti e la conformità ai requisiti e alle specifiche richieste. \\
L'analisi statica adotta comunemente due metodi di lettura:
\begin{itemize}
    \item \textbf{Walkthrough}: si tratta di una tecnica collaborativa che coinvolge il verificatore e l'autore del prodotto e che consiste nel revisionare nel suo complesso il codice e la documentazione forniti, con una successiva discussione degli eventuali problemi trovati;
    \item \textbf{Inspection}: si tratta di una tecnica che consiste nel revisionare parti specifiche del codice e della documentazione attraverso liste di controllo (\emph{checklist}) nel momento in cui si ha già un'idea di dove possano esserci possibili problemi in modo da intervenire tempestivamente e sistematicamente.
\end{itemize}
Nel documento \emph{"Piano di Qualifica"} vengono definite delle liste di controllo in modo da applicare la tecnica dell'\textit{inspection}, preferibile a quella del \textit{walkthrough}.
\subsubsection{Analisi dinamica}
L'analisi dinamica è una metodologia di verifica che si basa sull'esecuzione del codice. Le tecniche principali utilizzate in questa fase sono i $\textit{test}_G$ (definiti nel documento di "\emph{Piano di Qualifica}") finalizzati per individuare e verificare il comportamento del prodotto software.

\newpage
\section{Processi organizzativi}
\subsection{Ruoli di progetto}
In questa sezione viene riportata una breve descrizione dei ruoli e delle responsabilità dei membri di un gruppo dedicato allo sviluppo di un qualsiasi tipo di \textit{project}. 
\subsubsection{Responsabile di Progetto}
Il Responsabile di Progetto è la figura professionale, punto di $\textit{riferimento}_G$ sia per il committente sia per il fornitore, con lo scopo di
mediare tra le due parti. Assume la responsabilità delle decisioni del gruppo dopo averle approvate.\\
Le sue responsabilità includono:
\begin{itemize}
    \item Approvare l’emissione della documentazione;
    \item Approvare l’offerta economica sottoposta al committente;
    \item Pianificare e coordinare le attività di progetto;
    \item Gestire le risorse umane;
    \item Studiare e gestire i $\textit{rischi}_G$.
    \item Chiedere l'approvazione dei verbali alle persone esterne che hanno partecipato
    \item Assegnare le attività, tramite il $\textit{sistema}_G$ di tracking issues fornito da $\textit{Jira}_G$, ai membri che le dovranno svolgere
    \item Gestire le \textit{milestones} e fissarne di nuove, o modificare quelle attuali in base all'andamento del team
\end{itemize}
\subsubsection{Amministratore di Progetto}
L'Amministratore di Progetto è responsabile delle procedure di controllo e amministrazione dell’ambiente di
lavoro, con piena responsabilità sulla capacità operativa e sull’efficienza.\\
In particolare, si occupa di:
\begin{itemize}
    \item Ricercare, studiare e mettere in opera risorse per migliorare l’ambiente di lavoro, automatizzandolo quando possibile;
    \item Risolvere problemi legati alla gestione dei processi;
    \item Salvaguardare la documentazione di progetto;
    \item Effettuare il controllo di versioni e configurazioni del prodotto $\textit{software}_G$;
    \item Redigere e attuare i piani e le procedure per la gestione della qualità.
\end{itemize}
\subsubsection{Analista}
L'Analista è una figura con maggiori competenze riguardo al dominio applicativo del problema. \\
Le sue responsabilità includono:
\begin{itemize}
    \item Studiare il problema e il relativo contesto applicativo;
    \item Comprendere il problema e definire la complessità e i requisiti;
    \item Redigere il documento "Analisi dei Requisiti";
    \item Studiare i casi d'uso e redigere il loro relativo schema $\textit{UML}_G$.
\end{itemize}
\subsubsection{Progettista}
Il Progettista gestisce gli aspetti tecnologici e tecnici del progetto. \\
In particolare, si occupa di:
\begin{itemize}
    \item Effettuare scelte riguardanti gli aspetti tecnici e tecnologici del progetto, favorendone l'efficiacia e l'efficienza;
    \item Definire un'architettura del prodotto da sviluppare che miri all'economicità e alla manutenibilità a partire dal lavoro svolto dall'Analista;
    \item Redigere la "Specifica Tecnica" e la parte pragmatica del "$\textit{Piano di Qualifica}_G$".
\end{itemize}
\subsubsection{Verificatore}
Il Verificatore è responsabile della sorveglianza sul lavoro svolto dagli altri componenti del gruppo, sulla base delle proprie competenze tecniche, esperienza e conoscenza delle norme. \\
In particolare, si occupa di:
\begin{itemize}
    \item Esaminare i prodotti in fase di revisione, con l'ausilio delle tecniche e degli strumenti definiti nel presente documento;
    \item Verificare la conformità dei prodotti ai requisiti funzionali e di qualità;
    \item Verificare i documenti segnalando eventuali errori.
    \item Caricare i verbali, dopo averli verificati, all'interno della $\textit{repository}_G$ GitHub nel ramo \textit{develop}.
\end{itemize}
\subsubsection{Programmatore}
Il Programmatore è incaricato della $\textit{codifica}_G$ del progetto e delle componenti di supporto che verranno utilizzate per eseguire prove di verifica e di validazione del prodotto. \\
In particolare, si occupa di:
\begin{itemize}
    \item Implementare la "Specifica Tecnica" redatta dal Progettista;
    \item Scrivere un codice pulito e facilmente mantenibile che rispetti le norme definite nel presente documento;
    \item Realizzare gli strumenti per la verifica e la validazione del $\textit{software}_G$;
    \item Redigere il "$\textit{Manuale Utente}_G$" relativo alla propria $\textit{codifica}_G$;
    \item Redigere i verbali delle riunioni interne ed esterne del team.
\end{itemize}

\subsection{Gestione di progetto}

\subsubsection{Allineamento organizzativo}
Per l'allineamento e il coordinamento delle attività sono stati scelti due tipi di comunicazione:
\begin{itemize}
    \item Interne: coinvolge tutti i membri del gruppo;
    \item Esterne: coinvolge proponenti e committenti.
\end{itemize}

\subsubsection{Comunicazioni interne}
Le comunicazioni interne avvengono tramite le applicazioni:
\begin{itemize}
    \item \textit{Telegram}: permette una comunicazione veloce tra i membri del gruppo e viene utilizzato principalmente per organizzare incontri interni o discutere di eventuali quesiti.
    \item \textit{Discord}: viene utilizzato dai membri del gruppo per tenere incontri interni, in quanto dispone di un canale vocale; inoltre, permette la condivisione di contenuti multimediali in streaming video.
\end{itemize}

\subsubsection{Comunicazioni esterne}
Le comunicazioni esterne vengono gestite dal Responsabile di Progetto tramite l’utilizzo dei seguenti
canali:
\begin{itemize}
    \item  Posta Elettronica: tramite l’indirizzo e-mail del gruppo \textit{ramtastic6@gmail.com}.
    \item  \textit{Telegram}: permette di instaurare un canale veloce di comunicazione con la proponente.
\end{itemize}

\subsection{Gestione organizzazione del lavoro}
\subsubsection{Modello di sviluppo}
Il gruppo, al fine di minimizzare ritardi e massimizzare lo svolgimento delle proprie attività, ha deciso
di implementare il $\textit{framework}_G$ \textit{Scrum}, che permette di dividere il tempo di lavoro in intervalli piccoli, frammentati all’interno di un periodo di circa due settimane, definito come "\textit{sprint}". In particolare, all’interno di
questo, vengono svolte un numero di attività commisurate che devono essere completate entro i suoi termini. Individuiamo dunque una serie di fasi:
\begin{itemize}
    \item \textbf{sprint planning}: in questa fase, si discutono gli obbiettivi da raggiungere all'interno dello \textit{sprint}; vengono quindi pianificate le attivitá da svolgere durante lo \textit{sprint} in funzione di questi, facendo attenzione alle disponibilitá di ogni membro e preparando un $\textit{preventivo}_G$ per il periodo; è in questa fase, inoltre, che viene svolto il cambio dei ruoli.
    Lo \textit{sprint planning} consiste in una riunione tenuta tramite \textit{Discord} che si svolge all'inizio dello \textit{sprint} e richiede la partecipazione di tutti i membri del gruppo.
    
    \item \textbf{sprint review}: in questa fase, si discutono gli obiettivi da raggiunti nello \textit{sprint}. Alla fine, dovrebbe essere prodotto almeno un incremento, cioè un $\textit{software}_G$ utilizzabile. In particolare, in questa fase vengono definiti a $\textit{consuntivo}_G$ le risorse impiegate commisurate agli obiettivi, questi ultimi suddivisi in raggiunti e non raggiunti, capendo cosa va migliorato e cosa non è stato fatto per poter aggiungere ulteriori obiettivi per lo \textit{sprint} successivo;
    
    \item \textbf{sprint retrospective}: fase che conclude definitivamente lo \textit{sprint} appena svolto, avente l'obiettivo di valutarne l'andamento generale e cercando di capire cosa è stato fatto bene e cosa può essere migliorato. In questo modo, è meglio definire come ripianificare le attività, decidendo come iniziare/continuare/concludere attività presenti o future da realizzare.
\end{itemize}

\subsubsection{Rotazione dei ruoli}
\'E prevista una rotazione dei ruoli all'interno del gruppo a cadenza periodica.
L’attribuzione dei ruoli viene svolta secondo i seguenti criteri:
\begin{itemize}
    \item Equità;
    \item Disponibilità;
    \item Assenza di conflitti.
\end{itemize}


\subsubsection{Gestione delle attività}
Per gestire le attività da istanziare all'interno del $\textit{framework}_G$ \textit{Scrum}, è stato deciso di utilizzare \textit{Jira}. Un'attività, definita come \textit{ticket} all'interno di \textit{Jira}, ha un ciclo di vita; esso si compone di quattro stati:

\begin{itemize}
    \item \textbf{Da completare}: l'attività è stata istanziata e assegnata ad un membro del gruppo; è stata inoltre definita una stima in termini temporali associata ad essa.
    \item \textbf{In corso}: l'attività è entrata nella fase di svolgimento; all'interno di questa fase, il membro al quale è stata assegnata ha il compito di dover inserire le informazioni di tracciamento temporale associate al \textit{ticket}; per fare ciò, basta selezionare un \textit{ticket} e andare a modificare il campo \textit{Tracciamento temporale}, inserendo il tempo impiegato fino ad ora per svolgere l'attività.
    \item \textbf{In fase di verifica}: durante questa fase, il Verificatore si assicura che l'attività sia stata svolta correttamente e secondo dei criteri di qualità attesa (al momento, ancora da specificare).
    \item \textbf{Completata}: l'attività è stata verificata e approvata dal Responsabile di Progetto.
\end{itemize}
Il passaggio di un \textit{ticket} da uno stato all'altro è responsabilità del membro al quale è stato assegnato, eccetto per lo stato di completamento; infatti, spetta al Responsabile di Progetto decidere quando un'attività risulti o meno completata.

\subsection{Infrastruttura}
Strumenti per la gestione delle attività di progetto e procedure legate ad essi.
\subsection{Miglioramento}
Durante lo svolgimento delle attività e successiva stesura dei documenti, il gruppo si impegna ad operare secondo il \textit{principio di miglioramento continuo}, al fine di individuare facilmente attività, ruoli e possibili miglioramenti, cercando nuove o diverse soluzioni alle problematiche insorte.
\subsection{Formazione}
Al fine di operare continuativamente un miglioramento sulle attività svolte e proseguire nel mantenimento corretto delle attività in modo asincrono, occorre da parte di tutti i membri del gruppo lo studio in autonomia delle $\textit{tecnologie}_G$ e delle modalità operative presenti, al fine di velocizzare il $\textit{processo}_G$ di formazione e di conoscenza degli strumenti utilizzati. Si listano, al fine di
completezza, alcune documentazioni utilizzate durante lo sviluppo, sia a livello documentale che organizzativo:

\begin{itemize}
    \item \textit{GitHub}: https://docs.github.com/
    \item \textit{Jira}: https://confluence.atlassian.com/jira
    \item \textit{Git}: https://docs.github.com/en/get-started/using-git/about-git
    \item \textit{Scrum}: https://scrumguides.org/scrum-guide.html
\end{itemize}
\end{document}