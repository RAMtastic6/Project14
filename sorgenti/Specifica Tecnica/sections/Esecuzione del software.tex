\section{Esecuzione del software}
\subsection{Download del progetto}
Clonare il $\textit{repository}_G$ del progetto usando il comando: 
\begin{lstlisting}[language=bash]
git clone https://github.com/RAMtastic6/EasyMeal.git 
cd EasyMeal
\end{lstlisting}

\subsection{Avvio dell'applicazione} 
\subsubsection{Docker (consigliato)}
Questo progetto utilizza $\textit{Docker}_G$ Compose per gestire l'avvio dei container $\textit{Docker}_G$. 
Per avviare i container e utilizzare l'applicazione si seguando le seguenti istruzioni.

\begin{itemize}
\item \textbf{Avvio dei container} \\
Per avviare i container utilizzando $\textit{Docker}_G$ Compose, esegui il seguente comando nella directory del progetto:
\begin{lstlisting}[language=bash]
docker-compose up --build --watch
\end{lstlisting}

Oppure per far partire i servizi senza che la console di comando attendi la chiusura si usi il comando:
\begin{lstlisting}[language=bash]
docker-compose up -d
\end{lstlisting}

Una volta avviati i container, si potrà accedere all'applicazione utilizzando il browser o gli strumenti di sviluppo appropriati. \\ Per esempio per poter accedere al progetto NextJS ci si colleghi al link: 
\begin{quote}
\begin{verbatim}
http://localhost:3000/create_reservation 
\end{verbatim}
\end{quote}
Si tenga presente che NextJS utilizza la porta 3000, NestJS 6969, Postgres utilizza 7070 e Socket 8000.
\\
\item \textbf{Modifica dei file Dockerfile e compose.yaml} \\
Per applicare le modifiche e far partire il progetto si usi il comando:
\begin{lstlisting}[language=bash]
docker-compose up --build
\end{lstlisting}

\end{itemize}

\newpage
\subsubsection{NPM}
Senza l'utilizzo di $\textit{Docker}_G$ si devono installare manualmente le seguenti $\textit{tecnologie}_G$:
\begin{itemize}
    \item Node.js
    \item $\textit{npm}_G$ (Node Package Manager)
    \item postgresSQL
\end{itemize}
Si ricorda di importare il dump del database nel proprio computer, per farlo si può usare pgAdmin.\\ \\ Ora si eseguano le seguenti istruzioni:
\begin{enumerate}
\item Installare le dipendenze per il $\textit{backend}_G$ $\textit{Nest.js}_G$:
\begin{lstlisting}[language=bash]
cd nest-js
npm install
\end{lstlisting}

\item Avviare il server $\textit{backend}_G$ $\textit{Nest.js}_G$:
\begin{lstlisting}[language=bash]
npm run start:dev
\end{lstlisting}

\item Installare le dipendenze per il $\textit{frontend}_G$ $\textit{Next.js}_G$: Aprire una nuova shell lasciando la precedente in esecuzione
\begin{lstlisting}[language=bash]
cd next-js
npm install
\end{lstlisting}

\item Avviare il server $\textit{frontend}_G$ $\textit{Next.js}_G$:
\begin{lstlisting}[language=bash]
npm run dev
\end{lstlisting}

\item Installare le dipendenze per il progetto socket $\textit{Nest.js}_G$:
\begin{lstlisting}[language=bash]
cd websocket-server
npm install
\end{lstlisting}

\item Avviare il server socket $\textit{Nest.js}_G$:
\begin{lstlisting}[language=bash]
npm run start:dev
\end{lstlisting}

\item Accedere all'applicazione:
Una volta avviati sia il server $\textit{backend}_G$ $\textit{Nest.js}_G$ che il server $\textit{frontend}_G$ $\textit{Next.js}_G$, è possibile accedere all'applicazione utilizzando il browser. \\ Si apra il browser e si vada all'indirizzo 
\begin{quote}
\begin{verbatim}
    $\textit{http}_G$://localhost:3000/create_reservation
\end{verbatim}
\end{quote}
per accedere alla pagina di creazione delle prenotazioni.

\end{enumerate}

\subsection{Esecuzione dei test}
In questa sezione si elencano i comandi per eseguire i $\textit{test}_G$ tramite $\textit{jest}_G$ per tutti i servizi.
Per poter eseguire i $\textit{test}_G$ in locale per i tre servizi si dovrà prima installare $\textit{Node.js}_G$ versione 21.x . \href{https://nodejs.org/en/download/package-manager/current}{$\textit{Node.js}_G$ installazione}(consultato: 2024-05-27) \\
I comandi per eseguire i $\textit{test}_G$ dei vari servizi sono gli stessi, si deve solo cambiare la cartella dove vengono eseguiti i comandi:
\begin{itemize}
\item Per i $\textit{test}_G$ lato $\textit{backend}_G$:
\begin{lstlisting}[language=bash]
cd nest-js
\end{lstlisting}
\item Per i $\textit{test}_G$ lato $\textit{frontend}_G$:
\begin{lstlisting}[language=bash]
cd next-js
\end{lstlisting}
\item Per i $\textit{test}_G$ del $\textit{websocket}_G$:
\begin{lstlisting}[language=bash]
cd websocket-server
\end{lstlisting}
\end{itemize}
Quindi, di seguito, vengono riportati i comandi per eseguire i $\textit{test}_G$:
\begin{enumerate}
\item Eseguire il comando, se non fatto in precedenza:
\begin{lstlisting}[language=bash]
npm install
\end{lstlisting}
\item Eseguire il seguente script per l'esecuzione di tutti i $\textit{test}_G$ generando un report finale:
\begin{lstlisting}[language=bash]
npm run $\textit{test}_G$:cov
\end{lstlisting}
\item Aprire il seguente file per visualizzare il report:
\begin{lstlisting}[language=bash]
./coverage/coverage.txt
\end{lstlisting}
\end{enumerate}