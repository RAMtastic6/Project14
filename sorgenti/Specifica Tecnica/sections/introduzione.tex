\section{Introduzione}
\subsection{Scopo del documento}
Il presente documento ha lo scopo di descrivere in modo dettagliato le scelte progettuali effettuate dal gruppo RAMtastic6 per la realizzazione del $\textit{sistema}_G$ $\textit{Easy Meal}_G$ richiesto dall'azienda $\textit{Imola Informatica}_G$. Viene compresa l'$\textit{architettura}_G$ logica e l'$\textit{architettura}_G$ di $\textit{deployment}_G$ oltre che la lista delle $\textit{tecnologie}_G$ utilizzate e i $\textit{design}_G$ pattern adottati. \\
Inoltre, viene fornita una sezione relativa al tracciamento dei requisiti soddisfatti in linea con il documento di Analisi dei Requisiti. \\
Eventuali termini tecnici sono definiti all'interno del documento "Glossario Tecnico".

\subsection{Scopo del prodotto}
Il prodotto finale, realizzato tramite un'$\textit{Applicazione Web Responsive}_G$, si propone di realizzare un $\textit{software}_G$ innovativo volto a semplificate il $\textit{processo}_G$ di $\textit{prenotazione}_G$ e $\textit{ordinazione}_G$ nei ristoranti, contribuendo a migliorare l'esperienza per clienti e ristoratori. In particolare, $\textit{Easy Meal}_G$ dovrà consentire agli utenti di personalizzare gli ordini in base alle proprie preferenze, allergie ed esigenze alimentari; interagire direttamente con lo staff del ristorante attraverso una chat integrata; consentire di dividere il conto tra i partecipanti al tavolo.
\subsection{Riferimenti}
\subsubsection{Riferimenti normativi}
\begin{enumerate}
    \item Presentazione del $\textit{capitolato}_G$ d'appalto C3 - Progetto $\textit{Easy Meal}_G$: \\ \url{https://www.math.unipd.it/~tullio/IS-1/2023/Progetto/C3.pdf}\\
    (Consultato: 2024-05-13)
    \item Regolamento del progetto didattico: \\ \url{https://www.math.unipd.it/~tullio/IS-1/2023/Dispense/PD2.pdf}\\
    (Consultato: 2024-05-13)
    \item $\textit{Norme di Progetto}_G$ v2.0.0.
\end{enumerate}
\subsubsection{Riferimenti informativi}
\begin{enumerate}
    \item Glossario v2.0.0;
    \item Analisi dei Requisiti v3.0.0.
    \item Lezione \emph{"Progettazione e programmazione: diagrammi delle classi ($\textit{UML}_G$)"} del corso di Ingegneria del $\textit{Software}_G$:\\
    \url{https://www.math.unipd.it/~rcardin/swea/2023/Diagrammi%20delle%20Classi.pdf}\\
    (Consultato: 2024-05-13)
    \item Lezione \emph{"Progettazione: i pattern architetturali"} del corso di Ingegneria del $\textit{Software}_G$:\\
    \url{https://www.math.unipd.it/~rcardin/swea/2022/Software%20Architecture%20Patterns.pdf}\\
    (Consultato: 2024-05-13)
    \item Lezione \emph{"Progettazione: il pattern Dependency Injection"} del corso di Ingegneria del $\textit{Software}_G$:\\
    \url{https://www.math.unipd.it/~rcardin/swea/2022/Design%20Pattern%20Architetturali%20-%20Dependency%20Injection.pdf}\\
    (Consultato: 2024-05-13)
    \item Lezione \emph{"Progettazione software (T6)"} del corso di Ingegneria del $\textit{Software}_G$:\\
    \url{https://www.math.unipd.it/~tullio/IS-1/2023/Dispense/T6.pdf}\\
    (Consultato: 2024-05-13)
    \item Lezione \emph{"Progettazione: il pattern Model-View-Controller e derivati"} del corso di Ingegneria del $\textit{Software}_G$:\\
    \url{https://www.math.unipd.it/~rcardin/sweb/2022/L02.pdf}
    \item Lezione \emph{"Progettazione: i pattern creazionali (GoF)"} del corso di Ingegneria del $\textit{Software}_G$:\\
    \url{https://www.math.unipd.it/~rcardin/swea/2022/Design%20Pattern%20Creazionali.pdf}\\
    (Consultato: 2024-05-13)
    \item Lezione \emph{"Progettazione: i pattern strutturali (GoF)"} del corso di Ingegneria del $\textit{Software}_G$:\\
    \url{https://www.math.unipd.it/~rcardin/swea/2022/Design%20Pattern%20Strutturali.pdf}\\
    (Consultato: 2024-05-13)
    \item Lezione \emph{"Progettazione: i pattern di comportamento (GoF)"} del corso di Ingegneria del $\textit{Software}_G$:\\
    \url{https://www.math.unipd.it/~rcardin/swea/2021/Design%20Pattern%20Comportamentali_4x4.pdf}\\
    (Consultato: 2024-05-13)
    
\item $\textit{Nest.js}_G$ — Architectural Pattern, Controllers, Providers, and Modules: \\
  \href{https://medium.com/geekculture/nest-js-architectural-pattern-controllers-providers-and-modules-406d9b192a3a}{https://medium.com/geekculture/nest-js-architectural-pattern-controllers-providers-and-modules-406d9b192a3a} \\
  (Consultato: 2024-05-29)

\item Topic-based Publish/Subscribe $\textit{design}_G$ pattern implementation in OCaml — (Using socket programming) \\
\href{https://medium.com/@aryangodara\_19887/topic-based-publish-subscribe-design-pattern-implementation-in-ocaml-using-socket-programming-ba536f0a3ef3}{https://medium.com/@aryangodara\_19887/topic-based-publish-subscribe-design-pattern-implementation-in-ocaml-using-socket-programming-ba536f0a3ef3} \\
(Consultato: 2024-05-29)

\item $\textit{Repository}_G$ pattern explained with Laravel and NestJS examples \\
\href{https://draganatanasov.com/2023/01/15/repository-pattern-explained-with-laravel-and-nestjs-examples/}{https://draganatanasov.com/2023/01/15/repository-pattern-explained-with-laravel-and-nestjs-examples/} \\
(Consultato: 2024-05-29)

\item Boosting Your NestJS Skills: Exploring the Module Pattern $\textit{Design}_G$ \\
\url{https://www.bymoji.com/blog/Boosting\_Your\_NestJS\_Skills\_Exploring\_the\_Module\_Pattern\_Design} \\
(Consultato: 2024-05-29)

\item $\textit{Backend}_G$ for $\textit{Frontend}_G$ (BFF) Pattern in System Designing \\
\href{https://blog.bitsrc.io/backend-for-frontend-bff-pattern-in-system-designing-501a71df6bf7}{https://blog.bitsrc.io/backend-for-frontend-bff-pattern-in-system-designing-501a71df6bf7} \\
(Consultato 2024-05-29)

\item $\textit{React}_G$ $\textit{Design}_G$ Patterns \\
\href{https://refine.dev/blog/react-design-patterns}{https://refine.dev/blog/react-design-patterns} \\
(Consultato 2024-05-29)


\end{enumerate}