\section{Testing}
In questa sezione vengono esplorate le metodologie di \textit{Test_G}ing e la loro specifica. L'obiettivo è quello di seguire il modello a V in cui ad ogni fase di sviluppo corrisponde una tipologia di test da eseguire.\\
I \textit{Test_G} si dividono in:
\begin{itemize}
    \item \textbf{Test di unità}:\\
    vengono eseguiti sulle unità piu' semplici del codice. Viene fatta corrispondere all'attività di \textit{Codifica_G} (implementation).
    \item \textbf{Test di integrazione}:\\
    vengono eseguiti per verificare la corretta integrazione tra le diverse unità software. Viene fatta corrispondere all'attività di progettazione.
    \item \textbf{Test di sistema}:\\
    verificano il corretto funzionamento dell'intero \textit{Sistema_G} e, in particolare, che tutti i requisiti individuati siano soddisfatti. Viene fatta corrispondere all'attività di analisi dei requisiti. 
    \item \textbf{Test di accettazione}:\\
    verificano, alla presenza del committente, che il prodotto finale soddisfi tutti i requisiti. Se superati, si può procedere al \textit{Rilascio_G} dello stesso.
\end{itemize}