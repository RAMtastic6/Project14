\section{Introduzione}
\subsection{Scopo del documento}
Il presente documento si propone di definire le metriche e le metodologie di controllo e misurazione necessarie per garantire la qualità del prodotto e del $\textit{processo}_G$. In particolare, le metriche di valutazione del prodotto sono correlate ai requisiti e alle aspettative del fornitore.
Il $\textit{Piano di Qualifica}_G$ è concepito per essere dinamico ed incrementale, in particolar modo per quanto riguarda le metriche descritte e mira a fornire una valutazione il più obiettiva possibile di ciò che è stato realizzato.\\
Le procedure del $\textit{Way Of Working}_G$ devono essere costantemente osservate e migliorate, al fine di garantire che il prodotto soddisfi le aspettative del cliente e mantenga gli standard di qualità richiesti. Eventuali termini tecnici sono definiti all'interno del documento "Glossario Tecnico".
\subsection{Scopo del prodotto}
Il prodotto finale, realizzato tramite un'$\textit{Applicazione Web Responsive}_G$, si propone di realizzare un $\textit{software}_G$ innovativo volto a semplificate il $\textit{processo}_G$ di $\textit{prenotazione}_G$ e $\textit{ordinazione}_G$ nei ristoranti, contribuendo a migliorare l'esperienza per clienti e ristoratori. In particolare, \textit{Easy Meal} dovrà consentire agli utenti di personalizzare gli ordini in base alle proprie preferenze, allergie ed esigenze alimentari; interagire direttamente con lo staff del ristorante attraverso una chat integrata e in ultimo, consentire di dividere il conto tra i partecipanti al tavolo.
\subsection{Riferimenti}
\subsubsection{Riferimenti normativi}
\begin{enumerate}
    \item $\textit{Norme di Progetto}_G$ v2.0.0
    \item Presentazione del $\textit{capitolato}_G$ d'appalto C3 - Progetto $\textit{Easy Meal}_G$: \\ 
    \url{https://www.math.unipd.it/~tullio/IS-1/2023/Progetto/C3.pdf}
    \item Regolamento del progetto didattico: \\ 
    \url{https://www.math.unipd.it/~tullio/IS-1/2023/Dispense/PD2.pdf}
\end{enumerate}
\subsubsection{Riferimenti informativi}
\label{sec:rif_inf}
\begin{enumerate}
    \item Lezione \emph{"Progettazione $\textit{software}_G$ (T6)"} del corso di $\textit{Ingegneria del $\textit{software}_G$}_G$: \\
    \url{https://www.math.unipd.it/~tullio/IS-1/2023/Dispense/T6.pdf}
    \item Lezione \emph{"Qualità del $\textit{software}_G$ (T7)"} del corso di $\textit{Ingegneria del $\textit{software}_G$}_G$: \\
    \url{https://www.math.unipd.it/~tullio/IS-1/2023/Dispense/T7.pdf}
    \item Lezione \emph{"Qualità di $\textit{processo}_G$ (T8)"} del corso di $\textit{Ingegneria del software}_G$: \\
    \url{https://www.math.unipd.it/~tullio/IS-1/2023/Dispense/T8.pdf}
    \item Lezione \emph{"Verifica e validazione: introduzione (T9)"} del corso di $\textit{Ingegneria del software}_G$: \\
    \url{https://www.math.unipd.it/~tullio/IS-1/2023/Dispense/T9.pdf}
    \item Lezione \emph{"Verifica e validazione: analisi statica (T10)"} del corso di $\textit{Ingegneria del software}_G$: \\
    \url{https://www.math.unipd.it/~tullio/IS-1/2023/Dispense/T10.pdf}
    \item Lezione \emph{"Verifica e validazione: analisi dinamica (T11)"} del corso di $\textit{Ingegneria del software}_G$: \\
    \url{https://www.math.unipd.it/~tullio/IS-1/2023/Dispense/T11.pdf}
     \item Documento \emph{"Dichiarazione impegni v1.2"}: \\ \url{https://github.com/RAMtastic6/Project14/blob/main/documenti/CANDIDATURA/documento_impegni_v1.2.pdf}
     \item Metriche di progetto (\emph{Earned Value Analysis}):\\
     \url{https://it.wikipedia.org/wiki/Metriche_di_progetto}
     \item Glossario v2.0.0;
     \item Analisi dei Requisiti v3.0.0.
\end{enumerate}
\subsection{Codifica delle metriche}
In questa sottosezione verranno definite le metriche che utilizzeremo, utilizzando un codice standardizzato.

Una metrica è identificata dal seguente formato di codice:
\[
\text{M[Tipo][Id]-[Acronimo]}
\]

Dove:
\begin{itemize}
    \item \textbf{M} sta per "Metrica"
    \item \textbf{Tipo} può essere PC (per un $\textit{processo}_G$) o PD (per un prodotto)
    \item \textbf{Id} rappresenta un identificativo all'interno di una metrica di un certo tipo
    \item \textbf{Acronimo} indica l'acronimo del nome della metrica utilizzata
\end{itemize}

Per ciascuna metrica verranno fornite descrizioni, valori accettabili e valori preferibili.
\subsection{Codifica dei test}
In questa sottosezione verranno definiti i $\textit{test}_G$ che utilizzeremo, utilizzando un codice standardizzato.

Un $\textit{test}_G$ è identificato dal seguente formato di codice:
\[
\text{T[Tipo]-[Id]}
\]

Dove:
\begin{itemize}
    \item \textbf{T} sta per "$\textit{Test}_G$"
    \item \textbf{Tipo} può essere S (di $\textit{sistema}_G$) o I (di integrazione) o U (di unità) oppure A (di accettazione)
    \item \textbf{Id} rappresenta un identificativo all'interno di un $\textit{test}_G$ di un certo tipo
\end{itemize}

Per ciascun $\textit{test}_G$ verranno fornite descrizioni e il loro stato se implementato o meno oltre che un loro tracciamento.